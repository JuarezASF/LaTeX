\section{Introdução}

 Prática comum no mundo da automação industrial é o controle de reações químicas em larga escala. Tipicamente 
é necessário controlar a vazão dos componente da reação assim como  a temperatura e a pressão no reator para que o processo
ocorra segundo as condições especificadas. Essa automação é hoje feita com o auxílio dos controladores lógicos programáveis(CLP).

A automação deve implementar uma máquina de estados seguindo uma sequência de estados conforme a receita da reação. Normalmente a descrição do problema não virá em uma máquina de estados, já que 
essa não é uma técnica conhecida pelos químicos, mas sim numa receita ou passos a serem seguidos. É trabalho do engenheiro
então interpretar essas instruções e construir o diagrama de estados equivalente. Um dos métodos de se definir a relação 
sequencial entre os estados é a modelagem por redes de Petri, que deve levar em consideração tanto os estados de controle como
os estados da planta. Ao final da modelagem de Petri o engenheiro de automação
tem todas as informações na mão para começar a programação.

Dentre as linguagens utilizadas para a implementação do controle por meio de CLP's encontra-se o sequential functional
chart(SFC). Esta linguagem assemelha-se muito a modelagem por redes de Petri e a tradução do modelo para o programa é quase
que direta. A desvantagem dessa técnica é que o programa gerado após a compilação é grande em memória. Isso acontece pela linguagem
ser de alto nível e envolver diversas estruturas de dados complexas em sua definição. Outra linguagem também comumente
encontrada é o Ladder. Essa linguagem assemelha-se com os antigos quadros de relés e possuir estruturas de mais baixo nível que o SFC.
Como resultado, o programa compilado é menor. Os controladores de hoje possuem grandes quantidades de memória, mas quando utilizamos controladores mais simples ou ainda quando a automação cresce muito, memória pode vir a ser um problema e fator determinante na
escolha da linguagem utilizada.

\section{Descrição do Problema}
Consideramos o sistema na figura a seguir:
\begin{figure}[H]
	\centering
	\includegraphics[scale=1.0]{../images/planta.png}
	\caption{Planta Automatizada}
	\label{fig:planta}
\end{figure}

Inicialmente descrevemos os elementos da planta e seu funcionamento. A planta possui: 4 válvulas, 3 sensores de nível (bóias),
1 misturador, 1 sensor de presença e um botão liga. A válvula 1 quando aberta libera vapor de água no reator e causa um aumento
de temperatura. A válvula 2 quando acionada libera o fluxo da substância 1. A válvula 3 controla o fluxo de uma substância 3.
A princípio o controle das válvulas 2  e  3 não influenciam a temperatura do reator. O misturador quando acionado cataliza a reação
e pode ser que esta, se exotérmica, aumente a temperatura no sistema. O sensor de presença simplesmente detecta o posicionamento ou 
não de um tambor. Existem ainda 4 luzes de alarme: uma para cada válvula e uma para informar superaquecimento do reator. Durante
todo o processo somente uma válvula pode ser acionada por vez, por isso quando dizemos que a válvula 1 é acionada está implícito
que todas as outras não o estão.


Descrevemos agora a sequência de etapas normais do processo, isto é , aquilo que deve acontecer se tudo der certo. 


\begin{itemize}
	\item \textbf{Sequência Normal}
	\begin{itemize}
		\item O processo se inicia quando um sensor de presença detecta o correto posicionamento de um container abaixo
		do reator e um botão pulsador de início é pressionado
		\item abre-se a válvula 1 e espera-se que a temperatura aumente até 100ºC
		\item abre-se a válvula 2 e espera-se que o sensor de nível 2 seja acionado(pelo posicionamento, é claro que
		o  sensor 3 será acionado antes que o sensor 3 o seja)
		\item abre-se a válvula 3 e espera-se que o sensor de nível 1 seja atingido.
		\item abre-se a válvula 1 para aumentar a temperatura até 300ºC
		\item inicia-se a mistura ao acionar o misturador por 20s
		\item desliga-se então o misturador e abre-se a válvula 4 para liberar a mistura
		\item volta-se então para o estado inicial enquanto se espera comando para um novo ciclo
	\end{itemize}
\end{itemize}

 Passamos para a descrição de erros e seu tratamento. Existem duas fontes de erro a serem monitoradas durante
o processo: o mal funcionamento das válvulas e a temperatura.
\begin{itemize}
\item \textbf{Problemas e Tratamento}
	\begin{itemize}
	\item  O mal funcionamento das válvulas pode ser detectado
pela não ativação da transição do estado atual a partir de um determinado tempo. Uma vez que o processo é conhecido e controlado, é esperado que após o acionamento de algum estado este seja logo sucedido pelo seu sucessor. Por exemplo, assim que liberamos o fluxo
da primeira substância esperamos que não demore muito para que o sensor de nível 2 seja ativado. Caso isso não ocorra em, digamos, no máximo 2 minutos sabemos que deve haver algum problema com o sistema relacionado a essa substância(ou a válvula está quebrada 
ou pode haver um vazamento na rede).
	\item   Já a temperatura do sistema pode lançar dois alarmes que influenciam o processo: a temperatura
pode cair abaixo da temperatura de processo ou pode passar desta. No primeiro caso devemos parar o que quer que seja que esteja acontecendo e abrir a válvula 1 para injetar vapor e elevar a temperatura. No segundo caso devemos fechar todas as válvulas e esperar
para que a temperatura caia. O mistura é então descartada e volta-se para o estado inicial.
	\item Na situação de qualquer emergência
o alarme respectivo deve ser acionado e somente desativado quando do pressionar do botão DesligaEmergencia. Para finalizar,
quando o sistema está em parada de emergência ele não deve aceitar o comando de iniciar até que o estado de emergência seja 
desativado.
	\end{itemize}
\end{itemize}


\section{Modelagem em Rede de Petri}

Inicialmente montou-se a seguinte tabela 1 com as especificações de transições e estados. 

\begin{figure*}
	\centering
	\includegraphics[scale=0.8]{../images/tabela}
	\caption*{Tabela 1: especificações para modelagem em rede de Petri}
	\label{fig:planta}
\end{figure*}


 A modelagem foi feita em redes de Petri utilizando o programa Visual Object Net++ e o resultado é mostrado na figura \ref{fig:planta}..
\begin{figure*}
	\centering
	\includegraphics[scale=0.6, angle = 90]{../images/petri}
	\caption{Rede de Petri para o sistema de controle e da planta}
	\label{fig:planta}
\end{figure*}

\section{Programa em SFC}
O código desenvolvido em SFC no programa RSLogix5000 é mostrado a seguir. O projeto desenvolvido está disponível da URL
\url{https://github.com/JuarezASF/Code/blob/master/UnB2014/Automa%C3%A7%C3%A3o/ProjetoReator/Automacao_ProjetoFinal_Mauricio_e_Juarez.ACD}

\section{Interface Homem Máquina}
Desenvolveu-se uma interface homem máquina utilizando o programa RSView32.

\begin{figure*}
	\centering
	\includegraphics[scale=0.6, angle = 90]{../images/IHM}
	\caption{Interface Homem Máquina}
	\label{fig:planta}
\end{figure*}