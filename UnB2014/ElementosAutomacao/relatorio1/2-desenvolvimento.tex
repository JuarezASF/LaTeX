\section{Descrição do Problema}
\paragraph{}Consideramos o sistema na figura a seguir:
\begin{figure}[H]
	\centering
	\includegraphics[scale=0.5]{../images/planta.png}
	\caption{Planta Automatizada}
	\label{fig:planta}
\end{figure}
\paragraph{}
O objetivo da automação é desenvolver um sistema baseado em esteiras, sensores e um atuador que classifique as caixas colocadas manualmente sobre a esteira em duas categorias de acordo com seu tamanho.
Para a implementação da seleção utilizaremos três esteiras transportadoras \textbf{E1, E2} e  \textbf{E3} movimentadas por três motores \textbf{M1, M2} e  \textbf{M3}. Inicialmente as caixas são colocadas pelos operários de forma manual em\textbf{ E1}. Sensores \textbf{S1} posicionados na esteira devem
então detectar a caixa e dar partida no motor \textbf{M1} que movimentará a esteira. O motor deve ficar ligado tempo suficiente para que a caixa atinja
a outra ponta da esteira e seja então direcionado para a esteira correspondente. A seleção será feita por sensores \textbf{S4} dispostos ao final da esteira \textbf{E1}. Com base nos dados de \textbf{S4} um controlador irá decidir se a porta deve permanecer fechada, direcionando a caixa para a esteira \textbf{E2}, ou abrindo-a, fazendo com que a caixa siga para \textbf{E3}. A chegada da caixa  em \textbf{E2} ou \textbf{E3}
 é detectada por sensores\textbf{ S2 }ou \textbf{S3}, respectivamente, que devem então acionar os respectivos motores.

\section{Especificações da Automação}
\paragraph{}As figuras a seguir detalham o processo:
\begin{figure}[H]
	\centering
	\includegraphics[scale=0.5]{../images/detalhamento1.png}
	\caption{detalhes do processo: destaque para a posição horizontal dos sensores}
	\label{fig:planta}
\end{figure}

\begin{figure}[H]
	\centering
	\includegraphics[scale=0.5]{../images/detalhamento2.png}
	\caption{detalhes do processo: destaque para a posição vertical dos sensores}
	\label{fig:planta}
\end{figure}

\begin{enumerate}
	\item colocação manual das caixas pelos operários em 1
	\item  O sensor \textbf{S1} será posicionado no início da esteira \textbf{E1} a uma altura onde possa detectar as duas categorias de caixas.
	\item O sensor S1 está ligado ao controlador por uma porta digital e quando acionado o controlador deve dar partida no motor trifásico 
	M1	que movimenta a esteira E1. O controlador possui um contador T1 que mantém o motor ligado por 40 segundos desde o último acionamento de S1, isto é, toda vez que que S1 é acionado, resetamos o contador. Isso é feito para que a última caixa colocada tenha tempo 
	de percorrer a esteira.
	\item O acionamento do motor trifásico é feito com uma partida estrela-delta. Inicialmente a conexão das fases trifásicas é feita em
	estrela. Esse tipo de conexão fornece menos tensão entre as bobinas e limita a corrente de partida. Assim que o motor adquiri rotação
	e aumenta sua impedância, acionamos então a ligação em delta para dar potência total ao motor. Assumimos que um tempo de 2 segundos é adequado para esperar antes de chavear entre estrela e delta. Precisamos portanto de outro contador para medir esse tempo.
	\item A seleção entre conexão delta ou estrela é feita pelo controle de duas chaves contatoras controladas pelo controlador. Essas são então duas saídas digitais do controlador.
	\item  O sensor S4 é posicionado próximo da extremidade final da esteira E1 e vai ser o responsável por diferenciar as caixas. Ele é posicionado verticalmente de tal forma que somente caixas grandes são detectadas. Ao ser informado da presença de uma caixa grande, o controlador aciona o cilindro pneumático de dupla ação C1 para abrir a porta e permitir a passagem da caixa para E3.
	\item O cilindro pneumático de dupla ação C1 é controlado por uma válvula 5/2 acionada por relê com retorno por mola. Quando o relê é acionado a válvula permite fluxo de tal forma que o cilindro avança e abre a porta. Quando o sinal deixa de atuar no relê, a mola retorna a válvula para a posição de retorno do cilindro e a porta fecha. 
	\item Os sensores S2 e S3 fazem papel semelhante ao de S1 em suas respectivas esteiras. Quando acionados, o controlador deve iniciar
	um contador de 40 segundos durante os quais o motor é mantido acionado. Para o acionamento do motor novamente é necessário
	um circuito de partida estrela-delta com tempo em estrela de 2 segundos antes de chavear para delta. 
	
\end{enumerate}

\section{Defnição dos Sensores}
\paragraph{}Os 4 sensores utilizados detectaram os mesmos objetos e por isso serão todos de mesma natureza. De fato, serão todos iguais, mudando apenas o seu posicionamento na planta. Dentre os tipos comuns de sensores de proximidade, \textbf{descartamos}:
\begin{itemize}
	\item os indutivos pois nossas caixas não são metálicas a princípio
	\item de efeito Hall, pois nossas caixas não possuem ímãs 
	\item capacitivos, pois sua pequena faixa de detecção provavelmente não seria adequada.O operário tipicamente
	colocará as caixas na esteira de qualquer jeito, de modo que pequenas variações na posição da caixa poderiam fazer com que essa passasse 	        despercebida pelo sensor.
\end{itemize} 

Nos restam então os sensores ultra-sônicos e óticos. Pela proximidade com que os sensores serão instalados, descartamos os sensores ultra-sônicos, uma vez que poderiam interferir um no outro. Escolhemos portante um sensor \textbf{ótico}.

\paragraph{} Por simplicidade, procuramos um sensor cuja saída seja digital e de apena um bit. Isto é, só precisamos detectar presença, sem 
ter que medir distância. O sensor escolhido foi o \textbf{SHARP GP2D15}.  Suas especificações são ilustradas na figura a seguir:

\begin{figure}[H]
	\centering
	\includegraphics[scale=0.5]{../images/sensor1.png}
	\caption{dimensões do SHARP GP2D15}
\end{figure}
\begin{figure}[H]
	\centering
	\includegraphics[scale=0.5]{../images/sensor2.png}
	\caption{resposta do SHARP GP2D15}
\end{figure}

Notamos que o sensor deve ser instalado de forma que a caixa passe próximo a ele por não mais que 24 cm. Assim, quando não há caixa
o sensor está medindo o fundo da sala e sua saída está em alto, quando a caixa passa a saída muda para baixo. Isso deve ser levado
em consideração no desenvolvimento do programa.

\section{Definição dos Atuadores}
\paragraph{}Existem dois tipos de atuadores na planta: o motor trifásico e o cilindro pneumático de dupla ação.
\begin{itemize}
\item \textbf{Os motores trifásicos} serão utilizados apenas em uma direção, não sendo necessário então um circuito para inverter as fases. A forma de arranque dos motores vai ser estrela – delta, isto porque o arranque pelo método direto absorve uma alta corrente durante a conexão com a rede e pode danificar o sistema. O motor trifásico foi escolhido pelo alto torque que é capaz de produzir.

\item \textbf{O cilindro de dupla ação} é utilizado para controlar a posição da porta. Quando em avanço o cilindro fecha a porta e quando
em recuo a porta está fechada. O cilindro é controlado por uma válvula 5/2, justamente porque a porta só precisa está ou completamente
fechada ou completamente aberta. Além disso, a escolha por um cilindro foi feita pela sua simplicidade e robustez. O cilindro é hidráulico porque, na forma como pensamos o sistema, a porta não é pesada e não requer elevados esforços.
\end{itemize}

\section{Variáveis de Entrada e Saída}
A seguir listamos as variáveis de entrada do sistema para o controlador e de saída do controlador para o sistema.
\begin{table}[H]
	\centering 
	\begin{tabular}{|c|l|}\hline
	S1 &  indica a presença de caixa(qualquer) em E1 \\ \hline
	S2 &  indica a presença de caixa(qualquer) em E2 \\ \hline		
	S3 &  indica a presença de caix(qualquer)a em E3 \\ \hline		
	S4 & S1 indica a presença de caixa grande em E1 \\ \hline	
	\end{tabular}
	\caption{variáveis de entrada para o controlador}
\end{table}

\begin{table}[H]
	\centering 
	\begin{tabular}{|c|l|}\hline
	M1 & indica se o motor 1 está ligado ou não \\ \hline
	M1\_e & indica se o motor 1 está ligado  em estrela ou não \\ \hline	
	M1\_d & indica se o motor 1 está ligado  em delta ou não \\ \hline
			M2 & indica se o motor 2 está ligado ou não \\ \hline
	M2\_e & indica se o motor 2 está ligado  em estrela ou não \\ \hline	
	M2\_d & indica se o motor 2 está ligado  em delta ou não \\ \hline
		M3 & indica se o motor 3 está ligado ou não \\ \hline
	M3\_e & indica se o motor 3 está ligado  em estrela ou não \\ \hline	
	M3\_d & indica se o motor 3 está ligado  em delta ou não \\ \hline
	C & indica acionamento ou não do cilindro \\ \hline
	\end{tabular}
	\caption{variáveis de saída do controlador}
\end{table}

\section{Diagrama Elétrico de Montagem da Planta}
A seguir vemos o diagrama de montagem elétrico das fases do motor:
\begin{figure}[H]
	\centering
	\includegraphics[scale=0.5]{../images/ligacao_motor.png}
	\caption{ligação elétrica do motor}
\end{figure}

A seguir vemos a montagem do controlador LP CompactLogix 5370 L2: 1769-L24ER-QB1B.
Os pontos de entrada digital e pontos de saída digital estão localizados em um único módulo I / O incorporado.
\begin{figure}[H]
	\centering
	\includegraphics[scale=0.5]{../images/ligacao_controlador.png}
	\caption{ligação elétrica do controlador com a planta}
\end{figure}

\section{Módulos da linha CompactLogix}
Apresentamos algumas das opções encontradas:
\begin{itemize}
	\item CLP CompactLogix 5370 L2: 1769-L24ER-QB1B: possui módulo incorporado de entrada e saída
	\begin{figure}[H]
	\centering
	\includegraphics[scale=0.5]{../images/controlador1.png}
	\caption{ligação elétrica do motor}
\end{figure}

\begin{table}[H]
	\centering 
	\begin{tabular}{|c|c|}\hline
	Característica & 
1769-L24ER-QB1B\\ \hline

Memória disponível para o usuário &
0,75 MB \\ \hline

Cartão de memória &
1784 –SD1 (1 GB) e 1784 –SD2 (2 GB) \\ \hline

Portas de Comunicação &
2 EtherNet/IP
1 USB \\ \hline

E / S incorporado &
16 entradas de CC \\ &
16 saidas de CC \\ &
Conexões Ethernet / IP 
\\ \hline

8 EtherNet/IP &
120 TCP
\\ \hline

Capacidade de expansão de modulo &
4 módulos 1769
\\ \hline

Bateria & 
Nenhuma
\\ \hline
Fonte de alimentação embutida &
24 VCC\\ \hline
	\end{tabular}
	\caption{especificações }
\end{table}
	


\item Em vez de usar um controlador com IO embutido, podemos usar um módulo de entrada e saída como o  1769-Q6XOW4:

	\begin{figure}[H]
	\centering
	\includegraphics[scale=0.5]{../images/IO.png}
	\caption{módulo dedicado para IO}
	\end{figure}
	
		\begin{figure}[H]
	\centering
	\includegraphics[scale=0.5]{../images/IO2.png}
	\caption{módulo dedicado para IO possui 6 entradas digitais e 4 saídas digitas}
	\end{figure}
\end{itemize}
\section{Programa em Ladder}
Parte do código em Ladder é mostrado a seguir:
		\begin{figure}[H]
	\centering
	\includegraphics[scale=0.3]{../images/ladder.png}
	\caption{módulo dedicado para IO possui 6 entradas digitais e 4 saídas digitas}
	\end{figure}
	
	A lógica para os outros atuadores é a mesma, sendo apenas repetição da estrutura mostrada acima.

