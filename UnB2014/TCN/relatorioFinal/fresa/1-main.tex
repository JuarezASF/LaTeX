\documentclass[journal]{IEEEtran}

%% PARA USAR \FLOATBARRIER
\usepackage{float} 

\usepackage{hyperref} % For hyperlinks in the PDF

% The lettrine is the first enlarged letter at the beginning of the text
\usepackage{lettrine} 

%%PARA RODAR UM PT-BR DE BOAS
\usepackage[T1]{fontenc}
\usepackage[utf8]{inputenc}
\usepackage{lmodern}
\usepackage[portuguese]{babel}

%%PARA INCLUIR IMAGENS
\usepackage{graphicx}
\usepackage{fancyhdr}
\pagestyle{fancy}

\usepackage{caption}
\usepackage{subcaption}
\usepackage{placeins}

\usepackage{color}
\usepackage{cancel}

%%PARA EQUAÇÕES DENTRO DE UMA CAIXA E MUITO MAIS
\usepackage{amsmath}
%%para símbolos matemáticos avançados
\usepackage{amssymb}

\usepackage{tikz}
\usetikzlibrary{shapes,arrows}

%% PARA UTILIZAR ESTILIS E ENUMERAÇÃO
\usepackage{enumerate}

%%COMANDO PARA IMPRIMIR UMA LINHA VERTICAL MANEIRA
\newcommand{\HRule}{\rule{\linewidth}{0.5mm}}






%%$CONFIGURA TOPO DA PÁGINA
\fancyhead{} 

%%$CONFIGURA PÉ DA PÁGINA
\fancyfoot{}
\fancyfoot[C]{\thepage} 

\title{
    Universidade de Brasília \\
   Tecnologia de Comando Numérico\\
  Fresa\\
    \HRule
    \\
   Análise de Erro Dimensional e Geométrico de Capabilidade 
    \HRule \\
    {\normalsize \today}
}

\author{\begin{tabular}{llr}
    Professor: & Alberto J. Álvares & \\
    Aluno:& & \\
	&	Juarez Aires Sampaio Filho   & 11/0032829\\ 
    \end{tabular}
}


\begin{document}

\maketitle 

\thispagestyle{fancy}


%%INCLUI SEÇÕES DO TRABALHO
\fancyhead{} 
 
\section{Objetivos}
 Este trabalho tem como objetivo praticar as etapas do planejamento, execução e aferição da qualidade de uma peça de madeira MDF na 
 fresa disponível para os alunos no GRACO.  
 
\section{Código G}
\begin{verbatim}
$AddRegPart 1
(configurações)
G92 X100 Y100 Z20 //seta zero no centro da peça
T1 M6 M21 //troca ferramenta e fecha a porta
G90 M3 S900 F450 //coordenadas absolutas e parâmetros de usinagem

	(Braços início - 1.5 de profundidade)
G00 X0 Y0 Z5
G91 //coordenadas relativas
//BRAÇO 0º
G01 Z-6.5
X50 Y0
Z6.5
G00 X-50 Y-0
//BRAÇO 30º
G01 Z-6.5
Y37.5 X64.95
Z6.5
G00 Y-37.5 X-64.95
//BRAÇO 60º
G01 Z-6.5
X25 Y43.30
Z6.5
G00 X-25 Y-43.30
//BRAÇO 90º
G01 Z-6.5
X0 Y75
Z6.5
G00 X0 Y-75
//BRAÇO 120º
G01 Z-6.5
X-25 Y43.30
Z6.5
G00 X25 Y-43.30
//BRAÇO 150º
G01 Z-6.5
Y37.5 X-64.95
Z6.5
G00 Y-37.5 X64.95
//BRAÇO 180º
G01 Z-6.5
X-50 Y0
Z6.5
G00 X50 Y-0
//BRAÇO 210º
G01 Z-6.5
Y-37.5 X-64.95
Z6.5
G00 Y37.5 X64.95
//BRAÇO 240º
G01 Z-6.5
X-25 Y-43.30
Z6.5
G00 X25 Y43.30
//BRAÇO 270º
G01 Z-6.5
X0 Y-75
Z6.5
G00 X0 Y75
//BRAÇO 300º
G01 Z-6.5
X25 Y-43.30
Z6.5
G00 X-25 Y43.30
//BRAÇO 330º
G01 Z-6.5
Y-37.5 X64.95
Z6.5
G00 Y37.5 X-64.95

G90
	(Braços - fim)
	
	(COROA - INÍCIO)
G00 X0 Y0 Z5
G90 
G01 X25 Y0
Z-1.5
//PRIMEIRO QUADRANTE
X33.81 Y9.06
X21.65 Y12.5 
X24.75 Y24.75
X12.5 Y21.65
X9.06	Y33.81
X0	Y25
//SEGUNDO QUADRANTE
Y33.81 	X-9.06
Y21.65 	X-12.5 
Y24.75 	X-24.75
Y12.5 	X-21.65
Y9.06	X-33.81
Y0		X-25
//TERCEIRO QUADRANTE
X-33.81 Y-9.06
X-21.65 Y-12.5 
X-24.75 Y-24.75
X-12.5 Y-21.65
X-9.06	Y-33.81
X0	Y-25
//QUARTO QUADRANTE
Y-33.81 	X9.06
Y-21.65 	X12.5 
Y-24.75 	X24.75
Y-12.5 	X21.65
Y-9.06	X33.81
Y0		X25
Z5
G90
	(COROA - FIM)
	
	(TRAÇOS - INÍCIO - 1.0 de profundidade)
G00 X0 Y0 Z5
G91
//TRAÇOS BRAÇO 0º
G01 X40 
Z-6.0
X3.21 Y3.83
X-3.21 Y-3.83
X3.21 Y-3.83
X-3.21 Y3.83
Z6.0
G90
G00  X0 Y0
//TRAÇOS BRAÇO 30º
G00  X43.30 Y25 (50<30º)
Z-1.0
G91
G01  X0.87 Y4.92 (5<80º)
X-0.87 Y-4.92
X4.67 Y-1.71(5<-20º)
X-4.67 Y1.71
G90
Z5
G00  X51.96 Y30 (60<30º)
Z-1.0
G91
G01  X0.87 Y4.92 (5<80º)
X-0.87 Y-4.92
X4.67 Y-1.71(5<-20º)
X-4.67 Y1.71
G90
Z5
G00 X0 Y0

//TRAÇOS BRAÇO 60º
G00  X20 Y34.64
Z-1.0
G91
G01  X-1.71 Y4.70
X1.71 Y-4.70
X4.91 Y0.87
X-4.91 Y-0.87
G90
Z5
G00 X0 Y0

//TRAÇOS BRAÇO 90º
G00  X0 Y50 (50<90º)
Z-1.0
G91
G01  X3.83 Y3.21 (5<40º)
X-3.83 Y-3.21
X-3.83 Y3.21(5<140º)
X3.83 Y-3.21
G90
Z5
G00  X0 Y60 (60<90º)
Z-1.0
G91
G01  X3.83 Y3.21 (5<40º)
X-3.83 Y-3.21
X-3.83 Y3.21(5<140º)
X3.83 Y-3.21
G90
Z5


G00 X0 Y0

//TRAÇOS BRAÇO 120º
G00  X-20 Y34.64 (40<120º)
Z-1.0
G91
G01  X1.71 Y4.70 (5<70º)
X-1.71 Y-4.70
X-4.92 Y0.87(5<170)
X4.92 Y-0.87
G90
Z5
G00 X0 Y0

//TRAÇOS BRAÇO 150º
G00  X-43.30 Y25 (50<150º)
Z-1.0
G91
G01  X-4.67 Y-1.71 (5<200º)
X4.67 Y1.71
X-0.87 Y4.92(5<100º)
X0.87 Y-4.92
G90
Z5
G00  X-51.96 Y30 (60<150º)
Z-1.0
G91
G01  X-4.67 Y-1.71 (5<200º)
X4.67 Y1.71
X-0.87 Y4.92(5<100º)
X0.87 Y-4.92
G90
Z5
G00 X0 Y0


//TRAÇOS BRAÇO 180º
G00 X-40 (40<180º)
Z-1.0
G91
G01  X-3.21 Y3.83
X3.21 Y-3.83
X-3.21 Y-3.83
X3.21 Y3.83
G90
Z5
G00  X0 Y0

//TRAÇOS BRAÇO 220º
G00  X-43.30 Y-25 (50<210º)
Z-1.0
G91
G01  X-4.67 Y1.71 (5<160º)
X4.67 Y-1.71
X-0.87 Y-4.92(5<260º)
X0.87 Y4.92
G90
Z5
G00  X-51.96 Y-30 (60<210º)
Z-1.0
G91
G01  X-4.67 Y1.71 (5<160º)
X4.67 Y-1.71
X-0.87 Y-4.92(5<260º)
X0.87 Y4.92
G90
Z5
G00 X0 Y0

//TRAÇOS BRAÇO 240º
G00  X-20 Y-34.64 (40<240º)
Z-1.0
G91
G01  X1.71 Y-4.70 (5<70º)
X-1.71 Y4.70
X-4.92 Y-0.87(5<170)
X4.92 Y0.87
G90
Z5
G00 X0 Y0

//TRAÇOS BRAÇO 270º
G00  X0 Y-50 (50<270º)
Z-1.0
G91
G01  X3.83 Y-3.21 (5<320º)
X-3.83 Y3.21
X-3.83 Y-3.21(5<220º)
X3.83 Y3.21
G90
Z5
G00  X0 Y-60 (60<270º)
Z-1.0
G91
G01  X3.83 Y-3.21 (5<320º)
X-3.83 Y3.21
X-3.83 Y-3.21(5<220º)
X3.83 Y3.21
G90
Z5

G00 X0 Y0

//TRAÇOS BRAÇO 300º
G00  X20 Y-34.64 (40<300º)
Z-1.0
G91
G01  X-1.71 Y-4.70
X1.71 Y4.70
X4.91 Y-0.87
X-4.91 Y0.87
G90
Z5
G00 X0 Y0

//TRAÇOS BRAÇO 330º
G00  X43.30 Y-25 (50<330º)
Z-1.0
G91
G01  X0.87 Y-4.92 (5<280º)
X-0.87 Y4.92
X4.67 Y1.71(5<380º)
X-4.67 Y-1.71
G90
Z5
G00  X51.96 Y-30 (60<330º)
Z-1.0
G91
G01  X0.87 Y-4.92 (5<280º)
X-0.87 Y4.92
X4.67 Y1.71(5<380º)
X-4.67 Y-1.71
G90
Z5
G00 X0 Y0
	(TRAÇOS - FIM)

(Moldura - 1º passe - 1.5mm- INÍCIO)
G90
G00 X-75.0 Y-85.0 Z5 
G01  Z-1.5
X75
G03 X85 Y-75 R10 
G01 Y75
G03 X75 Y85 R10
G01  X-75
G03 X-85 Y75 R10
G01  Y-75
G03 X-75 Y-85 R10
G01  Z5
	(MOLDURA - 2º passe - FIM)

G90 
	(Braços 2º passe - 2.5mm)
G00 X0 Y0 Z5
G91 //coordenadas relativas
//BRAÇO 0º
G01 Z-7.5
X50 Y0
Z7.5
G00 X-50 Y-0
//BRAÇO 30º
G01 Z-7.5
Y37.5 X64.95
Z7.5
G00 Y-37.5 X-64.95
//BRAÇO 60º
G01 Z-7.5
X25 Y43.30
Z7.5
G00 X-25 Y-43.30
//BRAÇO 90º
G01 Z-7.5
X0 Y75
Z7.5
G00 X0 Y-75
//BRAÇO 120º
G01 Z-7.5
X-25 Y43.30
Z7.5
G00 X25 Y-43.30
//BRAÇO 150º
G01 Z-7.5
Y37.5 X-64.95
Z7.5
G00 Y-37.5 X64.95
//BRAÇO 180º
G01 Z-7.5
X-50 Y0
Z7.5
G00 X50 Y-0
//BRAÇO 210º
G01 Z-7.5
Y-37.5 X-64.95
Z7.5
G00 Y37.5 X64.95
//BRAÇO 240º
G01 Z-7.5
X-25 Y-43.30
Z7.5
G00 X25 Y43.30
//BRAÇO 270º
G01 Z-7.5
X0 Y-75
Z7.5
G00 X0 Y75
//BRAÇO 300º
G01 Z-7.5
X25 Y-43.30
Z7.5
G00 X-25 Y43.30
//BRAÇO 330º
G01 Z-7.5
Y-37.5 X64.95
Z7.5
G00 Y37.5 X-64.95

G90
	(Braços 2º passe - fim)
	
		(COROA - 2º passe - 2.5mm INÍCIO)
G00 X0 Y0 Z5
G90 
G01 X25 Y0
Z-2.5
//PRIMEIRO QUADRANTE
X33.81 Y9.06 Z-2.0
X21.65 Y12.5 Z-2.5
X24.75 Y24.75 Z-2.0
X12.5 Y21.65 Z-2.5
X9.06	Y33.81 Z-2.0
X0	Y25 Z-2.5
//SEGUNDO QUADRANTE
Y33.81 	X-9.06 Z-2.0
Y21.65 	X-12.5 Z-2.5
Y24.75 	X-24.75 Z-2.0
Y12.5 	X-21.65 Z-2.5
Y9.06	X-33.81 Z-2.0
Y0		X-25 Z-2.5
//TERCEIRO QUADRANTE
X-33.81 Y-9.06 Z-2.0
X-21.65 Y-12.5 Z-2.5
X-24.75 Y-24.75 Z-2.0
X-12.5 Y-21.65 Z-2.5
X-9.06	Y-33.81 Z-2.0
X0	Y-25 Z-2.5
//QUARTO QUADRANTE
Y-33.81 	X9.06 Z-2.0
Y-21.65 	X12.5 Z-2.5 
Y-24.75 	X24.75 Z-2.0
Y-12.5 	X21.65 Z-2.5
Y-9.06	X33.81 Z-2.0
Y0		X25 Z-2.5
Z5
G90
	(COROA - 2º passe - FIM)
	
	(Moldura - 2º passe - 2.5 mm- INÍCIO)
G90
G00 X-75.0 Y-85.0 Z5 
G01  Z-2.5
X75
G03 X85 Y-75 R10 
G01 Y75
G03 X75 Y85 R10
G01  X-75
G03 X-85 Y75 R10
G01  Y-75
G03 X-75 Y-85 R10
G01  Z5
	(MOLDURA - 2º passe - FIM)
	
	(furos - 1º passe)
G90
G01 X-75.0 Y-75.0 Z5
G01 Z-2.0
Z5
G01 X75.0 Y-75.0 Z5
G01 Z-2.0
Z5
G01 X75.0 Y75.0 Z5
G01 Z-2.0
Z5
G01 X-75.0 Y75.0 Z5
G01 Z-2.0
Z5

G01 X0 Y0 Z20 
M30 
\end{verbatim}

\section{Simulação}
A figura a seguir foi obtida com a simulação no programa CNC Simulador.
\begin{figure}[H]
\centering
\includegraphics[scale=0.5]{./images/simulacao}
\caption{Simulação obtida com CNC Simulator}
\end{figure}

\section{Resultados}
A peça produzida é mostrada a seguir.

\begin{figure}[H]
\centering
\includegraphics[scale=0.3]{./images/resultado.jpg}
\caption{resultado}
\end{figure}

Vamos na tabela \ref{tab:dados} dados dos valores medidos, projetados, médias e desvio padrão.

\begin{table*}[t]
\centering
\begin{tabular}{|c|c|c|c|c|c|c|c|c|c|} \hline
variável & $x_1$ (cm)& $x_2$(cm) & $x_3$(cm)& $x_4$(cm)& $x_5$ (cm)& $\overline{X}$(cm) & $\widehat{\sigma}$ (cm) & Projetado(mm) & Erro(\%) \\ \hline
medida 1& 16,600& 16,582&  16,604 & 16,600 & 16,584 &  16,594 &0,010  & 17,000 &2,39  \\ \hline
medida 2& 16,640& 16,650& 16,626& 16,634& 16,634 &16,637  & 0,009& 17,000 & 2,14\\ \hline
medida 3& 8,114 & 8,118 & 8,122& 8,124 & 8,138 & 8,123 & 0,009 &8,500 & 4,43\\ \hline
medida 4& 8,080& 8,104& 8,124& 8,086& 8,102 &8,099  & 0,017 & 8,500  & 4,72\\ \hline
\end{tabular}
\caption{Dados Experimentais e Estatísticos}
\label{tab:dados}
\end{table*}

\section{Análise de Capacidade}

Para uma análise rápida da capacidade de máquina, vamos tomar como variância das medidas feitas a média das variâncias de 
cada medida. 
\begin{equation}
	\overline{\widehat{\sigma}} =0,0113645925 cm
\end{equation}

O erro da máquina pode então ser estimado em:

\begin{equation}
\Delta = \frac{\widehat{\sigma}z}{\sqrt{n}}  = \frac{0,041 \cdot  2,7764}{\sqrt{5}} = 0,0141107762 cm
\end{equation}

Calculamos agora a capabilidade com um fator de segurança de 2 e uma tolerância de 1.0mm:
\begin{equation}
\begin{array}{l}
 	\mbox{Cp} = \frac{\mbox{Tolerância}}{ \mbox{fator de segurança} \cdot \mbox{Erro Inerente}}  \\
 	= \frac{0.5}{2 \cdot 2 \cdot 0,0882386589} \approx 1.8
\end{array}
 \end{equation}
 
 Vemos que a máquina em que a peça foi usinada é capaz, e com folga, de usinar uma peça com requisitos de projeto de tolerância dimensional
 de 1mm. Vemos ainda que o erro percentual ficou abaixo de 5\% em todas as medidas, mostrando que o processo foi preciso.
 
\textbf{CPK}


 Vamos calcular o CPK. Esta medida se aplica quando a tolerância sofre maior restrição em relação a um dos limites(superior ou inferior). 
 
 \begin{equation}
 	CPK = min(CPI, CPS)
 \end{equation}
 
 onde:
 
 \begin{equation}
 	CPI = \frac{\mbox{tolerância inferior}}{0.5 \cdot \mbox{variabilidade inerente}}
 \end{equation}
 
  \begin{equation}
 	CPS = \frac{\mbox{tolerância superior}}{0.5 \cdot \mbox{variabilidade inerente}}
 \end{equation}
 
 É claro que o menor CP está naquele com menor tolerância. Vamos calcular para uma tolerância inferior de 0.8mm e superior de 1mm:
\begin{equation}
\begin{array}{l}
CPI =2,83\\
CPS = 3,54 \\
CPK = CPI = 2,83
\end{array}
\end{equation} 

Vemos que ainda sim a máquina é capaz.
 
 \begin{thebibliography}{1}

\bibitem{PE-book}
  		Paul L. Meyer
  		\emph{Probabilidade Aplicações à Estatística}2ªed. LTC, 2009.(pag. 359, exemplo 14.18
	
\bibitem{t1}  		
		Duke University, Department of Statistical Science  	
  		\emph{FAQ'S ABOUT THE STUDENT-T DISTRIBUTION} www.isds.duke.edu/courses/Fall98/sta110b/tfaq.html
  		acesso em \today
  		
\bibitem{t2}  		
		Stat Trek 	
  		\emph{Student's t Distribution} http://stattrek.com/probability-distributions/t-distribution.aspx?tutorial=ap
  		acesso em \today

\end{thebibliography}

\end{document}
