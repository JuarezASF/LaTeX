\documentclass[journal]{IEEEtran}

%% PARA USAR \FLOATBARRIER
\usepackage{float} 

\usepackage{hyperref} % For hyperlinks in the PDF

% The lettrine is the first enlarged letter at the beginning of the text
\usepackage{lettrine} 

%%PARA RODAR UM PT-BR DE BOAS
\usepackage[T1]{fontenc}
\usepackage[utf8]{inputenc}
\usepackage{lmodern}
\usepackage[portuguese]{babel}

%%PARA INCLUIR IMAGENS
\usepackage{graphicx}
\usepackage{fancyhdr}
\pagestyle{fancy}

\usepackage{caption}
\usepackage{subcaption}
\usepackage{placeins}

\usepackage{color}
\usepackage{cancel}

%%PARA EQUAÇÕES DENTRO DE UMA CAIXA E MUITO MAIS
\usepackage{amsmath}
%%para símbolos matemáticos avançados
\usepackage{amssymb}

\usepackage{tikz}
\usetikzlibrary{shapes,arrows}

%% PARA UTILIZAR ESTILIS E ENUMERAÇÃO
\usepackage{enumerate}

%%COMANDO PARA IMPRIMIR UMA LINHA VERTICAL MANEIRA
\newcommand{\HRule}{\rule{\linewidth}{0.5mm}}






%%$CONFIGURA TOPO DA PÁGINA
\fancyhead{} 

%%$CONFIGURA PÉ DA PÁGINA
\fancyfoot{}
\fancyfoot[C]{\thepage} 

\title{
    Universidade de Brasília \\
   Tecnologia de Comando Numérico\\
   Torno Didático\\
    \HRule
    \\
   Análise de Erro Dimensional e Geométrico de Capabilidade 
    \HRule \\
    {\normalsize \today}
}

\author{\begin{tabular}{llr}
    Professor: & Alberto J. Álvares & \\
    Aluno:& & \\
	&	Juarez Aires Sampaio Filho   & 11/0032829\\ 
    \end{tabular}
}


\begin{document}

\maketitle 

\thispagestyle{fancy}


%%INCLUI SEÇÕES DO TRABALHO
\fancyhead{} 
\section{Introdução}

 No contexto das tecnologias de comando numérico aplicadas a usinagem, o\textbf{ código G} consiste numa sequências de instruções
 a serem interpretadas por uma máquina de modo a usinar uma determinada peça. A definição da linguagem apresenta
 diversos comandos de movimentação de ferramenta de forma a permitir que as mais variadas geometrias possam ser pelo código
 descritas.
 
 
 Antes mesmo da elaboração do código G, a receita deve ser escrita na forma de \textbf{planejamento de processo}. Nessa etapa,
 descreve-se os passos a serem seguidos para a usinagem em uma tabela. A máquina não irá interpretar essa tabela, mas ela serve de orientação para a codificação do código G e para as configurações de máquina, peça e ferramenta que podem ser necessárias. Em máquinas
 mais simples essa configuração é muitas vezes feita manualmente e o planejamento de processo informa o operador o que é necessário. Na prática deve-se primeiro escrever o planejamento do processo e só então passar para a elaboração do código G.
 
 
 Para a verificação da qualidade da peça diversos conceitos estatísticos devem ser apresentados. Em primeiro lugar deve-se falar sobre os tipos de \textbf{erros} envolvidos no processo. Primeiramente temos o \textbf{erro sistemático} que gera uma diferença sistemática entre  o valor projeto para uma dimensão da peça e o valor obtido. Dizemos que o erro é sistemático quando ele é oriundo de uma má configuração
 do sistema ou do processo. Por exemplo, digamos que estejamos medindo a gravidade e obtenhamos um valor de 18 m/$s^2$. O valor está muito distante daquele aceito como correto e deve então haver um erro sistemático no processo de medida. No contexto da usinagem, um erro sistemático ocorre, por exemplo, quando projetamos um diâmetro de 50mm para uma peça e medimos um diâmetro de 40mm. Quando o erro sistemático está presente em grande intensidade, dizemos que o processo foi \textbf{inacurado}, caso contrário foi \textbf{acurado}.
 
 
Uma outra modalidade de erro é o \textbf{erro aleatório}. Esta forma de erro está presente em qualquer processo de medida ou fabricação e está relacionado com as variações de natureza aleatório que ocorrem no processo. Ao medirmos o diâmetro de uma peça podemos obter valores próximos como 50.01mm ou 49.98mm e isso não significa que existam dois valores para a variável sendo medida. Significa que 
o instrumento sendo utilizado é suscetível a pequenas variações que, sem mais informações, são de natureza aleatória. Quando o erro aleatório é comparável ao valor da medida tomada dizemos que o processo foi\textbf{ impreciso}, quando não, é \textbf{preciso}.


Se medidas sucessíveis nos dão valores diferentes, como saber o valor real? De fato, nunca podemos ter absoluta certeza sobre o valor de uma variável sendo medida, mas podemos supor um comportamento para o erro aleatório e estimar duas grandezas importantes: a média e o desvio padrão dos valores. O primeiro informa a melhor aproximação para o valor real sendo medido e o segundo informa o quanto os valores medidos oscilam em torno da média. Podemos dizer que o processo é preciso quando possuir baixo desvio.


 Quanto ao comportamento esperado do erro aleatório, espera-se que ele varia seguindo uma distribuição normal, ou gaussiana.
No entanto, quando possuímos apenas uma pequena quantidade de amostras(<30) é mais recomendável adotar um modelo de distribuição \textbf{t de Student}. Com essa abordagem podemos, a partir de um pequeno conjunto de dados, afirmar algo do tipo "estima-se com X\% de
certeza que o valor real da média esteja entre A e B".



Para a utilização da t-student alguns conceitos são agora apresentados:
\begin{itemize}
\item \textbf{média da população ($\mu$)}: corresponde a média real da população ao qual os valores pertencem.

\item \textbf{média da amostra $\overline{X}$} : é a média dos dados. Notar que a média dos dados é, muito provavelmente, diferente
da média da população(conjunto de todas as medidas possíveis).
\begin{equation}
		\overline{X} = \frac{1}{n}\sum_{i=1}^{n} x_i
\end{equation}

\item \textbf{Desvio Padrão ($\sigma$)}: corresponde ao desvio real da população em torno da média real

\item \textbf{Estimação para o Desvio Padrão ($\widehat{\sigma}$)} corresponde a uma estimação para a variável acima
e é dado por;
	\begin{equation}
		\widehat{\sigma} = \sqrt{\frac{1}{n-1}\sum_{i=1}^{n} (x_i - \overline{X})^2}
	\end{equation}
\item  \textbf{Graus de Liberdade ($\xi$)}: corresponde ao números de variáveis aleatórios que podem variar desde que uma variável tenha
sido fixada. Para o caso em estudo:
\begin{equation}
	\xi = n-1
\end{equation}

\item \textbf{Consultando a tabela t de Student}: digamos que queiramos X\% de confiabilidade para a o valor da média $\mu$, determinamos então o valor $\alpha$
\begin{equation}
	\alpha = 1 - X
\end{equation}

Para consulta na tabela devemos olhar a linha $\xi$(graus de liberdade) coluna (1 - $\alpha/2$). Isso é feito pelo modo como
a tabela é construída. Teremos então o valor z. Calculamos então o comprimento de variação $\Delta$:
\begin{equation}
	\Delta = \frac{\widehat{\sigma}z}{\sqrt{n}}
\end{equation}

Finalmente, podemos dizer que "temos X\% de certeza de que a média real $\mu$ se encontra entre $\overline{X} - \Delta$ e $\overline{X} + \Delta$".
Para o caso X = 0.95 e n = 5(nosso caso), z = 2.7764.
\end{itemize}



Um vez calculado o comprimento $\Delta$ podemos adotá-lo como o erro aleatório envolvido em nossas medidas.  Aqui escolhemos 95\% arbritariamente, poderíamos escolher qualquer valor mas 95\% parece suficientemente conservador e rigoroso. Basta lembrar em que sentido definimos o erro:  "temos 95\% de certeza de que a média real $\mu$ se encontra entre $\overline{X} - \Delta$ e $\overline{X} + \Delta$".
 
 

Finalmente, apresentamos o conceito de capacidade. Como vimos anteriormente, todo processo apresenta erros. O projeto
 de uma peça deve ter isso em consideração e, portanto, definir tolerâncias dimensionais. O índice de capacidade indica se 
 uma determinada máquina é capaz ou não de produzir as especificações de erro informadas;
 \begin{equation}
 	\mbox{Capacidade} = \frac{\mbox{Tolerância}}{ \mbox{Erro Inerente}}
 \end{equation}
 A usinagem adequada é possível se esse índice for maior do que 1. Pode-se ainda acrescentar um  fator de segurança se
 adequado às necessidades do projeto.
 
\section{Objetivos}
 Este trabalho tem como objetivo praticar as etapas do planejamento, execução e aferição da qualidade de uma peça de cera usinada no torno
 didático.  
 
\section{Planejamento de Processo}
O planejamento de processo é visto na tabela \ref{tab:processo}.
\begin{table*}[htp]
\centering
\begin{tabular}{|c|l|c|c|c|c|}\hline
Nº & Operação & Z inicial(mm) & Z final(mm) & Profundidade(mm) & Raio(mm) \\ \hline
 1& desbaste &2 & -100 & 1 & \\\hline
  2& rasgo & -95 & &2.5 & \\\hline
   3& rasgo &-88 & &1.5 & \\\hline
   4 & rasgo & -81 & &0.5 & \\\hline
5& desbaste & -56 & -58  & 0.5 & \\\hline
6& desbaste& -56 & -60  &1.5 & \\\hline
7& perfilamento & -56 & -68 & 1.0 & \\\hline
8& desbaste & -36 & -38  & 0.5 & \\\hline
9& desbaste& -36 & -30  &1.5 & \\\hline
10& perfil. circular & -36 & -38 & 1.0 & 6\\\hline
11& desbaste & -16 & -18  & 0.5 & \\\hline
12& desbaste& -16 & -10  &1.5 & \\\hline
13& perfil. circular & -16 & -18 & 1.0 & 6\\\hline
 14& rasgo &-15 & &1.5 & \\\hline
 15& rasgo & -21 & &0.5 & \\\hline
\end{tabular}
\caption{Planejamento de Processo}
\label{tab:processo}
\end{table*}
\section{Código G}
\begin{verbatim}
($Lathe)
($Millimeters)
(110032829)
$AddRegPart 1
G90 G21 G40 ET8 M6
//G21  unidades em milimitros
//G40 cancela compensação de raio  
//ET8 M6 seleciona e troca ferramenta
G92 X0 Z128

(move para plano de segurança)
G00 X28 Z2

F250 S1000 M4 (liga a ferramente M4)

G01 	X24.0
Z-100(tira 1mm para homogeneizar)
X28
	(rasgos iniciais - começo)
	//primeiro rasgo profundidade 6
G00  Z-95 //G0
G01  X23.5 (tira 2cm)
X21.5
X19.5
X28 (segurança)


	//segundo rasgo profundidade 4
G00  Z-88 //G0
G01  X23.5
X21.5
X28 (segurança)

	//terceiro rasgo profundidade 2
G00  Z-81 //G0
G01	X23.5
X28 (segurança)
	(rasgos iniciais - fim)
	
	(declividade trapezoidal - inicio)
	(rasgo trapezoidal primeiro passe)
G00  Z-56 //G0 extremidade direita 
G01  X25.5 //aproximação
X23.5 Z-58
Z-72
X25.5 Z-74
X28 (segurança)
	(rasgo trapezoidal - segundo passe)
Z-56 //G0 extremidade direita 
X25.5 //aproximação
X21.5 Z-60
Z-70
X25.5 Z-74
X28 (segurança)
	(rasgo trapezoidal - terceiro passe)
Z-56 //G0 extremidade direita 
X25.5 //aproximação
X19.5 Z-62
Z-68
X25.5 Z-74
X28 (segurança)
	(declividade trapezoidal - fim)
	
	(declividade circular - inicio)
	(rasgo circular primeiro passe)
Z-36 //G0 extremidade direita 
X25.5 //aproximação
X23.5 Z-38
Z-50
X25.5 Z-52
X28 (segurança)
	(rasgo circular segundo passe)
Z-38 //G0 extremidade direita 
X25.5 //aproximação
X21.5 Z-40
Z-48
X25.5 Z-52
X28 (segurança)
	(rasgo circular terceiro passe)
Z-38 //G0 extremidade direita 
X25.5 //aproximação
X19.5 Z-42
Z-46
X25.5 Z-52
X28 (segurança)
	(rasco circular - acabamento)
Z-38 // G0 extremidade direita
X25.5
G2 X19.5 Z-42 R6  
G1 Z-46
G2 X25.5 Z-52 R6
G1 X28 (segurança) 
	(declividade circular - fim)
	
	(declividade circular com rasgos - início)
	(primeiro passe)
Z-16 //G0 extremidade direita 
X25.5 //aproximação
X23.5 Z-18
Z-28
X25.5 Z-30
X28 (segurança)
	(segundo passe)
Z-16 //G0 extremidade direita 
X25.5 //aproximação
X21.5 Z-20
Z-26
X25.5 Z-30
X28 (segurança)

	(primeiro rasgo)
Z-25
X23.5
X21.5
X19.5
X28(segurança)
	(segundo rasgo)
Z-21
X23.5
X21.5
X19.5
X28(segurança)
	(declividade circular com rasgos - FIM)
	
	(detalhe da ponta direita - inicio)
	(primeiro passe)
Z0
X23.5
X26.5 Z-2
	(segundo passe)
Z0
X21.5
X24.5 Z-4
	(terceiro passe)
Z0
X19.5
X24.5 Z-6
	(acabamento)
Z0 
X19.5
G2 X25.4 Z-6 R6 
X28(segurança)
	(detalhe da ponta direita - FIM)
	
	(rasgo na ponta direita - início)
Z-11
X25.5 (aproximação)
X23.5 (usinagem)
X21.5 (usinagem)
X19.5 (usinagem)
X28 (segurança)
	(rasgo na ponta direita - FIM)
	
G0	X40 Z40 
	
M30 (encerra programa)
\end{verbatim}

\section{Simulação}
A figura a seguir foi obtida com a simulação no programa CNC Simulador.
\begin{figure}[H]
\centering
\includegraphics[scale=0.5]{./images/simulacao}
\caption{Simulação obtida com CNC Simulator}
\end{figure}

\section{Resultados}
A peça produzida é mostrada a seguir.

\begin{figure}[H]
\centering
\includegraphics[scale=0.10, angle = -90]{./images/resultado.jpg}
\caption{resultado}
\end{figure}

Vamos na tabela \ref{tab:dados} dados dos valores medidos, projetados, médias e desvio padrão. As medidas são tomadas nos pontos indicados da figura a seguir.
\begin{figure}[H]
\centering
\includegraphics[scale=1.0]{./images/resultado_menor}
\caption{ pontos onde as medidas foram tomadas}
\end{figure}

\begin{table*}[htp]
\centering
\begin{tabular}{|c|c|c|c|c|c|c|c|c|c|} \hline
variável & $x_1$ (mm)& $x_2$(mm) & $x_3$(mm)& $x_4$(mm)& $x_5$ (mm)& $\overline{X}$(mm) & $\widehat{\sigma}$ (mm) & Projetado(mm) & Erro(\%) \\ \hline
medida 1& 42,971& 42,839&  42,906 & 42,924 & 42,920 &42,912  & 0,048 & 39,000 & 10,03 \\ \hline
medida 2& 45,084& 45,051& 45,030& 45,108& 45,111 &45,077  & 0,036 & 43,000 & 4,83\\ \hline
medida 3& 46,844 & 47,137 & 47,371& 47,124 & 47,194 &47,134 & 0,190 &47,000 &0,29 \\ \hline
medida 4& 47,611& 47,592& 47,517& 47,571& 47,644 &47,587  & 0,047 & 48,000 & 0,86\\ \hline
medida 5& 38,865 & 38,840 & 38,837 & 38,884 & 38, 791 & 38,843 &  0,035 &39,000 & 0,40\\ \hline 
\end{tabular}
\caption{Dados Experimentais e Estatísticos}
\label{tab:dados}
\end{table*}

\section{Análise de Capacidade}

Para uma análise rápida da capacidade de máquina, vamos tomar como variância das medidas feitas a média das variâncias de 
cada medida. 
\begin{equation}
	\overline{\widehat{\sigma}} =0,071065999
\end{equation}

O erro da máquina pode então ser estimado em:

\begin{equation}
\Delta = \frac{\widehat{\sigma}z}{\sqrt{n}}  = \frac{0,041 \cdot  2,7764}{\sqrt{5}} = 0,0882386589
\end{equation}

Calculamos agora a capabilidade com um fator de segurança de 2 e uma tolerância de 0.5mm:
\begin{equation}
\begin{array}{l}
 	\mbox{Cp} = \frac{\mbox{Tolerância}}{ \mbox{fator de segurança} \cdot \mbox{Erro Inerente}}  \\
 	= \frac{0.5}{2 \cdot 2 \cdot 0,0882386589} \approx 1.4
\end{array}
 \end{equation}
 
 Vemos que a máquina em que a peça foi usinada é capaz, e com folga, de usinar uma peça com requisitos de projeto de tolerância dimensional
 de 0.5mm. No entanto, as médias para as medidas tomadas destoam bastante dos valores projetados, mostrando a presença de
 erros sistemáticos. Apontamos duas fontes prováveis de erro: o zeramento da máquina e o diâmetro original da peça pode ter diferido
 daquela de projeto(50mm).
 
\textbf{CPK}


 Vamos calcular o CPK. Esta medida se aplica quando a tolerância sofre maior restrição em relação a um dos limites(superior ou inferior). 
 
 \begin{equation}
 	CPK = min(CPI, CPS)
 \end{equation}
 
 onde:
 
 \begin{equation}
 	CPI = \frac{\mbox{tolerância inferior}}{0.5 \cdot \mbox{variabilidade inerente}}
 \end{equation}
 
  \begin{equation}
 	CPS = \frac{\mbox{tolerância superior}}{0.5 \cdot \mbox{variabilidade inerente}}
 \end{equation}
 
 É claro que o menor CP está naquele com menor tolerância. Vamos calcular para uma tolerância inferior de 0.3 e superior de 0.8:
\begin{equation}
\begin{array}{l}
CPI = 1,70 \\
CPS = 4,53 \\
CPK = CPI = 1.70
\end{array}
\end{equation} 

Vemos que ainda sim a máquina é capaz.
 
 \begin{thebibliography}{1}

\bibitem{PE-book}
  		Paul L. Meyer
  		\emph{Probabilidade Aplicações à Estatística}2ªed. LTC, 2009.(pag. 359, exemplo 14.18
	
\bibitem{t1}  		
		Duke University, Department of Statistical Science  	
  		\emph{FAQ'S ABOUT THE STUDENT-T DISTRIBUTION} www.isds.duke.edu/courses/Fall98/sta110b/tfaq.html
  		acesso em \today
  		
\bibitem{t2}  		
		Stat Trek 	
  		\emph{Student's t Distribution} http://stattrek.com/probability-distributions/t-distribution.aspx?tutorial=ap
  		acesso em \today

\end{thebibliography}

\end{document}
