\section{Aritimética de Ponto Flutuante - Software}

\subsection{Raízes da Equação de Segundo Grau}
Consideramos um polinômio $p(x)$ de segunda ordem dado por:
\begin{equation}
	p(x) = ax^2 + bx + c
\end{equation}

\paragraph{} Queremos achar as raízes $r_1$ e $r_2$de $p(x)$. Isso foi feito
utilizando-se a fórmula de Bhaskara:
\begin{equation}
	r_{1,2} = \frac{-b \pm \sqrt{b^2 -4ac}}{2a}
\end{equation}
se $b^2 -4ac \geq 0$ e 
\begin{equation}
    r_{1,2} = \frac{-b \pm j\sqrt{-(b^2 -4ac)}}{2a}
\end{equation}
se $b^2 -4ac < 0$. 
\paragraph{} Vemos que a aplicação da fórmula requer uma rotina para cálculo de
raíz quadrada positiva. Quando fizemos o trabalho não sabíamos que o MIPS já
possuia uma rotina para essa finalidade e implementou-se então o método da
bisseção descrito a seguir:

\begin{figure}[H]
\centering
\begin{BVerbatim}
    recebe (X,e)
    L = 0
    R = X
    E = R - L
    enquanto E > e
        M = (L+R)/2
        se M^2 > x, então R = M
        se M^2 < x, então L = M
        se M^2 == x, então retorne M
    fim enquanto
    retorna M	
\end{BVerbatim} 
\end{figure}

\paragraph{}Em nossa implementação definimos 'e' fixo e igual a .Além disso,
para melhor precisão, todos os cálculos foram feitos com double e só ao final convertidos para float de
precisão simples.

\subsubsection{O cógido em assembly MIPS}
\paragraph{}O código completo desenvolvido pode ser visto
\href{https://github.com/JuarezASF/LaTeX/blob/master/UnB2014/OAC/exp2/code/raizes2grau.asm}{aqui}.

\subsubsection{O procedimento show}
O procedimento show pode ser visto no mesmo link acima.

\subsubsection{Resultados}
\paragraph{}O código é testado com algumas entradas e a resposta mostrada na
tabela a seguir:
\begin{table}[H]
\centering
\begin{tabular}{|c|c|c|c|c|} \hline
	\multicolumn{3}{|c|}{ Entrada} & \multicolumn{2}{|c|}{ Saída} \\ \hline
	a & b & c & $r_1$ & $r_2$ \\ \hline
	1 & 0 &  -9.86960440 &3.1415925 & -3.1415925 \\ \hline
	1 & 0 &  0           & 0.0     &-0.0 \\ \hline
	1 & 99&  2459        &-49.5 + i* 2.95804  &-49.5 + i* -2.95804 \\ \hline
	1 & -2468 & 33762440  &1234.0 + i* 5678.0 & 234.0 + i* -5678.0 \\ \hline
	0 & 10 & 100  &-10    & NaN \\ \hline
\end{tabular}
\caption{testes da implementação}
\end{table}
 
\subsection{Casas Decimais de $\pi$}
\paragraph{}Para calcular os dígitos de pi, usaremos o \emph{Spigot Algorithm
for the Digits of Pi}\footnote{em uma tradução livre: Algoritmo Torneira para os Dígitos
de Pi} descrito em \cite{SpigotPi}. A seguir apresentamos brevemente a
motivação teórica para o algoritmo e em seguida apresentamos o algoritmo
propriamente dito.


\paragraph{} Primeiramente faremos uma reinterpreação das casas decimais:
\begin{equation}
    \sqrt{2} = 1.41421356 = 1 + \frac{1}{10} \left( 4 + \frac{1}{10}\left(
    1 + \frac{1}{10} \left( 4 + \frac{1}{10} \left( 2 + \frac{1}{10}
    \left(1 + \ldots \right) \right) \right) \right) \right)
\end{equation}
 
Podemos resumir a notação dizendo que $\sqrt{2} = (1;4,1,4,2,1,\ldots)_{d}$,
onde d é a base decimal $\{\frac{1}{10}, \frac{1}{10},\frac{1}{10},\frac{1}{10},
\ldots \}$. A base d é aquela que estamos mais habituados, mas qualquer coleção
de frações pode formar uma base nesse sentido e todo número terá uma
representação nessa base criada. Na base binária, por exemplo, apenas
substituímos o 10 do denominador por 2. Veja que na base decimal e na base
binária, temos os mesmos elementos em cada entrada do vetor base, e se isso não acontecer? 

\paragraph{} Como um exemplo de uma base exótica e de aplicação
interessante, considere a base $b =
\{\frac{1}{2},\frac{1}{3},\frac{1}{4},\frac{1}{5}, \ldots, \frac{1}{n}\}$. 
Para ver como ela pode ser útil, utilize a série de potências da
exponencial para escrever o número de Euler $e$ como:
\begin{equation}
    exp(1) = e = \sum_{i=0}^{\infty} \frac{1^i}{i!} = \sum_{i=0}^{\infty}
    \frac{1}{i!} = 1 + \frac{1}{1} + \frac{1}{2} + \frac{1}{3*2} + \ldots 
\end{equation}
que pode ser reescrito como:
\begin{equation}
e = 1 + \frac{1}{1}(1 + \frac{1}{2}(1 + \frac{1}{3}(1 + \frac{1}{4}(1 +
\frac{1}{5}(1 + \ldots)))))
\end{equation}

ou seja,:
\begin{equation}
    e = (2;1,1,1,1,1,1, \ldots)_b
\end{equation}
isto é, o número $e$ é uma dízima periódica na base b! Será que não podemos
achar uma base que represente $\pi$ em uma dízima periódica? É justamente essa a
ideia do algoritmo utilizado. No caso, a base utilizada é $c = \{\frac{1}{3},
\frac{2}{5}, \frac{3}{7}, \frac{4}{9},\ldots, \frac{i-1}{2i-1}, \ldots \}$ e é
possível mostrar, também utilizando séries de potência, que:
\begin{equation}
    \pi = (2;2,2,2,2,2,2,2, \ldots)_c
\end{equation}

O trabalho para achar as casas decimais de pi consiste então em converter a
representação na base c para a base decimal usual! O algoritmo é apresentado a
seguir:

\begin{figure}[H]
    \begin{BVerbatim}
        1. Inicialize: Seja A = (2,2,2,2,2,....,2) um array de comprimento
        [10n/3] + 1
        2. Repite n + 1 vezes
            2.1: multiplica A por 10
            2.2: coloque na forma regular:
                comecando da direita ate o segundo elemento do vetor:
                    reduza o i-esimo elemento mod(2i-1), isto é:
                     calcule q e r tal que: A[i] = q*(2i-1) + r
                     A[i] = r
                     A[i-1] += q(i-1)
            2.3: calcule o proximo pre-digito
                   reduza o primeiro elemento mod(10), isto é:
                     A[0] = q*(10) + r
                     A[0] = r
                     salve q para a próxima etapa
            2.4: decida o que fazer com o vetor preDigitos com base em q:
                    se q == 9:
                        preDigitos.add(q)
                    caso contrario, se q == 10:
                        q = 0
                        adicione 1 a todos os pre-digitos anteriores( 9 vira 0)
                        digitosVerdadeiros.add(preDigitos)
                        esvazia preDigitos
                        adiciona q aos preDigitos
                     caso contrario:
                        digitosVerdadeiros.add(preDigitos)
                        esvazia preDigitos
                        adiciona q aos preDigitos
                        
                        
    \end{BVerbatim}
\caption{Algoritmo $\pi$-spigot}
\end{figure}

\paragraph{} Note que \textbf{o algoritmo usa apenas aritmética de interiso para
calcular os elementos de $\pi$} e é de simples implementação computacional!

\paragraph{} No trabalho desenvolvido, o algoritmo foi primeiramente
implementado em C++ para testar sua funcionalidade e em seguida implementado em
assembly para MIPS. Analisando o algoritmo, vemos que ele precisa de estruturas
de dados para armazenar os números já calculados e outra para armazenar os pre-dígitos, que deve ter o tamanho acrescentado
e resetado ao longo do algoritmo. Boa parte do trabalho em
assembly foi implementar essa estrutura de dados para lidar com dados de tamanho de variado.Em C++ utilizamos a estrutura
\emph{vector} que já está implementada e isso facilita muito a implementação.  Vale a pena comparar o
tamanho do código: em C++ foram necessárias 109 linhas, em assembly o código
completo usa 740. Além disso, o código em C++ foi escrito em torno de 30
minutos e o código em assembly precisou de quase 5 horas.

\subsubsection{O cógido em assembly MIPS}
\paragraph{}O código completo desenvolvido em C++ pode ser visto
\href{https://github.com/JuarezASF/LaTeX/blob/master/UnB2014/OAC/exp2/code/PI_digits_cpp.cpp}{aqui} e o desenvolvido
em assemlby MIPS pode ser visto \href{https://github.com/JuarezASF/LaTeX/blob/master/UnB2014/OAC/exp2/code/piDigits_assembly_mips.asm}{aqui}.

\subsubsection{A função show\_pi}
\paragraph{} A função pode ser vista no mesmo código assembly mips indicado anteriormente.

\subsubsection{O Custo Computacional}
O programa foi rodado diversas vezes para valores diferentes do número de casas desejadas e o número de operações
necessárias foi anotado. Obteve-se a seguinte tabela:
\begin{table}[H]
	\centering 
	\begin{tabular}{|c|c|}\hline
	número de casas & número de instruções \\ \hline
	5	& 19839 \\ \hline
10	& 	70565 \\ \hline
15	& 	152361 \\ \hline
20		& 261474 \\ \hline
25		& 404655 \\ \hline
30	& 	578862 \\ \hline
	\end{tabular}
	\caption{Número de instruções por número de casas calculadas para o Spigot Algorithm}
\end{table}

A seguir vemos os dados plotados em um gráfico junto com uma regressão polinomial de segunda ordem.

         \begin{figure}[H]
                 \centering
                 \includegraphics[scale = 0.4]{../images/AQ2c_graph.png}
                \caption{Dados e Regressão para o custo do Spigot Algorithm}
         \end{figure}
         
         O polinômio calculado para o custo é:
         \begin{equation}
         	c(n) = 616.4\cdot n^2 + 749.2\cdot n +  1133.7
         \end{equation}
         
         A extrapolação para n = 100000 é mostrada a seguir:
                  \begin{figure}[H]
                 \centering
                 \includegraphics[scale = 0.4]{../images/AQc_graph02.png}
                \caption{Extrapolação para o custo do Spigot Algorithm}
         \end{figure}
         
     \subsubsection{Limitações no MARS}
     O fator limitante para grandes valores de n é o tamanho da memória. O algoritmo utiliza 3 vetores:\textbf{ A},\textbf{ preDigits} e\textbf{ piDigits}. \textbf{A} tem tamanho $\frac{10}{3}n + 1$, \textbf{preDigits} e \textbf{piDigits} tem tamanho
     n. Além disso, cada dígito do vetor é um número de 0 a 9, então pode ser armazenado em 1 byte. O algoritmo precisa então de:
     \begin{equation}
     		m(n) = \frac{10}{3}n+1 + 2n = \frac{19}{3} \cdot n \mbox{ bytes}
     \end{equation}
     
     Estamos utilizando o recurso de armazenamento de dados na heap por meio do syscall 9 que aloca memória dinamicamente.
     Para tanto, o MARS disponibiliza os endereços de 0x10040000 até (0x100401e0 + 1c). Contando em words isso dá 8x16 words (8 colunas por
     16 linhas). Portanto, o número de bytes disponíveis para acesso dinâmico é 8x16x4 = 512 bytes. Basta então resolver para n:
     
     \begin{equation}
	     \frac{19}{3}n = 512 \Rightarrow n =  80.8421
     \end{equation}
     
     Portanto, com o MARS podemos calcular no máximo 80 dígitos de pi devido à limitação de memória
     
     \subsubsection{Tempo de Execução para n máximo}
     Como já dissemos,\textbf{ nosso algoritmo não utiliza aritimética de ponto flutuante}, por isso levamos em conta apenas o custo
     das operações de ponto fixo. Jogamos o n máximo na fórmula obtida por regressão para obter o número de instruções
     necessárias para n = 80:
     
     \begin{equation}
     			c(96) = 616.4\cdot 80^2 + 749.2\cdot 80 +  1133.7 = 4.0060 \cdot 10^6\mbox{ instruções}
     \end{equation}
     
     Calculamos o tempo de ciclo:
     
     \begin{equation}
     	T = \frac{1}{f} = \frac{1}{1E9} = 1\cdot 10^{-9} \mbox{ segundos}
     \end{equation}
     
     Utilizamos a fórmula clássica para tempo de execução:
     
     \begin{equation}
     	t = n \cdot CPI \cdot T =  4.0060E06 \cdot 1 \cdot 1E-9 =  4.0060\cdot 10^{-3} s = 4.0060 \mbox{ milisegundos}
     \end{equation}
     
     \subsubsection{Recorde Atual de casas para Pi}
     
     Segundo a wikipedia, em 2011 já havia-se calculado $\pi$ com mais de 10 trilhões de casas decimais
     
     
     



