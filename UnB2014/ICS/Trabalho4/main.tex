\documentclass{beamer}
\usepackage[T1]{fontenc}
\usepackage[utf8]{inputenc}
\usepackage{lmodern} 
\usepackage[portuguese]{babel}
\usepackage{graphicx}			%para imagens
\usepackage{epstopdf} 			%resolve problemas eps-pdf
\usepackage{fancyhdr}			% para o cabeçalho bonito
\usepackage{caption}				%para legendas
\usepackage{placeins} 			%controlar o lugar dos floats
\usepackage{hyperref}
\usepackage{listings}
\usepackage{color}
\usepackage{xcolor}
\usepackage{tikz}

\usepackage{subcaption}

\definecolor{dkgreen}{rgb}{0,0.6,0}
\definecolor{gray}{rgb}{0.5,0.5,0.5}
\definecolor{mauve}{rgb}{0.58,0,0.82}

\lstset{frame=tb,
  language=Java,
  aboveskip=3mm,
  belowskip=3mm,
  showstringspaces=false,
  columns=flexible,
  basicstyle={\tiny\ttfamily},
  numbers=none,
  numberstyle=\tiny\color{gray},
  keywordstyle=\color{blue},
  commentstyle=\color{dkgreen},
  stringstyle=\color{mauve},
  breaklines=true,
  breakatwhitespace=true
  tabsize=3
}
\lstset{language=Java}

\defbeamertemplate*{title page}{customized}[1][]
{
  \usebeamerfont{title}\inserttitle\par
  \usebeamerfont{subtitle}\usebeamercolor[fg]{subtitle}\insertsubtitle\par
  \bigskip
 %% \usebeamerfont{author}\insertauthor\par
  \usebeamerfont{institute}\insertinstitute\par
  \usebeamerfont{date}\insertdate\par
  \usebeamercolor[fg]{titlegraphic}\inserttitlegraphic
}

%DEFINE TEMA BEAMER A SER UTILIZADO
\usetheme{Warsaw}
\setbeamertemplate{navigation symbols}{}%remove navigation symbols
%DEFINIÇÃO DE TÍTULO, AUTORES .....
\title[Introdução a Computação Sônica]{ ICS - Trabalho IV \\ Introdução ao SOM-A}
%\title[pequeno título que vai no bottom da página]{título grande}%
\author{Juarez Aires Sampaio Filho 11/0032829}
\institute{Universidade de Brasília}
\date{\today}

\begin{document}


\begin{frame}
        \titlepage
\end{frame}

\AtBeginSection[]
{
}
\section{Enunciado}
\begin{frame}{Enunciado}
	\begin{block}{}
		Escreva uma carta espectral SOM-A -- correspondente a uma partitura de sua escolha -- 
que contenha dois ou mais instrumentos aditivos (dependendo de quantos instrumentos 
houver na orquestra), cada um deles sendo construído com base na Transformada Rápida 
de Fourier (FFT) aplicada ao sinal (de uma nota) de um dado instrumento acústico, também 
de sua escolha.
	\end{block}
\end{frame}

\section{Programas Utilizados}
\begin{frame}{Programas Utilizados}
Utilizou-se os seguintes programas no decorrer do trabalho;
\begin{itemize}
	\item SOM-A 
	\item INS-A: para gerar um instrumento na linguagem do SOM-A com base em uma amostra
	e por meio da transformada rápida de Fourrier.
	\item TuxGuitar: programa utilizado muitas vezes para executar partituras de guitarra; aqui utilizado para converter partitura em arquivo MIDI.
	\item Audacity: utilizado para recortar amostra de som.
\end{itemize}

\end{frame}


\begin{frame}{TuxGuitar}
\begin{figure}
	\includegraphics[scale=0.2]{./images/TuxGuitarPrint.png}
	\caption{TuxGuitar}
\end{figure}
\end{frame}

\begin{frame}{TuxGuitar}
	Com o TuxGuitar é possível	
	\begin{itemize}
		\item editar a partitura
		\item adicionar, remover e renomear trilhas
		\item exportar a partitura criada/editada para formato midi
	\end{itemize}

	Com essa ferramenta ganhamos uma imensa biblioteca de arquivos midi disponíveis:
	as partituras originalmente feitas para o programa GuitarPro. 	
	
\end{frame}


\begin{frame}{Audacity}
\begin{figure}
	\includegraphics[scale=0.2]{./images/AudacityPrint.png}
	\caption{Audacity}
\end{figure}
\end{frame}

\begin{frame}{Audacity}
	Com o Audacity é possível	
	\begin{itemize}
		\item editar um arquivo .aff
		\item exportar o som editado para formato wave
	\end{itemize}

	As amostras disponíveis pela Universidade de Iowa possuem notas diversas em um mesmo arquivo aff e utilizamos o audacity para isolar uma nota em particular que irá servir como
	entrada para o INS-A.	
\end{frame}

\begin{frame}{INS-A}
\begin{figure}
	\includegraphics[scale=0.2]{./images/InsA_Print.png}
	\caption{INS-A}
\end{figure}
\end{frame}

\begin{frame}{INS-A}
	Com o INS-A é possível	
	\begin{itemize}
		\item carregar um arquivo wave
		\item tocar esse arquivo
		\item obter a transformada de Fourrier do sinal
		\item obter um instrumento no formato SOM-A
	\end{itemize}

\end{frame}


\begin{frame}{SOM-A}
\begin{figure}
	\includegraphics[scale=0.2]{./images/SomA_Print.png}
	\caption{SOM-A}
\end{figure}
\end{frame}

\begin{frame}{SOM-A}
	Dentre as inúmeras funcionalidades do SOM-A é possível	
	\begin{itemize}
		\item carregar uma carta espectral a partir de um arquivo midi
		\item carregar e editar instrumentos criados com INS-A
		\item interpretar a carta e produzir o respectivo arquivo wave
	\end{itemize}

\end{frame}

\section{A Música Escolhida}
\begin{frame}{A Música Escolhida}
Escolheu-se interpretar a música \emph{Canon in D} ou \emph{Cânone em Ré Maior}  de Pachelbel. Nessa peça, composta no século 17, três violinos tocam o cânone (cada parte entrando com a mesma música a dois compassos de intervalo), enquanto um baixo contínuo executa uma passagem de fundo curta e simples.


Curiosidades:
\begin{itemize}
	\item música comumente tocada em casamentos
	\item a música \emph{Basket Case} do Green Day foi composta com base na sequência
	de acordes de Canon in D.
	\item a mesma sequência de acordes aparece em diversas outras músicas pops.
	\item Os acordes em questão, quando tocados na escala Ré Maior são: Ré Maior, Lá maior, Si Menos, Fá Sustenido Menor, Sol Maior, Ré Maior, Sol maior e Lá Maior 
\end{itemize}
\end{frame}

\section{Instrumentos Produzidos}
\begin{frame}{Instrumentos Produzidos}
Para as partituras trabalhadas, produziu-se três instrumentos:
\begin{itemize}
	\item cello
	\item violino
	\item violão
\end{itemize}

A seguir vemos gráficos dos primeiros harmônicos de cada instrumento
\end{frame}

\begin{frame}{Cello 1/2}
\vspace{-1 cm}
\begin{figure}
  \begin{subfigure}[b]{.15\linewidth}
    \caption*{}
    \includegraphics[height=1.5cm]{./images/cello_f1.png}
  \end{subfigure}
  \hspace{2 cm}
  \begin{subfigure}[b]{.15\linewidth}
    \caption*{} 
    \includegraphics[height=1.5cm]{./images/cello_f2.png}
  \end{subfigure}
\hspace{2 cm}
  \begin{subfigure}[b]{.15\linewidth}
    \caption*{} 
    \includegraphics[height=1.5cm]{./images/cello_f3.png}
  \end{subfigure}  
  \\ \vspace{-0.5 cm}
  \begin{subfigure}[b]{.15\linewidth}
    \caption*{}
    \includegraphics[height=1.5cm]{./images/cello_f4.png}
  \end{subfigure}
  \hspace{2 cm}
  \begin{subfigure}[b]{.15\linewidth}
    \caption*{} 
    \includegraphics[height=1.5cm]{./images/cello_f5.png}
  \end{subfigure}
\hspace{2 cm}
  \begin{subfigure}[b]{.15\linewidth}
    \caption*{} 
    \includegraphics[height=1.5cm]{./images/cello_f6.png}
  \end{subfigure}  
\\ \vspace{-0.5 cm}
  \begin{subfigure}[b]{.15\linewidth}
    \caption*{}
    \includegraphics[height=1.5cm]{./images/cello_f7.png}
  \end{subfigure}
  \hspace{2 cm}
  \begin{subfigure}[b]{.15\linewidth}
    \caption*{} 
    \includegraphics[height=1.5cm]{./images/cello_f8.png}
  \end{subfigure}
\hspace{2 cm}
  \begin{subfigure}[b]{.15\linewidth}
    \caption*{} 
    \includegraphics[height=1.5cm]{./images/cello_f9.png}
  \end{subfigure}  
\caption{Harmônicos 1-9 do Cello}
\end{figure}
\end{frame}

\begin{frame}{Cello 2/2}
\vspace{-1 cm}
\begin{figure}
  \begin{subfigure}[b]{.15\linewidth}
    \caption*{}
    \includegraphics[height=1.5cm]{./images/cello_f10.png}
  \end{subfigure}
  \hspace{2 cm}
  \begin{subfigure}[b]{.15\linewidth}
    \caption*{} 
    \includegraphics[height=1.5cm]{./images/cello_f11.png}
  \end{subfigure}
\hspace{2 cm}
  \begin{subfigure}[b]{.15\linewidth}
    \caption*{} 
    \includegraphics[height=1.5cm]{./images/cello_f12.png}
  \end{subfigure}  
  \\ \vspace{-0.5 cm}
  \begin{subfigure}[b]{.15\linewidth}
    \caption*{}
    \includegraphics[height=1.5cm]{./images/cello_f13.png}
  \end{subfigure}
  \hspace{2 cm}
  \begin{subfigure}[b]{.15\linewidth}
    \caption*{} 
    \includegraphics[height=1.5cm]{./images/cello_f14.png}
  \end{subfigure}
\hspace{2 cm}
  \begin{subfigure}[b]{.15\linewidth}
    \caption*{} 
    \includegraphics[height=1.5cm]{./images/cello_f15.png}
  \end{subfigure}  
\caption{Harmônicos 10-15 do Cello}
\end{figure}
\end{frame}

\begin{frame}{Violino}
\vspace{-1 cm}
\begin{figure}
  \begin{subfigure}[b]{.15\linewidth}
    \caption*{}
    \includegraphics[height=1.5cm]{./images/violino_f1.png}
  \end{subfigure}
  \hspace{2 cm}
  \begin{subfigure}[b]{.15\linewidth}
    \caption*{} 
    \includegraphics[height=1.5cm]{./images/violino_f2.png}
  \end{subfigure}
\hspace{2 cm}
  \begin{subfigure}[b]{.15\linewidth}
    \caption*{} 
    \includegraphics[height=1.5cm]{./images/violino_f3.png}
  \end{subfigure}  
  \\ \vspace{-0.5 cm}
  \begin{subfigure}[b]{.15\linewidth}
    \caption*{}
    \includegraphics[height=1.5cm]{./images/violino_f4.png}
  \end{subfigure}
  \hspace{2 cm}
  \begin{subfigure}[b]{.15\linewidth}
    \caption*{} 
    \includegraphics[height=1.5cm]{./images/violino_f5.png}
  \end{subfigure}
\hspace{2 cm}
  \begin{subfigure}[b]{.15\linewidth}
    \caption*{} 
    \includegraphics[height=1.5cm]{./images/violino_f6.png}
  \end{subfigure}  
\\ \vspace{-0.5 cm}
  \begin{subfigure}[b]{.15\linewidth}
    \caption*{}
    \includegraphics[height=1.5cm]{./images/violino_f7.png}
  \end{subfigure}
  \hspace{2 cm}
  \begin{subfigure}[b]{.15\linewidth}
    \caption*{} 
    \includegraphics[height=1.5cm]{./images/violino_f8.png}
  \end{subfigure}
\hspace{2 cm}
  \begin{subfigure}[b]{.15\linewidth}
    \caption*{} 
    \includegraphics[height=1.5cm]{./images/violino_f9.png}
  \end{subfigure}  
\caption{Harmônicos 1-9 do Violino}
\end{figure}
\end{frame}

\begin{frame}{Violão}
\vspace{-1 cm}
\begin{figure}[htp]
  \begin{subfigure}[b]{.15\linewidth}
    \caption*{}
    \includegraphics[height=1.5cm]{./images/violao_f1.png}
  \end{subfigure}
  \hspace{2 cm}
  \begin{subfigure}[b]{.15\linewidth}
    \caption*{} 
    \includegraphics[height=1.5cm]{./images/violao_f2.png}
  \end{subfigure}
\hspace{2 cm}
  \begin{subfigure}[b]{.15\linewidth}
    \caption*{} 
    \includegraphics[height=1.5cm]{./images/violao_f3.png}
  \end{subfigure}  
  \\ \vspace{-0.5 cm}
  \begin{subfigure}[b]{.15\linewidth}
    \caption*{}
    \includegraphics[height=1.5cm]{./images/violao_f4.png}
  \end{subfigure}
  \hspace{2 cm}
  \begin{subfigure}[b]{.15\linewidth}
    \caption*{} 
    \includegraphics[height=1.5cm]{./images/violao_f5.png}
  \end{subfigure}
\hspace{2 cm}
  \begin{subfigure}[b]{.15\linewidth}
    \caption*{} 
    \includegraphics[height=1.5cm]{./images/violao_f6.png}
  \end{subfigure}  
\caption{Violão 1-6 do Violão}
\end{figure}
\end{frame}


\section{Problemas Encontrados}
\begin{frame}{Problemas Encontrados}
\begin{itemize}
	\item Alguns efeitos encontrados na partitura não foram corretamente traduzidos para MIDI
	\item É preciso rodar a montagem diversas vezes e observar se não houve overflow e ir
	reduzindo a envoltória até que este não ocorra. Há grandes chances de ocorrer o overflow quando a música possui vários instrumentos e alguns deles tocam acordes.
	\item Para acelerar a 'compilação' do som, é necessário reduzir o número de harmônicos presentes no instrumento. Arbitrariamente escolhemos utilizar 10 harmônicos em todos os instrumentos.
	\item não conseguiu-se obter um som 'decente' de violão com poucos harmônicos.
\end{itemize}

\end{frame}

\section{Referências}
\begin{frame}
  \frametitle{Referências}
  \begin{itemize}
  \item \href{http://theremin.music.uiowa.edu/MIS.html}{University of Iowa - Digital Instrument Studios - Banco de Amostras de Som}
  \item \href{https://www.freesound.org/people/Chem/packs/1995/}{Pack: Distortion guitar samples by Chem}
  \item \href{http://www.cic.unb.br/docentes/lcmm/deptcic2014_1/ics/a20/a20.html}{UnB - Introdução à Computação Sônica, Capítulo 20 - A Linguagem do SOM-A}
\item \href{http://www.ultimate-guitar.com/}{Ultimate Guitar - banco de tablaturas para guitarra}  
\item \href{http://audacity.sourceforge.net/?lang=pt-BR}{Audacity - home page}
\item \href{http://tuxguitar.herac.com.ar/}{TuxGuitar - home page}
\item \href{http://drownedinsound.com/in_depth/4146352-canon-in-the-1990s--from-spiritualized-to-coolio-regurgitating-pachelbels-canon}{Drowned in Sound - artigo sobre Pachelbel's Canon}
  \end{itemize}
\end{frame}


\end{document}
