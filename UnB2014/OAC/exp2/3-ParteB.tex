\section{Aritimética de Ponto Flutuante - Hardware}

\subsection{ULA de Inteiros}
	\subsubsection{Análise de Funções}
A ULA implementada é uma versão bem mais completa do que a vista em sala, com um total de 18 operações. A ULA é uma parte do circuito que tem apenas dois registradores internos, HI e LO e os utiliza para operações de multiplicação e divisão (devido ao fato de essas operações causarem overflow frequentemente). Todas as outras operações têm seus resultados salvos ou no banco de registradores ou na memória. 
Os componentes da ULA são:
\begin{itemize}
\item A e B, números de 32 bits que servem como entrada.
\item Sinal de controle, parâmetro que escolhe a operação a ser realizada.
\item ALUresult, o resultado da operação.
\item Shamt, um parâmetro importante para as operações de sll, srl, e sra, já que nesse número consta o argumento da operação.
\item Zero e Overflow, duas flags que são ligadas caso o resultado da operação atenda aos requisitos.
\item Relógio para habilitar a operação e manter o sistema síncrono.
\item Reset para zerar a memória da ULA (registradores HI e LO).
\end{itemize}
Para manter a tabela o mais limpa possível, apenas selecionamos as operações e colocamos os resultados finais, sem se preocupar com as flags, memória da ULA e casos específicos.
	\begin{table}[H]
	\begin{tabular}{|c|c|c|} \hline
	Sinal de Controle & Operação & ALUresult \\ \hline
0 & AND & A AND B \\ \hline
1 & OR & A OR B \\ \hline
2 & ADD & A + B \\ \hline
3 & MFHI & HI \\ \hline
4 & SLL & B << shamt \\ \hline
5 & MFLO  & LO \\ \hline
6 &  SUB  & A-B \\ \hline
7  & SLT  & A<B: 0 else: 1 \\ \hline
8  & SRL  & B >> shamt \\ \hline
9  & SRA  & B >> shamt (si	gn extended) \\ \hline
10  & XOR  & A XOR B \\ \hline
11  & SLTU  & A < B \\ \hline
12 &  NOR  & A NOR B \\ \hline
13 &  MULT  & A * B \\ \hline
14 &  DIV  & A / B \\ \hline
15 &  LUI  & B << 16 \\ \hline
16  & SLLV & B << A \\ \hline
17 &  SRAV &  B >> A (sign extended) \\ \hline
18  & SRLV  & B >> A \\ \hline
Outro  & Padrão  & 0 \\ \hline
	\end{tabular}
	\caption{Opcode, operação e resultado}
	\end{table}
	
	Casos especiais:
\begin{itemize}
\item mult e div colocam seu resultado tanto em hi/lo quanto na saída, mas o controle não cede permissão de escrita no banco de registradores da CPU com os resultados da operação, fazendo com que seja necessárias as instruções mfhi e mflo.
\item Zero é uma flag que ativa sempre que todos os bits do resultado da operação são nulos.
\item Overflow é uma flag que ativa quando o sinal do resultado não coincide com o sinal que é esperado devido à operação e os sinais dos dados de entrada. Essa é uma propriedade dos números em complemento de dois que facilita e detecção do overflow.
\end{itemize}	


\subsubsection{Verificação}
Para s simulação, foi feita uma pequena quantidade de operações na ULA para verificar os resultados e confirmar seu funcionamento adequado. Testamos todas as operações e os resultados condizem com o esperado.
\begin{figure}[H]
	\centering
	\includegraphics[scale=0.5]{../images/danilo_fig1_new.png}
	\caption{Forma de onda de teste: entrada e saída}
\end{figure}

\begin{figure}[H]
	\centering
	\includegraphics[scale=0.5]{../images/danilo_fig2_new.png}
	\caption{continuação}
\end{figure}

Todas as operações fazem sentido, já que os registradores inicializam vazios e, portanto as operações mfhi e mflo vão resultar em 0 antes de algum comando mult ou div.

\subsubsection{Sintetizando na FPGA}
Foi feito o teste do sistema na FPGA e de fato, foi verificado que as operações da ULA funcionam bem. Devido à pequena quantidade de entradas (temos 18 switches e 4 botões enquanto precisaríamos de 69 formas de entrada apenas  para ter os dois argumentos e o controle), para efeito de teste usamos entradas de 5 bits cada uma e mais o controle de 5 bits também.

O link para o vídeo do teste pode ser visto \href{https://www.youtube.com/watch?v=dkLe0ooJMr4&feature=youtu.be}{aqui}.

\subsubsection{Requisitos}
Fazendo uma análise pelo timequest obtemos os seguintes parâmetros de desempenho:
\begin{itemize}
\item	Numero de elementos lógicos: 3945
\item	Frequência máxima: 6.28 MHz
\item	Setup Time: 153.18 ns
\item	Hold Time: -6.061 ns (negativo? Segundo o google, é possível.)
\item	Propagation Delay: 159.18 ns
\end{itemize}


\subsection{Banco de Registradores da CPU Principal}
	\subsubsection{Construção} 
	O banco de registradores é a memória mais rápida e cara de um processador. A
	arquitetura mips tem um banco de 32 registradores de 4 bytes cada um e esses registradores tem função de leitura e escrita. A implementação pode ser feita como um módulo, retirando as responsabilidades de controle e seleção dos registradores para um módulo de controle separado e deixando para esse módulo apenas a leitura e escrita na sua memória interna.
	
	         \begin{figure}[H]
	                 \centering
	                 \includegraphics[scale = 0.2]{../images/danilo_fig2.png}
	                \caption{banco de registradores}
	         \end{figure}
	         
	         Para esse conjunto de entradas, temos como principais entradas os argumentos de 5 bits que selecionam os registradores para serem lidos ou sofrer a escrita. As outras entradas são o relógio para manter o sincronismo e sinais emitidos pelo controle que dão ou não permissão para resetar todos os registradores (iCLR) e permissão para escrever (iRegWrite).
	        
\subsubsection{Verificação}
  	 	Em seguida fazendo o teste do banco de registradores. Para o teste verificamos que sempre na subida do relógio ocorre a escrita, e por isso os registradores que leem o valor só atualizam o valor um pouco depois da subida do relógio. Verificamos também que a escrita só ocorre se a permissão (iRegWrite) estiver ativa e o RST funciona perfeitamente.
	         \begin{figure}[H]
	                 \centering
	                 \includegraphics[scale = 0.5]{../images/danilo_fig3.png}
	                \caption{teste do banco de registradores}
	         \end{figure}
	\subsubsection{Requisitos}
Fazendo a análise na ferramenta timequest, os parâmetros necessários para o datasheet são:
\begin{itemize}
\item Numero de elementos lógicos: 2149
\item Frequência máxima: 52.23 MHz
\item Setup Time: 10.832 ns
\item Hold Time: -3.5 ns
\item Propagation Delay: 21.692 ns
\end{itemize}

	
	  

\subsection{ULA de Ponto Flutuante}
	\subsubsection{Análise de Funções}
	A FPU tem uma aplicação bem mais específica que a CPU, ela trabalha com operações numéricas, portanto tem menos funcionalidades. A FPU a ser trabalhada tem 12 funções, onde as operações 1-7 são operações numéricas, 8-10 são funções lógicas que apenas alteram o banco de flags e mais duas operações que convertem o tipo de dado guardado para possibilitar a comunicação com a CPU.
	\begin{table}[H]
		\centering 
		\begin{tabular}{|c|c|c|}\hline
		Sinal de Controle & Operação & oresult \\ \hline
1  & ADD  & A+B \\ \hline
2 &  SUB &  A-B \\ \hline
3 &  MUL &  A*B \\ \hline
4  & DIV &  A/B \\ \hline
5 &  SQRT  & \\ \hline
6 &  ABS &  |A| \\ \hline
7  & NEG &  -A \\ \hline
8  & CEQ &  0* \\ \hline
9 &  CLT  & 0* \\ \hline
10  & CLE & 0* \\ \hline
11  & CVTSW &  A single \\ \hline
12  & CVTWS &  A word \\ \hline
		\end{tabular}
		\caption{Funcionalidades da ULA de ponto flutuante}
	\end{table}
	
	\subsubsection{Simulação}
	Para a simulação em forma de onda, coloquei os valores 4 e -4 e verifiquei os cálculos. Os valores são arbitrários e com a principal função de ter como resultados valores que não interferem na mantissa dos resultados e precisar de poucos ciclos (já que tem o mesmo expoente), assim facilitando a visualização e a correção dos erros.
	  \begin{figure}[H]
	                 \centering
	                 \includegraphics[scale = 0.5]{../images/danilo_fig4.png}
	                \caption{teste da ULA de ponto flutuante}
	         \end{figure}
	         
	         \subsubsection{Sintetização da FPGA}
	         Para o funcionamento, fizemos um experimento em verificar as operações com liberdade de alterar um operando e mantendo o outro fixo. O intuito é mostrar de forma simplificada que a FPU funciona de forma adequada sem precisar de muita complexidade no vídeo.
	         O vídeo de teste se encontra \href{
https://www.youtube.com/watch?v=1_VCYLCTwAA&feature=youtu.be}{aqui}
	         
	         \subsubsection{Requisitos}
Ao fazer as simulações com o timequest, obtivemos os seguintes parâmetros de projeto:
	         \begin{itemize}
\item Numero de elementos lógicos: 3488
\item Frequência máxima: 51.97 MHz
\item Setup Time: 16.986 ns
\item Hold Time: -0.387 ns
\item Propagation Delay: 10.372 ns     
	         \end{itemize}

\subsection{Banco de Registradores do Coprocessador 1}

\subsubsection{Construção}
A banco de registradores da FPU é idêntico ao da CPU. A mudança da forma de se escrever os dados (da notação em inteira em complemento de dois para a notação em F.P.) e a ULA mais complicada não altera nada do ponto de vista dos registradores.

\subsubsection{Verificação}
Ele é idêntico ao banco da CPU, portanto o teste que foi aplicado lá funciona igualmente aqui. O banco funciona perfeitamente bem para um processador uniciclo e para um processador multiciclo, bastando o controle definir bem em que momento a permissão de escrita pode ser dada.

\subsubsection{Requisitos}
Ao fazer a análise pelo timequest, observamos pequenas diferenças em relação ao teste da ULA. A razão é que a implementação do banco de registradores não foi escrita exatamente com o mesmo arquivo e por isso os resultados acabam sendo levemente diferentes um do outro.  
\begin{itemize}
\item Numero de elementos lógicos: 2096
\item Frequência máxima: 51.97 MHz
\item Setup Time: 12.859 ns
\item Hold Time: -3.487 ns
\item Propagation Delay: 22.044 ns
\end{itemize}
