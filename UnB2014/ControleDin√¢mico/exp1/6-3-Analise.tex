\subsection{3º Teste}
\paragraph{}
Primeiramente, são deduzidas abaixo as equações referentes ao sistema com realimentação de primeira ordem. O diagrama de blocos referente ao sistema é mostrado abaixo.

\input{diagrams/diagramaA.tex}

\paragraph{}
Para o sistema com realimentação negativa, sabe-se que a função de transferência é da forma:
\begin{equation}
FT(s)=\frac{G(s)\cdot H(s)}{1+H(s)F(s)}
\end{equation}
Onde G(s) é o ganho dado pelo módulo de ganho, H(s) é a função de transferência do sistema motor-tacômetro e F(s) é o ganho de realimentação. Para os experimentos realizados com realimentação, F(s) é tipicamente unitário.

\paragraph{}
Logo, a função de transferência do sistema realimentado pode ser dada por:\\
\begin{equation}
FT(s)=\frac{V_4}{E_1}=\frac{G\cdot \frac{K}{1+sT}}{1+G\cdot\frac{K}{1+sT}}=\frac{G\cdot K}{1+ sT + G\cdot K}
\end{equation}

\begin{equation}
FT(s)=\frac{\frac{G\cdot K}{1 + G\cdot K}}{1+ s\frac{T}{1 + G\cdot K} }
\label{eq:eqDoJuara}
\end{equation}

\paragraph{}
Aplica-se agora o Teorema do valor final para verificar o ganho do sistema realimentado com entrada degrau:\\
\begin{equation}
v_4(\infty)= \lim_{s\rightarrow 0} \cancel{s}\cdot\frac{A}{\cancel{s}} \frac{G\cdot K}{1+ sT + G\cdot K} = \frac{AGK}{1+GK}
\end{equation}

\paragraph{}
Para o primeiro caso, de ganho adicional unitário, tem-se que o valor de $v_4(\infty)$ será dado por:
\begin{equation}
v_4(\infty)=\frac{AGK}{1+GK}|_{G=1}=\frac{AK}{K+1}
\end{equation}

Onde A denota a amplitude do degrau utilizado. Ou seja, o valor de $K_1$ é dado por:\\
\begin{equation}
K_1=\frac{v_4(\infty)}{A}=\frac{K}{K+1}
\end{equation}

Utilizando o valor de K apropriado para a situação (K=0.957), calcula-se um valor teórico para esse ganho $K_1$:
\begin{equation}
\boxed{K_{1 - teórico}=\frac{K}{K+1}=\frac{0.957}{1+0.957}\approx 0.4890}
\end{equation}


O valor prático é obtido da seguinte forma:
\begin{equation}
K_{1 - prático}=\frac{v_4(\infty)}{A}=\frac{6.56}{14.2}
\end{equation}

\begin{equation}
\begin{array}{c}
\boxed{K_{1 - prático} \approx 0.4620} \\ \mbox{ (K de malha fechada sem ganho)}
\end{array}
\end{equation}

O mesmo pode ser feito para o sistema com ganho. Nesse sistema, utilizou-se resistores de 22k$\Omega$ e 2.2k$\Omega$ de forma que o ganho do AmpOp não-inversor totaliza 11, ou seja, G(s)=11.

Dessa forma, temos:\\
\begin{equation}
v_4(\infty)=\frac{AGK}{1+GK}|_{G=11}=\frac{11AK}{11K+1}
\end{equation}

E o novo ganho, denotado $K_2$ é dado por:\\
\begin{equation}
K_2=\frac{v_4(\infty)}{A}=\frac{v_4(\infty)}{E_1}=\frac{11K}{11K+1}
\end{equation}

Realizando o procedimento análogo, temos os seguintes valores teórico e experimental para $K_2$:

\begin{equation}
\begin{array}{l}
    K_{2 - prático} = \frac{V_4}{E_1} = \frac{9.6}{10.4}\\
\end{array}
\end{equation}

\begin{equation}
\begin{array}{c}
\boxed{K_{\mbox{2 - prático}} = 0.865} \\ \mbox{ (K de malha fechada com ganho)}
\end{array}
\end{equation}

\begin{equation}
\begin{array}{l}
    K_{2 - teórico} = \frac{11K}{11K+1} = \frac{11 . 0.957}{11 . 0.957 + 1} \\
    \boxed{K_{\mbox{2 - teórico}} = 0.913 }
\end{array}
\end{equation}

Podemos comparar o erro percentual entre as previsões teóricas e experimentais pela forma:

\begin{equation}
  \epsilon = \frac{|X_{exp} - X_{teo}|}{X_{teo}}
\end{equation}

para os ganhos $K_1$ e $K_2$ temos:

\begin{equation}
\begin{array}{l}
    \epsilon_1 = 4.9\% \\
    \epsilon_2 = 5.3\% \\
\end{array}
\end{equation}

Vemos que os dois erros são relativamente baixos e compatíveis com a precisão das medidas tomadas. Verifica-se aqui concordância entre as previsões teóricas e os resultados experimentais.


Passamos agora para a análise das constantes de tempo. Vemos pela equação \ref{eq:eqDoJuara} que a constante de tempo do sistema será dada por:

\begin{equation}
    \tau ' = \frac{\tau}{1 + G.K}
\end{equation}

para G = 1, temos:

\begin{equation}
\begin{array}{l}
    \tau '_{G = 1} = \frac{\tau}{1 + G.K} = \frac{24.8e-3}{1 + 1 \cdot 0.957} \\
    \boxed{\tau '_{G = 1} = 12.7 \mbox{ms} } \\
    \mbox{($\tau$ ' para malha fechada sem ganho)}
\end{array}
\end{equation}

para G = 11, temos:

\begin{equation}
\begin{array}{l}
    \tau '_{G = 11} = \frac{\tau}{1 + G.K} = \frac{24.8e-3}{1 + 11 \cdot 0.957} \\
    \boxed{\tau '_{G = 11} = 2.2 \mbox{ms} } \\
    \mbox{($\tau$ ' para malha fechada com ganho)}
\end{array}
\end{equation}

Notamos que a adição do ganho diminuiu em torno de 5x a constante de tempo. Podemos resumir o sistema com ganho na função de transferência:

\begin{equation}
\begin{array}{l}
    \boxed{F.T._{\mbox{\begin{tabular}{l}
    1º ordem \\
    G = 11, malha fechada
    \end{tabular} }}
    = \frac{0.865}{1 + 0.0022s}} \\
    \mbox{(F.T. para malha fechada com ganho)}
\end{array}
\end{equation}

e o polo é:

\begin{equation}
\begin{array}{l}
    \boxed{p_{\mbox{\begin{tabular}{l}
    1º ordem \\
    G = 11, malha fechada
    \end{tabular} }} = - 454.55} \\
    \mbox{(polo para malha fechada com ganho)}
\end{array}
\end{equation}