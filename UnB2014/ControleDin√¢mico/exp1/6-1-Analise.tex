\subsection{1º Teste}
\paragraph{}
Pelo teorema do valor final, temos:\\
\begin{equation}
f(\infty)=\lim_{t\rightarrow 0} s\cdot F(s)
\end{equation}

\paragraph{}
Aplicando isso à equação da função de transferência assumindo uma função degrau de entrada com amplitude genérica A, temos:
\begin{equation}
U_{tac}(s)=U_{in}(s)\frac{K_MK_T}{1+sT}=\frac{A}{s}\cdot\frac{K_MK_T}{1+sT} 
\end{equation}

\begin{equation}
\begin{array}{l}
U_{tac}(\infty)=\lim_{s \rightarrow 0}s\cdot F(s)=\lim_{s \rightarrow 0}\cancel{s}\frac{A}{\cancel{s}}\cdot\frac{K_MK_T}{1+sT} \\
=A\cdot K_M\cdot K_T= A\cdot K
\end{array}
\end{equation}

\paragraph{}
Denota-se K para o produto dos coeficientes $K_M$ e $K_T$. Ou seja, é possível calcular o coeficiente K pela fórmula:
\begin{equation}
K=\frac{U_{tac}(\infty)}{A}
\label{eq:procedimento1-2}
\end{equation}
onde A é a amplitude da entrada rm forma de onda quadrada. O resultado é aplicado aos dados da tabela \ref{tab:teste1} e mostrado a seguir:

%COLOCANDO ESSE [H] O ERRO DE COMPILAÇÃO SUMIU
\begin{table}[H]
\centering
\begin{tabular}{|c|c|c|}
\hline
$U_2(V)$ & $V_4(V)$ & $K=\frac{V_4}{U_2}$\\
\hline
6.020 & 5.76 & 0.957\\
\hline
4.140 & 3.84 & 0.928\\
\hline
3.033 & 2.72 & 0.897\\
\hline
1.422 & 1.12 & 0.788\\
\hline
\end{tabular}
\caption{Cálculo de K}
\label{tab:analise1}
\end{table}

Vemos que, em nossas medidas, o ganho total do sistema parece variar com a amplitute do sinal de entrada. Para prosseguirmos devemos escolher um ganho específico. Escolhemos fazer todos os procedimentos então com uma amplitude de 6.00V e para isso:

\begin{equation}
\boxed{
	K = 0.957
    }
\end{equation}

Podemos aplicar o mesmo teorema para a equação \ref{eq:U-W}:

\begin{equation}
\begin{array}{l}
\Omega(s) = U_{in}(s)\frac{K_m}{1 + s \cdot T} \\
w(\infty)=\lim_{s \rightarrow 0}\cancel{s}\frac{A}{\cancel{s}}\cdot\frac{K_M}{1+sT} \\
=A\cdot K_M
\end{array}
\end{equation}

De forma que determinamos $K_M$ por:

\begin{equation}
	K_M = \frac{w(\infty)}{A}
\end{equation}

As contas são feitas na tabela a seguir:

\begin{table}[H]
\centering
\begin{tabular}{|c|c|c|}
\hline
$U_2(V)$ & $\omega (rpm)$ & $\frac{w(\infty)}{A}$(rpm/V)\\
\hline
6.020 & 11500 & 1.9103E03\\
\hline
4.140 & 7600 & 1.8357E3\\
\hline
3.033 & 5500 & 1.8134E3\\
\hline
1.422 & 2120 & 1.4909E3\\
\hline
\end{tabular}
\caption{Cálculo de $K_M$}
\label{tab:teste1-analise2}
\end{table}

Novamente escolhemos o valor de ganho para 6V:

\begin{equation}
	\boxed{K_M = 1910}
\end{equation}

Agora voltamos para $K_t$:

\begin{equation}
\begin{array}{l}
	K = K_MK_t = \frac{V_4(\infty)}{A} \\
    \mbox{, mas }K_M = \frac{w(\infty)}{A}\\
    \Rightarrow K = \frac{w(\infty)}{A} K_t = \frac{V_4(\infty)}{A} \\
    \therefore K_t = \frac{V_4(\infty)}{w(\infty)}
\end{array}
\end{equation}

a tabela mostra os cálculos:

\begin{table}[H]
\centering
\begin{tabular}{|c|c|c|}
\hline
 $\omega (rpm)$ & $V_4(V)$ & $\frac{V_4(\infty)}{w(\infty)}$(V/rpm)\\
\hline
11500 & 5.76 & 0.5009E-3 \\
\hline
 7600 & 3.84 & 0.5053E-3\\
\hline
5500 & 2.72 & 0.4945E-3\\
\hline
2120 & 1.12 & 0.5283E-3\\
\hline
\end{tabular}
\caption{Cálculo de $K_t$}
\label{tab:teste1-analise3}
\end{table}

E então:
\begin{equation}
	\boxed{K_t = 0.5009E-3}
\end{equation}


\paragraph{}
A constante de tempo foi medida utilizando-se o fato de que o intervalo de tempo que ela representa é da partida do sinal até que o mesmo atinja 63\% de seu valor final. O valor medido foi de 

\begin{equation}
    \boxed{\tau = 24.8ms}
\end{equation}
