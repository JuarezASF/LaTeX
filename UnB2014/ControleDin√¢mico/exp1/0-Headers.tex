\documentclass[journal]{IEEEtran}

%% PARA USAR \FLOATBARRIER
\usepackage{float} 

\usepackage{hyperref} % For hyperlinks in the PDF

% The lettrine is the first enlarged letter at the beginning of the text
\usepackage{lettrine} 



%%PARA RODAR UM PT-BR DE BOAS
\usepackage[T1]{fontenc}
\usepackage[utf8]{inputenc}
\usepackage{lmodern}
\usepackage[portuguese]{babel}

%%PARA INCLUIR IMAGENS
\usepackage{graphicx}
\usepackage{fancyhdr}
\pagestyle{fancy}

\usepackage{caption}
\usepackage{subcaption}
\usepackage{placeins}

\usepackage{color}
\usepackage{cancel}

%%PARA EQUAÇÕES DENTRO DE UMA CAIXA E MUITO MAIS
\usepackage{amsmath}
%%para símbolos matemáticos avançados
\usepackage{amssymb}

\usepackage{tikz}
\usetikzlibrary{shapes,arrows}



%%COMANDO PARA IMPRIMIR UMA LINHA VERTICAL MANEIRA
\newcommand{\HRule}{\rule{\linewidth}{0.5mm}}