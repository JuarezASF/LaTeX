\subsection{5º Teste}
\paragraph{}
O sistema de $3^a$ ordem construído em laboratório consiste em adicionar um capacitor em paralelo à realimentação do Amp-Op. Dessa forma, o ganho não-inversor do Amp-Op conta com uma impedância. Isso é demonstrado abaixo:

\paragraph{} O ganho não inversor é dado por:\\
\begin{equation}
FT(s)=\frac{O(s)}{I(s)}=1+\frac{Z_2}{Z_1}
\end{equation}
Onde $Z_4$ é a impedância de realimentação.
No caso, $Z_4$ é dado pela associação em paralelo de um resistor e um capacitor e $Z_3$ é dado por um resistor:
\begin{equation}
FT(s)=1+\frac{\frac{1}{sC}||R_4}{R_3}=1+\frac{\frac{R_4}{R_3}}{1+sCR_4}=\frac{s+\frac{\frac{R_4}{R_3}+1}{R_4C}}{s+\frac{1}{R_4C}}
\end{equation}

Os valores utilizados foram de C=$1\mu F$, $R_4=22k\Omega$ e $R_3=2.2k\Omega$. Dessa forma, a FT do bloco de ganho é dada por:

\begin{equation}
FT_{ganho}=\frac{s+500}{s+45.45}
\end{equation}{}

Que é um bloco de $1^a$ ordem.

Adicionalmente, ainda no módulo dinâmico, há uma divisão de tensão entre $Z_1$ e $Z_2$. Essa divisão, lembrando que $Z_1=2.2k\Omega$ e $Z_2$ é a impedância de um capacitor de $1\mu F$, pode ser modelada da seguinte forma:

\begin{equation}
FT_{divisor}=\frac{Z_2}{Z_1+Z_2}=\frac{\frac{1}{sC}}{\frac{1}{sC}+R_1}
\end{equation}

\begin{equation}
FT_{divisor}=\frac{\frac{1}{R_2C}}{s+\frac{1}{R_2C}}=\frac{454.54}{s+454.54}
\end{equation}

Por fim, a FT do motor-tacômetro é obtida utilizando-se os valores de K e T:

\begin{equation}
FT_{sistema}=\frac{K}{1+sT}|_{K=0.957, T=24.8ms}=\frac{38.59}{s+40.32}
\end{equation}


\paragraph{}
O sistema foi modelado a partir da \emph{toolbox} Simulink do software Matlab e sua simulação é mostrada abaixo:


\begin{figure*}[ht]
\centering
\includegraphics[scale = 0.6]{images/matlab_diagram.png}
\caption{Diagrama de Blocos do sistema de 3a ordem.}
\label{fig:blockdiagram-3a-ordem}
\end{figure*}


\begin{figure}[H]
\centering
\includegraphics[width = 0.4\textwidth, height=6cm,keepaspectratio]{images/matlab_resposta_2.png}
\caption{Resposta ao sistema de 3a ordem obtida em simulação.}
\label{fig:3a-ordem-resposta-simulada}
\end{figure}

Novamente, com o matlab a função de transferência é calculada:

\begin{equation}
\begin{array}{l}
    \boxed{F.T.
    = \frac{46.32 s + 2.316e04}{  0.02641 s^3 + 3.465 s^2 + 197.7 s + 2.536e04}} \\
    \mbox{(F.T. para malha fechada de 3ª ordem)}
\end{array}
\end{equation}


e os polos:

\begin{equation}
\begin{array}{l}
    \boxed{p =    1.0e+02 * 
    \left[ \begin{array}{l}
  -1.3033          \\
  -0.0045 + 0.8584i \\
  -0.0045 - 0.8584i \\
    \end{array} \right]
} \\
    \mbox{(polos para malha fechada de 3ª ordem)}
\end{array}
\end{equation}

Vemos que um dos polos possui parte real  mais negativa que os outros dois. Isso faz com que sua resposta caia muito mais rapidamente que a resposta aos outros dois polos e seu efeito não é facilmente observável. Mesmo assim a resposta do sistema é altereada devido ao terceiro polo.

Notamos que a simulação obtida difere um pouco do comportamento observado em laboratório, mas que, em ambos os casos, a adição do terceiro polo aumenta a oscilação do sistema.

