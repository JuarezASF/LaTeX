EM CASO DE QUERERMOS VOLTAR COM PARTES DELETADAS.

PARTE QUE O RODRIGO COLOCOU DE COMO MEDIR A CONSTANTE DE TEMPO:

\begin{figure}[H]
  \centering
	\includegraphics[width = 0.4\textwidth, height=6cm,keepaspectratio]{images/dia1_respostaAmortecida_vpp.jpg}
\caption{Medição do valor pico a pico do pulso.}
\label{fig:dia1-1ordem-vpp}
\end{figure}

\begin{figure}[H]
  \centering
	\includegraphics[width = 0.4\textwidth, height=6cm,keepaspectratio]{images/dia1_63percentVpp.jpg}
\caption{Medição equivalente a 63\% do valor pico a pico.}
\label{fig:dia1-1ordem-063Vpp}
\end{figure}

\begin{figure}[H]
  \centering
	\includegraphics[width = 0.4\textwidth, height=6cm,keepaspectratio]{images/dia1_respostaAmortecida_cstTempo.jpg}
\caption{Medição da constante de tempo.}
\label{fig:dia1-1ordem-cteTempo}
\end{figure}
