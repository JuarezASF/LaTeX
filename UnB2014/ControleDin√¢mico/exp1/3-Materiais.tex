\section{Materiais}

Foram utilizados durante o experimento:

\begin{itemize}
	\item kit Motor - Tacômetro nº 6
    \item Osciloscópio
    \item Gerador de Sinal
    \item Estroboscópio
    \item Cabos Conectores
\end{itemize}

As figuras em \ref{fig:montagem} e \ref{fig:kit} a seguir ilustram os equipamentos utilizados no experimento. Em \ref{fig:kit} vemos o kit motor-tacômetro utilizado. Na parte superior da figura vemos o motor e o tacômetro propriamente ditos, em baixo deles vemos um esquemático explicativo dos bornes, e em baixo vemos em preto e em vermelho os bornes de conexão. Os sinais de entrada, medidas e realimentação são feitos por meio dos bornes. No painel esquemático existem conectores por onde podemos adicionar elementos resistivos e capacitivos externamente. Os bornes são nomeados e as funcionalidades dos principais bornes utilizados são descritas a seguir:

\begin{itemize}
    \item $E_1$ : entrada positiva do bloco somador 
    \item $E_2$ : entrada negativa do bloco somador

	\item $U_1$ : entrada do módulo amplificador e saída do somador
    \item $U_2$ : sinal que vai para o drive do motor
    
    \item $V_1$ : saída do módulo amplificador
	\item $V_4$ : saída do tacômetro

    \item com : é o terra do circuito
\end{itemize}

\begin{figure}[H]
  \centering
	\includegraphics[width = 0.4\textwidth, height=6cm,keepaspectratio]{images/equipamentos.jpg}
\caption{Equipamentos utilizados}
\label{fig:montagem}
\end{figure}

\begin{figure}[H]
  \centering
	\includegraphics[width = 0.4\textwidth, height=6cm,keepaspectratio, angle = 90]{images/modulo_experimental.jpg}
\caption{kit motor-tacômetro}
\label{fig:kit}
\end{figure}