\section{Procedimentos}
O experimento é dividido em três momentos: inicialmente projeta-se e implementa-se compensadores analógicos de forma a superar o compensador padrão já disponível no kit. Nessa etapa implementa-se 4 compensadores: de avanço, de avanço otimizado, PD e PD otimizado. A ideia dos compensadores otimizados é colocar os zeros do compensador justamente em cima de um dos polos da planta. Em um segundo momento apresenta-se o compensador digital e observa-se a sua funcionalidade. Na terceira etapa, os mesmos compensadores implementados na primeira etapa são implementados digitalmente e, finalmente, projeta-se e implementa-se um compensador PID pelo método de Ziegler-Nichols.

Em todas as etapas e em todos os projetos, salva-se imagens do LGR e da resposta ao degrau simulada em projeto e da saída real do osciloscópio obtida mostrando os canais Ss1 e Vref1.