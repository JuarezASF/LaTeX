\section{Conclusão}
 O experimento permitiu observar a validade e as dificuldades
 do projeto de compensadores para uma planta real. Observou-se que os resultados obtidos não são exatamente aqueles projetados, mas que, mesmo assim, é possível lidar com essas imperfeições e  controlar um sistema inicialmente instável. Observou-se também a praticidade e facilidade de se utilizar compensadores digitais. Uma vez feito o projeto pelo LGR, os parâmetros do compensador digital foram rapidamente definidos e implementados. Pôde-se projetar e implementar compensadores de avanço, PD e PID pelo método de Zigler-Nichols, sendo que os dois primeiros foram feitos analogicamente e digitalmente e o último apenas digitalmente. Finalmente, observamos que a estabilidade do sistema compensado digitalmente foi superior aquela do sistema compensado analogicamente.