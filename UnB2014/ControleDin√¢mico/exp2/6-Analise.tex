\section{Análise de Dados}
Inicialmente determinamos os parâmetros $K_r$, $K_R$, $K_{ifsr}$ e $K_{f}$  a partir dos dados calculados:
\begin{itemize}
\item \textbf{$K_s$} pode ser determinado a partir da fórmula:
\begin{equation}
     K_s = \frac{S_{Smax} - S_{Smin}}{d}
\end{equation}

\begin{equation}
    \boxed{
	     $K_s$ = 1031.58 V/m 
        }
\end{equation}

\item \textbf{$K_R$} pode ser determinado a partir da fórmula:
\begin{equation}
     K_R = m \omega_r^2
\end{equation}
\begin{equation}
    \boxed{
         $K_R$ = 53.6243 $\frac{kg}{s^2}$
        }
\end{equation}


\item \textbf{$K_{ifsr}$}


O cálculo de $K_{ifsr}$ é feito pela fórmula:
\begin{equation}
    K_{ifsr} = \frac{S_s}{I_b} \cdot \left( 1 + (\frac{2 \pi f}{\omega_r}^2)\right)
\end{equation}

O cálculo é feito para os valores no domínio do tempo e da frequência e apresentados nas tabelas a seguir:

\begin{table}[H]
\centering
\begin{tabular}{|c|c|c|c|}
\hline
$f(Hz)$ & $\theta$ & $\frac{S_S}{I_b}$ & $K_{ifsr}$  \\
\hline
0 & 0 & 51.25 & 51.25\\
\hline
5 & $0^o$ &6.852 &41.028  \\
\hline
10 &$0^o$ &2.9 & 86.075\\
\hline
15 &$0^o$ &1.58 & 72.507 \\
\hline
20 &$0^o$ &0.735 &59.415 \\
\hline
25 &$0^o$ &0.653 &82.137 \\
\hline
30 &$0^o$ &0.607 &109.686 \\
\hline
\end{tabular}
\caption{Cálculo de $K_{ifrs}$ no domínio do tempo}
\label{tab:ch1-tempo2}
\end{table}

\begin{table}[H]
\centering
\begin{tabular}{|c|c|c|}
\hline
$f(Hz)$ & $\frac{S_S}{I_b}$ & $K_{ifsr}$  \\
\hline
0 & 51.25 & 51.25\\
\hline
5 &11.11 & 66.53  \\
\hline
10 &3.224 &67.540   \\
\hline
15 &1.5845 & 72.713 \\
\hline
20 &0.682 & 55.094 \\
\hline
25 &0.613 & 77.039 \\
\hline
30 &0.544 & 98.246 \\
\hline
\end{tabular}
\caption{Cálculo de $K_{ifrs}$ no domínio da frequeência}
\label{tab:ch1-freq2}
\end{table}

Vemos que os valores do parâmetro $K_{ifsr}$ variam consideravelmente, é necessário então calcular a média e o desvio padrão dessa medida para cada uma das situações: domínio do tempo e domínio da frequência.

Os valores obtidos foram:\\
\begin{itemize}
\item 
Para o domínio do tempo:\\
    \begin{equation}
        \overline{K_{ifsr}}= \mu =  71.7283
        \end{equation}
    
    \begin{equation}
        \sigma_{Kifsr}= 23.3418
    \end{equation}

\item
Para o domínio da frequência:\\
    \begin{equation}
        \overline{K_{ifsr}}= \mu =  69.7731
    \end{equation}

    \begin{equation}
        \sigma_{Kifsr}= 15.5254
    \end{equation}
\end{itemize}
\item \textbf{$K_{f}$}


O valor de $K_{if}$ é obtido a partir da seguinte equação:\\
\begin{equation}
K_{ifsr}=\frac{-K_{if}K_s}{k_r} \Rightarrow K_{if}=\frac{-K_{ifsr}\cdot 53.6243}{1031.58}
\end{equation}


Como $K_{if}$ é obtido a partir de $K_{ifsr}$ multiplicado por constante, a distribuição probabilística dos valores não é afetada e a média e a variância de $K_{if}$ podem ser calculadas a partir da média e variância de $K_{ifsr}$ pela mesma fórmula. Dessa forma, tem-se:\\
\begin{itemize}
\item 
No domínio do tempo:
    \begin{equation}
        \overline{K_{if}} = -3.7286
    \end{equation}

    \begin{equation}
        |\sigma_K|=1.2134
    \end{equation}
\item 
No domínio da frequência:\\
    \begin{equation}
        \overline{K_{if}} =-3.6270
    \end{equation}

    \begin{equation}
        |\sigma_K|= 0.8071
    \end{equation}
\end{itemize}

Por fim, a partir do princípio de máxima verossimilhança, obtém-se um valor único de média e variância combinadas:\\
\begin{equation}
        \overline{K_{if}}=\frac{\frac{x_A}{\sigma_A^2}+\frac{x_B}{\sigma_B^2}}{\frac{1}{\sigma_A^2}+\frac{1}{\sigma_B^2}}
\end{equation}

\begin{equation}
    \boxed{
        \overline{K_{if}}= -3.6582
    }
\end{equation}

\begin{equation}
\sigma_{Kif}^2=\left(\frac{1}{\sigma_1^2}+\frac{1}{\sigma_2^2} \right)^{-1} \approx 0.4516
\end{equation}
Logo, 
\begin{equation}
\sigma_{Kif}\approx 0.672
\end{equation}

\end{itemize}

Calcula-se agora a função de transferência G(s) do compensador. Sabe-se que ela é dada por uma divisão de tensão por impedâncias:\\
\begin{equation}
\frac{E_P}{S_M}=\frac{12k\Omega}{12k\Omega+(47k\Omega|| \frac{1}{s\cdot C})}
\end{equation}
Onde $C=1\mu F$.
\begin{equation}
    \boxed{
        \frac{E_P}{S_M}=\approx \frac{s+21.277}{s+104.601}
    }
\end{equation}



Nota-se que o compensador acima é um compensador de avanço de fase.

Tendo obtidos todos os parâmetros descritivos do sistema, é possível montar o seguinte diagrama de blocos para simulação no simulink:\\

\begin{figure}[H]
\centering
\includegraphics[scale=0.60]{images/simulink-blocks.jpg}
\caption{Diagrama de blocos utilizado para simulação.}
\label{fig:simulink_blocks}
\end{figure}

Simulando a entrada para uma das frequências utilizadas (5Hz, no caso), obtém-se a seguinte resposta, onde os sinais mostrados são Ib e Ss:\\

\begin{figure}[H]
\centering
\includegraphics{images/simulation.jpg}
\caption{Resposta simulada do sistema.}
\label{fig:simulation}
\end{figure}

Para efeito de comparação, é mostrada abaixo a resposta real obtida em laboratório:\\

\begin{figure}[H]
\centering
\includegraphics[scale=0.1]{images/onda3.jpg}
\caption{Resposta real do sistema.}
\label{fig:simulation-real}
\end{figure}

Nota-se que o ganho e a defasagem dos gráficos é extremamente similar ao que foi obtido em laboratório. Dessa forma, conclui-se que a modelagem do sistema se deu de forma efetiva.
