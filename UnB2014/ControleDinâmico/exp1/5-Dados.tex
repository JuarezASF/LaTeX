\section{Dados Experimentais}
Para ordenar a apresentação os dados são dispostos na ordem das etapas do experimento.
\subsection{1º Teste}
\paragraph{}
Abaixo é mostrada a tabela com os valores de entrada e saída obtidos, assim como os valores calculados para K.


%COLOCANDO ESSE [H] O ERRO DE COMPILAÇÃO SUMIU
\begin{table}[H]
\centering
\begin{tabular}{|c|c|c|}
\hline
$U_2(V)$ & $\omega (rpm)$ & $V_4(V)$\\
\hline
6.020 & 11500 & 5.76\\
\hline
4.140 & 7600 & 3.84\\
\hline
3.033 & 5500 & 2.72\\
\hline
1.422 & 2120 & 1.12\\
\hline
\end{tabular}
\caption{Dados da parte 1.}
\label{tab:teste1}
\end{table}

Em seguida, mediu-se o intervalo de tempo para que a entrada atingisse 63\% de seu valor final:


\begin{equation}
\boxed{\tau=24.8ms}
\end{equation}


\subsection{2º Teste}
As figuras \ref{fig:dadosB_nao_linearidade} e \ref{fig:dadosB_saturacao} a seguir mostram os resultados obtidos.

\begin{figure}[H]
  \centering
	\includegraphics[width = 0.4\textwidth, height=6cm,keepaspectratio]{images/nao_linearidade.jpg}
\caption{zona morta}
\label{fig:dadosB_nao_linearidade}
\end{figure}

\begin{figure}[H]
  \centering
	\includegraphics[width = 0.4\textwidth, height=6cm,keepaspectratio]{images/saturacao.jpg}
\caption{saturação}
\label{fig:dadosB_saturacao}
\end{figure}

\subsection{3º Teste}
Os dados dos procedimentos são mostrados na tabela a seguir:

%COLOCANDO ESSE [H] O ERRO DE COMPILAÇÃO SUMIU
\begin{table}[H]
\centering
\begin{tabular}{|c|c|c|}
\hline
			& $E_{1}$ & $V_4$\\  \hline
sem ganho 	& 14.2 V & 6.56 \\ \hline
com ganho 	& 10.4   & 9.0 \\ \hline
\end{tabular}
\caption{Dados da parte 3}
\label{tab:teste3}
\end{table}

\subsection{4º Teste}
Observou-se uma resposta transitória característica de segunda ordem sub-amortecida ao acrescentar o ganho. A resposta do sistema sem ganho foi sobre-amortecida. Para o último caso aproximamos medimos apenas a constante de tempo:

\begin{equation}
\begin{array}{cc}
    \boxed{\tau = 29 ms} & \mbox{, resposta amortecida} \\
\end{array}
\end{equation}

Para o caso sub-amortecido medimos o sobrevalor percentual e o tempo de pico:
\begin{equation}
\begin{array}{l}
    O.S. \% = 18.681 \% \\
    t_p = 52 ms \\
\end{array}
\end{equation}

\subsection{5º Teste}
O resultado da construção desse sistema é mostrado abaixo:\\

\begin{figure}[H]
  \centering
	\includegraphics[width = 0.4\textwidth, height=6cm,keepaspectratio]{images/dia3_resposta_ordem3.jpg}
\caption{Comportamento em osciloscópio do sistema de $3^a$ ordem.}
\label{fig:3a-ordem-response}
\end{figure}