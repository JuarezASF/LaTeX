\section{Análise de Dados}
\subsection{Considerações Prévias}
A seguir são demonstradas algumas relações que auxiliam o entendimento do experimento.
Primeiramente, são apresentadas as funções de transferência:

Para o sistema de primeira ordem, é sabido:
\begin{equation}
H(s)=\frac{V_4}{U_2}= \frac{K_MK_T}{1+sT}= \frac{K}{1+sT}
\end{equation}

O mesmo sistema, agora com um ganho G constante, e em \emph{feedback} negativo, é dado por:
\begin{equation}
H_1(s)=\frac{V_4}{E_1}=\frac{G\cdot H(s)}{1+G\cdot H(s)}=\frac{GK}{1+sT+GK}
\end{equation}

O polo desse sistema é $p_1=-\frac{1+GK}{T}$.

Em seguida, adicionando um bloco de primeira ordem no caminho direto (dado por $D(s)=\frac{1}{1+sR_1C_2}$), desenvolvemos a seguinte expressão para o sistema de $2^a$ ordem de malha fechada:

\begin{equation}
H_2(s)=\frac{V_4}{E_1}=\frac{D(s)\cdot H_1(s)}{1+D(s)\cdot H_1(s)}=
\end{equation}

\begin{equation}
=\frac{\frac{1}{1+sR_1C_2}\cdot \frac{GK}{1+sT+GK}}{1+\frac{1}{1+sR_1C_2}\cdot \frac{GK}{1+sT+GK}}
\end{equation}

Logo,
\begin{equation}
H_2(s)=\frac{V_4}{E_1}=\frac{KG}{s^2(R_1C_2T)+s(R_1C_2+T)+(1+KG)}
\end{equation}

Os pólos apresentados por esse sistema são:
\begin{equation}
\begin{array}{c}
 \frac{-R_1C_2+T+\sqrt{(R_1C_2+T)^2-4R_1C_2T(1+KG)}}{2R_1C_2T}  \\
 \frac{-R_1C_2+T-\sqrt{(R_1C_2+T)^2-4R_1C_2T(1+KG)}}{2R_1C_2T}
\end{array}
\end{equation}

Por fim, para calcular a FT do sistema de 3a ordem, basta substituir o ganho G, antes constante, por um ganho com impedância:

\begin{equation}
G(s)=1+\frac{Z}{R_3}=\frac{s+\frac{\frac{R_4}{R_3}+1}{R_4C}}{s+\frac{1}{R_4C}}=\frac{s+\beta}{s+\alpha}
\end{equation}

Fazendo essa substituição, encontra-se o sistema de 3a ordem:

\begin{equation}
H_3=\frac{V_4}{E_1}=
\end{equation}
\begin{equation}
=\frac{(s+\beta)K}{(1+sR_1C_2)(s+\alpha)(1+sT)+K(s+\beta)}
\end{equation}

\subsection{1º Teste}
\paragraph{}
Pelo teorema do valor final, temos:\\
\begin{equation}
f(\infty)=\lim_{t\rightarrow 0} s\cdot F(s)
\end{equation}

\paragraph{}
Aplicando isso à equação da função de transferência assumindo uma função degrau de entrada com amplitude genérica A, temos:
\begin{equation}
U_{tac}(s)=U_{in}(s)\frac{K_MK_T}{1+sT}=\frac{A}{s}\cdot\frac{K_MK_T}{1+sT} 
\end{equation}

\begin{equation}
\begin{array}{l}
U_{tac}(\infty)=\lim_{s \rightarrow 0}s\cdot F(s)=\lim_{s \rightarrow 0}\cancel{s}\frac{A}{\cancel{s}}\cdot\frac{K_MK_T}{1+sT} \\
=A\cdot K_M\cdot K_T= A\cdot K
\end{array}
\end{equation}

\paragraph{}
Denota-se K para o produto dos coeficientes $K_M$ e $K_T$. Ou seja, é possível calcular o coeficiente K pela fórmula:
\begin{equation}
K=\frac{U_{tac}(\infty)}{A}
\label{eq:procedimento1-2}
\end{equation}
onde A é a amplitude da entrada rm forma de onda quadrada. O resultado é aplicado aos dados da tabela \ref{tab:teste1} e mostrado a seguir:

%COLOCANDO ESSE [H] O ERRO DE COMPILAÇÃO SUMIU
\begin{table}[H]
\centering
\begin{tabular}{|c|c|c|}
\hline
$U_2(V)$ & $V_4(V)$ & $K=\frac{V_4}{U_2}$\\
\hline
6.020 & 5.76 & 0.957\\
\hline
4.140 & 3.84 & 0.928\\
\hline
3.033 & 2.72 & 0.897\\
\hline
1.422 & 1.12 & 0.788\\
\hline
\end{tabular}
\caption{Cálculo de K}
\label{tab:analise1}
\end{table}

Vemos que, em nossas medidas, o ganho total do sistema parece variar com a amplitute do sinal de entrada. Para prosseguirmos devemos escolher um ganho específico. Escolhemos fazer todos os procedimentos então com uma amplitude de 6.00V e para isso:

\begin{equation}
\boxed{
	K = 0.957
    }
\end{equation}

Podemos aplicar o mesmo teorema para a equação \ref{eq:U-W}:

\begin{equation}
\begin{array}{l}
\Omega(s) = U_{in}(s)\frac{K_m}{1 + s \cdot T} \\
w(\infty)=\lim_{s \rightarrow 0}\cancel{s}\frac{A}{\cancel{s}}\cdot\frac{K_M}{1+sT} \\
=A\cdot K_M
\end{array}
\end{equation}

De forma que determinamos $K_M$ por:

\begin{equation}
	K_M = \frac{w(\infty)}{A}
\end{equation}

As contas são feitas na tabela a seguir:

\begin{table}[H]
\centering
\begin{tabular}{|c|c|c|}
\hline
$U_2(V)$ & $\omega (rpm)$ & $\frac{w(\infty)}{A}$(rpm/V)\\
\hline
6.020 & 11500 & 1.9103E03\\
\hline
4.140 & 7600 & 1.8357E3\\
\hline
3.033 & 5500 & 1.8134E3\\
\hline
1.422 & 2120 & 1.4909E3\\
\hline
\end{tabular}
\caption{Cálculo de $K_M$}
\label{tab:teste1-analise2}
\end{table}

Novamente escolhemos o valor de ganho para 6V:

\begin{equation}
	\boxed{K_M = 1910}
\end{equation}

Agora voltamos para $K_t$:

\begin{equation}
\begin{array}{l}
	K = K_MK_t = \frac{V_4(\infty)}{A} \\
    \mbox{, mas }K_M = \frac{w(\infty)}{A}\\
    \Rightarrow K = \frac{w(\infty)}{A} K_t = \frac{V_4(\infty)}{A} \\
    \therefore K_t = \frac{V_4(\infty)}{w(\infty)}
\end{array}
\end{equation}

a tabela mostra os cálculos:

\begin{table}[H]
\centering
\begin{tabular}{|c|c|c|}
\hline
 $\omega (rpm)$ & $V_4(V)$ & $\frac{V_4(\infty)}{w(\infty)}$(V/rpm)\\
\hline
11500 & 5.76 & 0.5009E-3 \\
\hline
 7600 & 3.84 & 0.5053E-3\\
\hline
5500 & 2.72 & 0.4945E-3\\
\hline
2120 & 1.12 & 0.5283E-3\\
\hline
\end{tabular}
\caption{Cálculo de $K_t$}
\label{tab:teste1-analise3}
\end{table}

E então:
\begin{equation}
	\boxed{K_t = 0.5009E-3}
\end{equation}


\paragraph{}
A constante de tempo foi medida utilizando-se o fato de que o intervalo de tempo que ela representa é da partida do sinal até que o mesmo atinja 63\% de seu valor final. O valor medido foi de 

\begin{equation}
    \boxed{\tau = 24.8ms}
\end{equation}

\subsection{2º Teste}
\paragraph{}
O motor possui um efeito de zona morta que pode ser explicado da seguinte forma: 

\begin{itemize}
\item Para baixas tensões, é necessária uma tensão mínima para que ele comece a girar.
\item Para altas tensões, há uma tensão máxima a partir da qual a velocidade do motor para de acompanhar a tensão de entrada devido às limitações físicas do motor devidas à construção do mesmo e outros fatores, como o atrito viscoso em seu interior.
\end{itemize}

\paragraph{}
Isso se traduz no ceifamento da tensão de saída do tacômetro para baixas e altas tensões de entrada.
\subsection{3º Teste}
\paragraph{}
Primeiramente, são deduzidas abaixo as equações referentes ao sistema com realimentação de primeira ordem. O diagrama de blocos referente ao sistema é mostrado abaixo.

\input{diagrams/diagramaA.tex}

\paragraph{}
Para o sistema com realimentação negativa, sabe-se que a função de transferência é da forma:
\begin{equation}
FT(s)=\frac{G(s)\cdot H(s)}{1+H(s)F(s)}
\end{equation}
Onde G(s) é o ganho dado pelo módulo de ganho, H(s) é a função de transferência do sistema motor-tacômetro e F(s) é o ganho de realimentação. Para os experimentos realizados com realimentação, F(s) é tipicamente unitário.

\paragraph{}
Logo, a função de transferência do sistema realimentado pode ser dada por:\\
\begin{equation}
FT(s)=\frac{V_4}{E_1}=\frac{G\cdot \frac{K}{1+sT}}{1+G\cdot\frac{K}{1+sT}}=\frac{G\cdot K}{1+ sT + G\cdot K}
\end{equation}

\begin{equation}
FT(s)=\frac{\frac{G\cdot K}{1 + G\cdot K}}{1+ s\frac{T}{1 + G\cdot K} }
\label{eq:eqDoJuara}
\end{equation}

\paragraph{}
Aplica-se agora o Teorema do valor final para verificar o ganho do sistema realimentado com entrada degrau:\\
\begin{equation}
v_4(\infty)= \lim_{s\rightarrow 0} \cancel{s}\cdot\frac{A}{\cancel{s}} \frac{G\cdot K}{1+ sT + G\cdot K} = \frac{AGK}{1+GK}
\end{equation}

\paragraph{}
Para o primeiro caso, de ganho adicional unitário, tem-se que o valor de $v_4(\infty)$ será dado por:
\begin{equation}
v_4(\infty)=\frac{AGK}{1+GK}|_{G=1}=\frac{AK}{K+1}
\end{equation}

Onde A denota a amplitude do degrau utilizado. Ou seja, o valor de $K_1$ é dado por:\\
\begin{equation}
K_1=\frac{v_4(\infty)}{A}=\frac{K}{K+1}
\end{equation}

Utilizando o valor de K apropriado para a situação (K=0.957), calcula-se um valor teórico para esse ganho $K_1$:
\begin{equation}
\boxed{K_{1 - teórico}=\frac{K}{K+1}=\frac{0.957}{1+0.957}\approx 0.4890}
\end{equation}


O valor prático é obtido da seguinte forma:
\begin{equation}
K_{1 - prático}=\frac{v_4(\infty)}{A}=\frac{6.56}{14.2}
\end{equation}

\begin{equation}
\begin{array}{c}
\boxed{K_{1 - prático} \approx 0.4620} \\ \mbox{ (K de malha fechada sem ganho)}
\end{array}
\end{equation}

O mesmo pode ser feito para o sistema com ganho. Nesse sistema, utilizou-se resistores de 22k$\Omega$ e 2.2k$\Omega$ de forma que o ganho do AmpOp não-inversor totaliza 11, ou seja, G(s)=11.

Dessa forma, temos:\\
\begin{equation}
v_4(\infty)=\frac{AGK}{1+GK}|_{G=11}=\frac{11AK}{11K+1}
\end{equation}

E o novo ganho, denotado $K_2$ é dado por:\\
\begin{equation}
K_2=\frac{v_4(\infty)}{A}=\frac{v_4(\infty)}{E_1}=\frac{11K}{11K+1}
\end{equation}

Realizando o procedimento análogo, temos os seguintes valores teórico e experimental para $K_2$:

\begin{equation}
\begin{array}{l}
    K_{2 - prático} = \frac{V_4}{E_1} = \frac{9.6}{10.4}\\
\end{array}
\end{equation}

\begin{equation}
\begin{array}{c}
\boxed{K_{\mbox{2 - prático}} = 0.865} \\ \mbox{ (K de malha fechada com ganho)}
\end{array}
\end{equation}

\begin{equation}
\begin{array}{l}
    K_{2 - teórico} = \frac{11K}{11K+1} = \frac{11 . 0.957}{11 . 0.957 + 1} \\
    \boxed{K_{\mbox{2 - teórico}} = 0.913 }
\end{array}
\end{equation}

Podemos comparar o erro percentual entre as previsões teóricas e experimentais pela forma:

\begin{equation}
  \epsilon = \frac{|X_{exp} - X_{teo}|}{X_{teo}}
\end{equation}

para os ganhos $K_1$ e $K_2$ temos:

\begin{equation}
\begin{array}{l}
    \epsilon_1 = 4.9\% \\
    \epsilon_2 = 5.3\% \\
\end{array}
\end{equation}

Vemos que os dois erros são relativamente baixos e compatíveis com a precisão das medidas tomadas. Verifica-se aqui concordância entre as previsões teóricas e os resultados experimentais.


Passamos agora para a análise das constantes de tempo. Vemos pela equação \ref{eq:eqDoJuara} que a constante de tempo do sistema será dada por:

\begin{equation}
    \tau ' = \frac{\tau}{1 + G.K}
\end{equation}

para G = 1, temos:

\begin{equation}
\begin{array}{l}
    \tau '_{G = 1} = \frac{\tau}{1 + G.K} = \frac{24.8e-3}{1 + 1 \cdot 0.957} \\
    \boxed{\tau '_{G = 1} = 12.7 \mbox{ms} } \\
    \mbox{($\tau$ ' para malha fechada sem ganho)}
\end{array}
\end{equation}

para G = 11, temos:

\begin{equation}
\begin{array}{l}
    \tau '_{G = 11} = \frac{\tau}{1 + G.K} = \frac{24.8e-3}{1 + 11 \cdot 0.957} \\
    \boxed{\tau '_{G = 11} = 2.2 \mbox{ms} } \\
    \mbox{($\tau$ ' para malha fechada com ganho)}
\end{array}
\end{equation}

Notamos que a adição do ganho diminuiu em torno de 5x a constante de tempo. Podemos resumir o sistema com ganho na função de transferência:

\begin{equation}
\begin{array}{l}
    \boxed{F.T._{\mbox{\begin{tabular}{l}
    1º ordem \\
    G = 11, malha fechada
    \end{tabular} }}
    = \frac{0.865}{1 + 0.0022s}} \\
    \mbox{(F.T. para malha fechada com ganho)}
\end{array}
\end{equation}

e o polo é:

\begin{equation}
\begin{array}{l}
    \boxed{p_{\mbox{\begin{tabular}{l}
    1º ordem \\
    G = 11, malha fechada
    \end{tabular} }} = - 454.55} \\
    \mbox{(polo para malha fechada com ganho)}
\end{array}
\end{equation}
\subsection{4º Teste}

Para o caso com ganho, que gerou uma resposta sub-amortecida, podemos calcular o fator de amortecimento $\xi$ e a frequência natural $w_n$:

\begin{equation}
\begin{array}{l}
    \xi = \frac{-ln(O.S.)}{\sqrt{\pi^2 + \ln^2(O.S.)}} \\
    \xi = \frac{-ln(0.18681)}{\sqrt{\pi^2 + \ln^2(0.18681)}} \\
    \boxed{\xi = 0.471}
\end{array}
\end{equation}

\begin{equation}
\begin{array}{l}
    w_n = \frac{\pi}{T_p \sqrt{1 - \xi^2}} \\
    w_n = \frac{\pi}{0.052 \sqrt{1 - 0.471^2}}\\
    \boxed{w_n = 68.488 \mbox{  rad/sec}}
\end{array}
\end{equation}

Para o caso em que a resposta foi sobre-amortecida, podemos aproximar o sistema de segunda ordem por um de primeira. Um sistema de primeira ordem está caracterizado pela sua constante de tempo. Para esse sistema temos então:

\begin{equation}
    \boxed{\tau = 29 ms}
\end{equation}

Com a ajuda da rotina feedback() do matlab a função de transferência é calculada:

\begin{equation}
\begin{array}{l}
    \boxed{F.T.
    = \frac{10.53}{4.96e-5 s^2 + 0.0268 s + 11.53}} \\
    \mbox{(F.T. para malha fechada de 2ª ordem com G = 11)}
\end{array}
\end{equation}

com o comando pole(), também do matlab, obtém-se os polos da função

\begin{equation}
\begin{array}{l}
    \boxed{p = 1.0e+02( -2.7016 \pm 3.9926i)
} \\
    \mbox{(polos para malha fechada de 2ª ordem com G = 11)}
\end{array}
\end{equation}


\subsection{5º Teste}
\paragraph{}
O sistema de $3^a$ ordem construído em laboratório consiste em adicionar um capacitor em paralelo à realimentação do Amp-Op. Dessa forma, o ganho não-inversor do Amp-Op conta com uma impedância. Isso é demonstrado abaixo:

\paragraph{} O ganho não inversor é dado por:\\
\begin{equation}
FT(s)=\frac{O(s)}{I(s)}=1+\frac{Z_2}{Z_1}
\end{equation}
Onde $Z_4$ é a impedância de realimentação.
No caso, $Z_4$ é dado pela associação em paralelo de um resistor e um capacitor e $Z_3$ é dado por um resistor:
\begin{equation}
FT(s)=1+\frac{\frac{1}{sC}||R_4}{R_3}=1+\frac{\frac{R_4}{R_3}}{1+sCR_4}=\frac{s+\frac{\frac{R_4}{R_3}+1}{R_4C}}{s+\frac{1}{R_4C}}
\end{equation}

Os valores utilizados foram de C=$1\mu F$, $R_4=22k\Omega$ e $R_3=2.2k\Omega$. Dessa forma, a FT do bloco de ganho é dada por:

\begin{equation}
FT_{ganho}=\frac{s+500}{s+45.45}
\end{equation}{}

Que é um bloco de $1^a$ ordem.

Adicionalmente, ainda no módulo dinâmico, há uma divisão de tensão entre $Z_1$ e $Z_2$. Essa divisão, lembrando que $Z_1=2.2k\Omega$ e $Z_2$ é a impedância de um capacitor de $1\mu F$, pode ser modelada da seguinte forma:

\begin{equation}
FT_{divisor}=\frac{Z_2}{Z_1+Z_2}=\frac{\frac{1}{sC}}{\frac{1}{sC}+R_1}
\end{equation}

\begin{equation}
FT_{divisor}=\frac{\frac{1}{R_2C}}{s+\frac{1}{R_2C}}=\frac{454.54}{s+454.54}
\end{equation}

Por fim, a FT do motor-tacômetro é obtida utilizando-se os valores de K e T:

\begin{equation}
FT_{sistema}=\frac{K}{1+sT}|_{K=0.957, T=24.8ms}=\frac{38.59}{s+40.32}
\end{equation}


\paragraph{}
O sistema foi modelado a partir da \emph{toolbox} Simulink do software Matlab e sua simulação é mostrada abaixo:


\begin{figure*}[ht]
\centering
\includegraphics[scale = 0.6]{images/matlab_diagram.png}
\caption{Diagrama de Blocos do sistema de 3a ordem.}
\label{fig:blockdiagram-3a-ordem}
\end{figure*}


\begin{figure}[H]
\centering
\includegraphics[width = 0.4\textwidth, height=6cm,keepaspectratio]{images/matlab_resposta_2.png}
\caption{Resposta ao sistema de 3a ordem obtida em simulação.}
\label{fig:3a-ordem-resposta-simulada}
\end{figure}

Novamente, com o matlab a função de transferência é calculada:

\begin{equation}
\begin{array}{l}
    \boxed{F.T.
    = \frac{46.32 s + 2.316e04}{  0.02641 s^3 + 3.465 s^2 + 197.7 s + 2.536e04}} \\
    \mbox{(F.T. para malha fechada de 3ª ordem)}
\end{array}
\end{equation}


e os polos:

\begin{equation}
\begin{array}{l}
    \boxed{p =    1.0e+02 * 
    \left[ \begin{array}{l}
  -1.3033          \\
  -0.0045 + 0.8584i \\
  -0.0045 - 0.8584i \\
    \end{array} \right]
} \\
    \mbox{(polos para malha fechada de 3ª ordem)}
\end{array}
\end{equation}

Vemos que um dos polos possui parte real  mais negativa que os outros dois. Isso faz com que sua resposta caia muito mais rapidamente que a resposta aos outros dois polos e seu efeito não é facilmente observável. Mesmo assim a resposta do sistema é altereada devido ao terceiro polo.

Notamos que a simulação obtida difere um pouco do comportamento observado em laboratório, mas que, em ambos os casos, a adição do terceiro polo aumenta a oscilação do sistema.








