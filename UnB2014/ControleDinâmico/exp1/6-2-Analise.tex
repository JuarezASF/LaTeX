\subsection{2º Teste}
\paragraph{}
O motor possui um efeito de zona morta que pode ser explicado da seguinte forma: 

\begin{itemize}
\item Para baixas tensões, é necessária uma tensão mínima para que ele comece a girar.
\item Para altas tensões, há uma tensão máxima a partir da qual a velocidade do motor para de acompanhar a tensão de entrada devido às limitações físicas do motor devidas à construção do mesmo e outros fatores, como o atrito viscoso em seu interior.
\end{itemize}

\paragraph{}
Isso se traduz no ceifamento da tensão de saída do tacômetro para baixas e altas tensões de entrada.