\subsection{4º Teste}

Para o caso com ganho, que gerou uma resposta sub-amortecida, podemos calcular o fator de amortecimento $\xi$ e a frequência natural $w_n$:

\begin{equation}
\begin{array}{l}
    \xi = \frac{-ln(O.S.)}{\sqrt{\pi^2 + \ln^2(O.S.)}} \\
    \xi = \frac{-ln(0.18681)}{\sqrt{\pi^2 + \ln^2(0.18681)}} \\
    \boxed{\xi = 0.471}
\end{array}
\end{equation}

\begin{equation}
\begin{array}{l}
    w_n = \frac{\pi}{T_p \sqrt{1 - \xi^2}} \\
    w_n = \frac{\pi}{0.052 \sqrt{1 - 0.471^2}}\\
    \boxed{w_n = 68.488 \mbox{  rad/sec}}
\end{array}
\end{equation}

Para o caso em que a resposta foi sobre-amortecida, podemos aproximar o sistema de segunda ordem por um de primeira. Um sistema de primeira ordem está caracterizado pela sua constante de tempo. Para esse sistema temos então:

\begin{equation}
    \boxed{\tau = 29 ms}
\end{equation}

Com a ajuda da rotina feedback() do matlab a função de transferência é calculada:

\begin{equation}
\begin{array}{l}
    \boxed{F.T.
    = \frac{10.53}{4.96e-5 s^2 + 0.0268 s + 11.53}} \\
    \mbox{(F.T. para malha fechada de 2ª ordem com G = 11)}
\end{array}
\end{equation}

com o comando pole(), também do matlab, obtém-se os polos da função

\begin{equation}
\begin{array}{l}
    \boxed{p = 1.0e+02( -2.7016 \pm 3.9926i)
} \\
    \mbox{(polos para malha fechada de 2ª ordem com G = 11)}
\end{array}
\end{equation}

