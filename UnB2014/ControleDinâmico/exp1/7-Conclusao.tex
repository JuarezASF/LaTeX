\section{Conclusão}
O experimento permitiu determinar diversas constantes envolvidas na planta motor-tacômetro. Gradativamente, componentes foram adicionados à malha de controle e pôde-se observar a validação das previsões teóricas no cálculo das constantes de tempo e ganho devido a cada componente adicionado. No sistema de segunda ordem pôde-se observar diferentes tipos de resposta a medida que um ganho em malha direta foi adicionado. Para esse sistema calculou-se a fração de amortecimento e a frequência natural. Ao final observou-se experimentalmente um sistema de terceira ordem e o aumento na oscilação do sistema devido à um polo extra. Na análise computacional envolvida, o Matlab foi utilizado para calcular as funções de transferência e polos de cada sistema estudado e, por fim, simular o sistema de terceira ordem.