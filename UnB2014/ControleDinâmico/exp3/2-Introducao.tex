\section{Introdução}

\subsection{O Sistema}
No último experimento determinamos a função de transferência do kit levitação magnética. A seguir mostramos o sistema a ser compensado.

\begin{figure}[H]
\centering
\includegraphics[scale=0.60]{images/sistema.png}
\caption{Modelo simplificado do sistema físico a ser compensado: notar que a realimentação é positiva}
\label{fig:avanco-fisico}
\end{figure}

onde já obtemos:
\begin{equation}
\left\{
   \begin{array}{l}
    K_{ifsr} = 69.7731\\
    w_r = 14.067 rad/s  
   \end{array}
   \right.
\end{equation}

\subsection{Implementação Física do Compensador de Avanço}
Na figura a seguir vemos a implementação física do compensador de avanço.

\begin{figure}[H]
\centering
\includegraphics[scale=0.60]{images/lead_compensator.jpg}
\caption{Implementação Física do Compensador de Avanço}
\label{fig:avanco-fisico}
\end{figure}

A sua função de transferência é dada por;

\begin{equation}
    G_c(s) = k\frac{s + \frac{1}{R_1 C}}{s + \frac{R_1 + R_2}{R_1 R_2 C}}, k = \frac{1}{R_1^2C^2}
\end{equation}

de onde tiramos:

\begin{equation}
\left\{
   \begin{array}{l}
    z_c = -\frac{1}{R_1C}  \\
    p_c = -\frac{R_1 + R_2}{R_1R_2C}  
   \end{array}
   \right.
\end{equation}

A função do compensador de avanço é alterar o LGR da planta ao acrescentar a fase desejada para que este passe por um ponto especificado do plano complexo. De forma mais prática, o compensador de avanço permite alterar a resposta dinâmica do sistema, isto é, melhorar o sobrevalor percentual e o tempo de acomodação.

\subsection{Compensadores}

Compensadores são estruturas adicionadas a malha de controle do sistema para melhorar a resposta obtida em malha fechada. Em geral o compensador é acrescentado na malha direta na parte de baixa potência dos sinais e pode corrigir os resultados do sistema não-compensado de duas formas: melhorando o erro estático ou melhorando a resposta dinâmica. Neste experimento trataremos principalmente de 
dois tipos de compensadores: o compensador proporcional derivativo (PD) e o compensador de avanço. Esses dois compensadores tem a mesma funcionalidade: corrigir a resposta dinâmica.

\subsection{Implementação Física do Compensador PD}
Na figura a seguir vemos uma configuração  capaz de gerar diversos controladores.
\begin{figure}[H]
\centering
\includegraphics[scale=0.60]{images/PD_controller.png}
\caption{Implementação Física do Compensador de Avanço}
\label{fig:avanco-fisico}
\end{figure}

O compensador PD é obtido fazendo $Z_1 = C//R_1$ e $Z_2 = R_2$
A sua função de transferência é dada por:

\begin{equation}
    G_c(s) = k(s + \frac{1}{R_1 C}), k = R_2 C
\end{equation}

de onde tiramos:

\begin{equation}
    z_c = -\frac{1}{R_1C}  \\
\end{equation}

A função de um compensador PD é a mesma do compensador de avanço. A diferença está no fato do compensador PD utilizar um componente ativo(o amplificado operacional) e o compensador de avanço utilizar somente componentes passivos(resistores, capacitores e indutores).

\subsection{Relação Polos Dominantes - Sobrevalor}
 A figura a seguir relaciona a posição no plano complexo dos polos dominantes de malha fechada com o fator de amortecimento $\xi$ do sistema.

\begin{figure}[H]
\centering
\includegraphics[scale=0.60]{images/bitola_ishihara.png}
\caption{relação polos - $\xi$}
\label{fig:bitola}
\end{figure}

Além disso, vale a fórmula que relaciona o sobrevalor percentual OS\% com o fator de amortecimento $\xi$:

\begin{equation}
   OS\% = e^{- \pi \frac{\xi}{\sqrt{1 - \xi ^2}}}
\end{equation}


    Além da compensação analógica, existe também a compensação digital. A compensação digital é feita substituindo o ganho direto e o compensador analógico por um computador, que faz um processamento do sinal em sua entrada seguindo uma equação de diferenças e retorna em sua saída o valor da equação correspondente.

    Para esse tipo de compensação, é necessário traduzir a linguagem do mundo real, tipicamente analógica, para a linguagem digital por meio do uso de conversores A/D e D/A. Esse tipo de compensação é ilustrado abaixo pelo diagrama de blocos.

\begin{figure}[H]
\centering
\includegraphics[scale=0.5]{images/digital_blockdiagram.jpg}
\caption{Diagrama de blocos de um sistema com compensação digital.}
\label{fig:digital_blocks}
\end{figure}

