\section{Dados Experimentais}
\subsection{Compensador Padrão}
Mediu-se o sobrevalor percentual para compensador padrão em:
\begin{equation}
    O.S.\%_{padrão} = 84\%
\end{equation}

\subsection{Compensadores Analógicos}

A seguir mostramos as formas de onda obtidas com os compensadores analógicos.

\begin{figure}[H]
\centering
\includegraphics[scale=0.1]{images/ganhoBaixo.jpg}
\caption{Compensador analógico com ganho baixo}
\label{fig:digital_blocks}
\end{figure}

\begin{figure}[H]
\centering
\includegraphics[scale=0.1]{images/ganhoMedio.jpg}
\caption{Compensador analógico com ganho medio}
\label{fig:digital_blocks}
\end{figure}

\begin{figure}[H]
\centering
\includegraphics[scale=0.1]{images/ganhoTopado.jpg}
\caption{Compensador analógico com ganho máximo}
\label{fig:digital_blocks}
\end{figure}

A seguir vemos o resultado obtido utilizando-se um controlador PD analógico.

\begin{figure}[H]
\centering
\includegraphics[scale=0.3]{images/PD_analogico.png}
\caption{Compensador PD analógico}
\label{fig:digital_blocks}
\end{figure}

\subsection{Compensadores Digitias}

A seguir vemos os resultados obtidos com compensadores digitais.

\begin{figure}[H]
\centering
\includegraphics[scale=0.2]{images/PD_otimizado_digital.png}
\caption{Compensador PD digital otimizado}
\label{fig:digital_blocks}
\end{figure}

\begin{figure}[H]
\centering
\includegraphics[scale=0.2]{images/avanco_otimizado_digital.png}
\caption{Compensador de avanço digital otimizado}
\label{fig:digital_blocks}
\end{figure}


