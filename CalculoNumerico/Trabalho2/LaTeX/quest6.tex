\section*{Questão 6}

\paragraph{}Primeiro começamos resolvendo o sistema da questão 5.
Simples operações sobre o sistema nos dão:

\begin{equation}
\left\lbrace\begin{array}{ll}
		a_i = 2(y_i - y_{i+1}) + (D_i + D_{i+1}) \\
		b_i = -3(y_i - y_{i+1}) - (2D_i + D_{i+1})
	\end{array}\right.
	\label{eq:a_b_i}
\end{equation}

Para auxiliar os cálculo por vir rescrevemos o sistema acima para $a_{i+1}$
e $b_{i+1}$:

\begin{equation}
	\left\lbrace
	\begin{array}{ll}
		a_{i+1} = 2(y_{i+1} - y_{i+2}) + (D_{i+1} + D_{i+2}) \\
		b_{i+1} = -3(y_{i+1} - y_{i+2}) - (2D_{i+1} + D_{i+2})
	\end{array}
	\right.
	\label{eq:a_b_i+1}
\end{equation}

\paragraph{}Derivando duas vezes a expressão para $Y_i$ e $Y_{i+1}$:
\begin{equation}
\begin{array}{l}
	\frac{d^2}{ds^2}Y_i(s) = \left( 6a_i \cdot s^1 + 2b_i \right) \\
	\frac{d^2}{ds^2}Y_{i+1}(s) = \left( 6a_{i+1} \cdot s^1 + 2b_{i+1}
	\right) 
\end{array}
\end{equation}

Se fizermos $\frac{d^2}{ds^2}Y_{i+1} = \frac{d^2}{ds^2}Y_{i}$ obtemos:

\begin{equation}
6a_i + 2b_i = 2b_{i+1}
\end{equation}

Usando os valores de \ref{eq:a_b_i} e \ref{eq:a_b_i+1} temos:
\begin{equation}
\begin{array}{l}
6(y_i - y_{i+1}) + 2D_i + 4D_{i+1} = 6(y_{i+2} - y_{i+1}) - 4D_{i+1} - 2D_{i+2} \\
\Rightarrow  D_i + 4D_{i+1} + D_{i_2} = 3(y_{i+2} - y_{i})
\end{array}
\label{eq:n-2}
\end{equation}

Onde a equação obtida vale para $i = 1,2. \ldots, n-2$. 

\paragraph{}Vamos impor agora a condição de contorno 
$\frac{d^2}{ds^2}Y_{1}(0) = \frac{d^2}{ds^2}Y_{n-1}(1) = 0$.
Para o primeiro caso temos:
\begin{equation}
\begin{array}{l}
	\frac{d^2}{ds^2}Y_1(0) = \left( 6a_1 \cdot 0^1 + 2b_1 \right) = 2b_1 = 0 \\
	\Rightarrow b_1 = 3(y_2 - y_1) - (2D_1 + D_2) = 0 \\
\end{array}
\end{equation}
\begin{equation}
	\therefore 2D_1 + D_2 =  3y_2 - 3y_1
	\label{eq:1}
\end{equation}


Para o segundo:

\begin{equation}
\begin{array}{l}
	\frac{d^2}{ds^2}Y_{n-1}(1) = \left( 6a_{n-1} \cdot 1^1 + 
	2b_{n-1} \right) = 6A_{n-1} + 2b_{n-1} = 0 \\
	\Rightarrow 3a_{n-1} = b_{n-1} \\
	\Rightarrow 3(2(y_{n-1} - y_{n}) + (D_{n-1} + D_{n})) =
	-3(y_{n-1} - y_{n}) - (2D_{n-1} + D_{n})\\
	\therefore D_{n-1} + 2D_{n} = 3y_{n} - 3y_{n-1}
\end{array}
\end{equation}
\begin{equation}
	\therefore D_{n-1} + 2D_{n} = 3y_{n} - 3y_{n-1}
	\label{eq:2}
\end{equation}

\paragraph{}Temos então n - 2 equações dadas por \ref{eq:n-2} e as duas últimas
dadas por \ref{eq:1} e \ref{eq:2}. O sistema para as n variáveis $D_i$ com n equações
pode ser escrito como:

\begin{equation}
\left(
\begin{array}{llllll}
2 & 1& & & & \\
1 & 4& 1& & & \\
  & 1 & 4& 1& &  \\
&& 1 & 4& 1&  \\
 & & \ddots& \ddots&\ddots & \\
  &&& 1 & 4& 1  \\
  &  & & & 1&2  \\
\end{array} \right)
\cdot
\left(
\begin{array}{c}
D_1 \\
D_2 \\
D_3 \\
\vdots \\
D_{n-1} \\
D_{n} 
\end{array} \right)
=
\left(
\begin{array}{c}
3y_2 - 3y_1 \\
3y_3 - 3y_1 \\
3y_4 - 3y_2 \\
\vdots \\
3y_n - 3y_{n-2} \\
3y_n - 3y_{n-1} \\
\end{array} \right)
\end{equation}  

\paragraph{}Basta agora resolver o sistema e teremos todos os $D_i$'s. 
Com esses dados podemos resolver os coeficientes dos $Y_i$'s e determinar
os splines sobre os dados considerados.
