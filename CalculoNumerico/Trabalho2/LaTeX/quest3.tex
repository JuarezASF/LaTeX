\section*{Questão 3}

\paragraph{}Primeiro vamos derivar as equações de diferenças finitas de segunda ordem 
para a primeira e a segunda derivada. Para isso escrevamos a fórmula de Taylor de primeira 
, segunda e terceira ordem:

\begin{equation}
  \begin{array}{l}
      f(x_0 + h) = \frac{f^{(0)}(x_0)h^0}{0!} + 
          \frac{f^{(2)}(x_0)h^2}{2!} + \frac{f^{(2)}(\xi )h^2}{2!} \\
      f(x_0 + h) = \frac{f^{(0)}(x_0)h^0}{0!} + 
          \frac{f^{(2)}(x_0)h^2}{2!} + \frac{f^{(2)}(x_0)h^2}{2!} 
              + \frac{f^{(3)}(\xi)h^3}{3!}\\
      f(x_0 + h) = \frac{f^{(0)}(x_0)h^0}{0!} + 
          \frac{f^{(2)}(x_0)h^2}{2!} + \frac{f^{(2)}(x_0)h^2}{2!} 
              + \frac{f^{(3)}(x_0)h^3}{3!}
              +  \frac{f^{(4)}(\xi)h^4}{4!}\\

  \end{array}
\end{equation}

onde $\xi \in (x_0, h)$ nos dois casos. Vamos rescrever essas equações com a notação O(h):

\begin{equation}
  \begin{array}{l}
      f(x_0 + h) = \frac{f^{(0)}(x_0)h^0}{0!} + 
          \frac{f^{(2)}(x_0)h^2}{2!} + O(h^2) \\
      f(x_0 + h) = \frac{f^{(0)}(x_0)h^0}{0!} + 
          \frac{f^{(2)}(x_0)h^2}{2!} + \frac{f^{(2)}(x_0)h^2}{2!} 
              + O(h^3)\\
      f(x_0 + h) = \frac{f^{(0)}(x_0)h^0}{0!} + 
          \frac{f^{(2)}(x_0)h^2}{2!} + \frac{f^{(2)}(x_0)h^2}{2!} 
              + \frac{f^{(3)}(x_0)h^3}{3!}
               + O(h^4)\\
  \end{array}
\end{equation}

Usando a expressão com $O(h^4)$ e avaliando em f(x-h) e f(x+4) temos:
\begin{equation}
  f(x+h) + f(x-h) = 2f(x) + f^{(2)}h^2 + O(h^4)
\end{equation}

de onde obtemos:
\begin{equation}
  f^{(2)}(x) = \frac{f(x+h) - 2f(x) + f(x-h)}{h^2} + O(h^2)
\end{equation}

Essa é a expressão de diferenças finitas para a segunda derivada nos pontos interiores.
Fazendo $h =\triangle t$ esse método é $O(\triangle t^2)$. O mesmo procedimento
refeito com a expressão de Taylor com $O(h ^3)$ nos dá:

\begin{equation}
  f^{(1)}(x) = \frac{f(x+h) - f(x-h)}{2h} + O(h^2)
\end{equation}

Essa será a fórmula usada para a primeira derivada nos pontos interiores. Para os
pontos de fronteira teremos problemas, uma vez que $x_0-h$ e $x_n+h$ não estão tabelados
e seriam necessários para as fórmulas até aqui descritas. Procedimentos semelhantes
analisando mais pontos nos dão as seguinte fórmulas de diferenças finitas de
segunda ordem para frente:

\begin{equation}
  \begin{array}{l}
  f^{(1)}(x) = \frac{-\frac{3}{2}f(x) + 2f(x+h) -\frac{1}{2} f(x+2h}{h} + O(h^2) \\
  f^{(2)}(x) = \frac{2f(x) -5 f(x+h) +4 f(x+2h) - f(x+3h}{h} + O(h^2) \\
  \end{array}
\end{equation}

e as diferenças finitas de segunda ordem para trás:
\begin{equation}
  \begin{array}{l}
  f^{(1)}(x) = \frac{\frac{3}{2}f(x) - 2f(x-h) +\frac{1}{2} f(x-2h}{h} + O(h^2) \\
  f^{(2)}(x) = \frac{2f(x) -5 f(x-h) +4 f(x-2h) - f(x-3h}{h} + O(h^2) \\
  \end{array}
\end{equation}

Essa fórmulas serão usadas para o cálculo das derivadas primeiras e segundas nas extremidades.
O código é implementado em C e o resultado mostrado a seguir.

\FloatBarrier
\begin{table}
\parbox{.45\linewidth}
		{
		\centering
		\begin{tabular}{|l|l|}\hline
			t & $f^{(1)}(t)$ \\ \hline
					 0.1 &  47.062092 \\ \hline 
		 0.2 &  -35.654472 \\ \hline 
		 0.3 &  -5.798364 \\ \hline 
		 0.4 &  2.509927 \\ \hline 
		 0.5 &  -30.692114 \\ \hline 
		 0.6 &  17.910242 \\ \hline 
		 0.7 &  -6.173170 \\ \hline 
		 0.8 &  -12.601877 \\ \hline 
		 0.9 &  -1.512384 \\ \hline 
		 1.0 &  14.839912 \\ \hline 
		 1.1 &  22.500589 \\ \hline 
		 1.2 &  -16.399300 \\ \hline 
		 1.3 &  -10.041074 \\ \hline 
		 1.4 &  4.017377 \\ \hline 
		 1.5 &  3.913653 \\ \hline 
		 1.6 &  4.011237 \\ \hline 
		 1.7 &  10.324777 \\ \hline 
		 1.7 &  10.324777 \\ \hline 

		\end{tabular}
		\caption{Primeira derivada}
		\label{tab:quest1-X1}
		}
\parbox{.45\linewidth}
	{
	\centering
		\begin{tabular}{|l|l|}\hline
			t & $f^{(2)}(t)$ \\ \hline
					 0.1 &  -3078.619141 \\ \hline 
		 0.2 &  -827.165649 \\ \hline 
		 0.3 &  1424.287842 \\ \hline 
		 0.4 &  -1258.121948 \\ \hline 
		 0.5 &  594.081177 \\ \hline 
		 0.6 &  377.965942 \\ \hline 
		 0.7 &  -859.634155 \\ \hline 
		 0.8 &  731.060059 \\ \hline 
		 0.9 &  -509.270203 \\ \hline 
		 1.0 &  836.316101 \\ \hline 
		 1.1 &  -683.102539 \\ \hline 
		 1.2 &  -94.895279 \\ \hline 
		 1.3 &  222.059799 \\ \hline 
		 1.4 &  59.109219 \\ \hline 
		 1.5 &  -61.183701 \\ \hline 
		 1.6 &  63.135391 \\ \hline 
		 1.7 &  187.454483 \\ \hline 
		 1.7 &  187.454483 \\ \hline 

		\end{tabular}
		\caption{Segunda derivada}
		\label{tab:quest1-X2}	
	}
	

\end{table}

\FloatBarrier

\paragraph{}O ajuste escolhido como o melhor na questão 2 tem primeira derivada constante e igual a
$\alpha_1 = -1.819038$ e segunda derivada constante igual a 0. Quanto a derivada
primeira podemos dizer apenas que os valores tem mesma ordem de grandeza. Já a segunda
derivada difere totalmente. As duas aproximações levaram a valores completamente distintos
para as derivadas. Vemos que a derivada, por ser um processo local, não permite boas aproximações
quando o conjunto de dados tabelados é pequeno e o espaçamento entre eles é grande.


 
