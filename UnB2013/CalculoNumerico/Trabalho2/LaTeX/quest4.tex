\section*{Questão 4}

\paragraph{}Vamos agora estimar a integral da função que gerou esses dados por meio da regra 
do trapézio e da regra um-terço de Simpson. Na regra do trapézio aproximamos a função como linear
por partes sendo em cada subintervalo o segmento de reta que liga os extremos destes. Na regra 
de Simpson a função é aproximada pelo polinômio de Lagrange de segunda ordem para cada intervalo contendo
trincas de pontos consecutivos tabelados. Para o método de Simpson é necessário que o domínio seja dividido
em um número para de pontos, e é justamente o caso: com 17 pontos temos 16 intervalos entre pontos.
Considerando o conjuntos de dados $(t_0, y_1)$, $(t_1, y_1)$, $\ldots$, $(t_n, y_n)$ os algoritmos são:

\begin{equation}
\begin{array}{ll}
  J = h[\frac{1}{2} y_0 + y_1 + y_2 + \ldots +y_{n-1} + \frac{1}{2} y_n] &
  \mbox{(Regra do Trapézio)}
\end{array}
\label{eq:trapezio}
\end{equation} 

\begin{equation}
\begin{array}{ll}
  J = \frac{h}{3}[y_0 + 4y_1 + 2y_2 + \ldots + 2y_{n-2} +4y_{n-1} + \frac{1}{2} y_n] &
  \mbox{(Regra do $\frac{1}{3}$ de Simpson)(n par!)}
\end{array}
\label{eq:trapezio}
\end{equation} 

Onde nos dois casos $h = (t_n - t_0)/n$. Para o caso em questão n=16 $x_0 = 0.1$ e $x_{16} = 1.7$. 
Os resultados obtidos são mostrados na tabela. 

\begin{table}[!htp]
  
  \centering
  \begin{tabular}{|l|l|}\hline
      Método & Integral \\ \hline
      Trapézio &  5.855827 \\ \hline
      Simpson & 5.874593 \\ \hline
  \end{tabular}
\end{table}

Para a aproximação linear escolhida como o melhor ajuste polinomial e 
com $\alpha_0 = 5.449156$ e $\alpha_1 = -1.819038$ da tabela \ref{tab:quest1-X1}
temos a estimativa para a integral:
\begin{equation}
\begin{array}{ll}
  J = \int_{0.1}^{1.7} \alpha_0 + \alpha_1t dt = \alpha_0(1.6) + \frac{\alpha_1}{2}(1.7^2 - 0.1^2)  & \\
  J = 6.099233
 \end{array}
\label{eq:int_linear}
\end{equation} 

\paragraph{} Vemos que as três aproximações nos deram valores semelhantes. É interessante
notar como as aproximações para a integral estiveram próximas enquanto as derivadas foram 
totalmente diferentes. Evidencia-se aqui o caráter global da integral e o local da derivada.
