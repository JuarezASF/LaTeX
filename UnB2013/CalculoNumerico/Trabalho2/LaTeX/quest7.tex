\section*{Questão 7}
\paragraph{}O algoritmo descrito anteriormente é implementado em C. A função 
SplitCalculate2D e suas auxiliares exemplificam a implementação.
A resolução do sistema foi feita de 3 formas diferentes: invertendo matrizes,
usando o método de Jacobi e pelo método de Gauss-Seidel. Para o caso em estudo
os três métodos convergiram para a solução e em tempo de execução quase que instantâneo.
Para o pequeno problema sob estudo e para as máquinas atuais não houve diferença
entre a utilização de um método ou de outro.

\paragraph{}O resultado é apresentado no gráfico \ref{fig:quest7}. Notamos que, 
como esperado, o ajuste coincide com os pontos tabelados e além disso a curva 
é suave e não apresenta inclinações muito elevadas com as obtidas no método
de interpolação polinomial. 
\FloatBarrier
\begin{figure}[!htp]
	\section*{Questão 7}
\paragraph{}O algoritmo descrito anteriormente é implementado em C. A função 
SplitCalculate2D e suas auxiliares exemplificam a implementação.
A resolução do sistema foi feita de 3 formas diferentes: invertendo matrizes,
usando o método de Jacobi e pelo método de Gauss-Seidel. Para o caso em estudo
os três métodos convergiram para a solução e em tempo de execução quase que instantâneo.
Para o pequeno problema sob estudo e para as máquinas atuais não houve diferença
entre a utilização de um método ou de outro.

\paragraph{}O resultado é apresentado no gráfico \ref{fig:quest7}. Notamos que, 
como esperado, o ajuste coincide com os pontos tabelados e além disso a curva 
é suave e não apresenta inclinações muito elevadas com as obtidas no método
de interpolação polinomial. 
\FloatBarrier
\begin{figure}[!htp]
	\section*{Questão 7}
\paragraph{}O algoritmo descrito anteriormente é implementado em C. A função 
SplitCalculate2D e suas auxiliares exemplificam a implementação.
A resolução do sistema foi feita de 3 formas diferentes: invertendo matrizes,
usando o método de Jacobi e pelo método de Gauss-Seidel. Para o caso em estudo
os três métodos convergiram para a solução e em tempo de execução quase que instantâneo.
Para o pequeno problema sob estudo e para as máquinas atuais não houve diferença
entre a utilização de um método ou de outro.

\paragraph{}O resultado é apresentado no gráfico \ref{fig:quest7}. Notamos que, 
como esperado, o ajuste coincide com os pontos tabelados e além disso a curva 
é suave e não apresenta inclinações muito elevadas com as obtidas no método
de interpolação polinomial. 
\FloatBarrier
\begin{figure}[!htp]
	\section*{Questão 7}
\paragraph{}O algoritmo descrito anteriormente é implementado em C. A função 
SplitCalculate2D e suas auxiliares exemplificam a implementação.
A resolução do sistema foi feita de 3 formas diferentes: invertendo matrizes,
usando o método de Jacobi e pelo método de Gauss-Seidel. Para o caso em estudo
os três métodos convergiram para a solução e em tempo de execução quase que instantâneo.
Para o pequeno problema sob estudo e para as máquinas atuais não houve diferença
entre a utilização de um método ou de outro.

\paragraph{}O resultado é apresentado no gráfico \ref{fig:quest7}. Notamos que, 
como esperado, o ajuste coincide com os pontos tabelados e além disso a curva 
é suave e não apresenta inclinações muito elevadas com as obtidas no método
de interpolação polinomial. 
\FloatBarrier
\begin{figure}[!htp]
	\input{./graph/quest7.tex}
	\caption{Interpolação com splines}
	\label{fig:quest7}
\end{figure}
\FloatBarrier

	\caption{Interpolação com splines}
	\label{fig:quest7}
\end{figure}
\FloatBarrier

	\caption{Interpolação com splines}
	\label{fig:quest7}
\end{figure}
\FloatBarrier

	\caption{Interpolação com splines}
	\label{fig:quest7}
\end{figure}
\FloatBarrier
