\section{Códigos}

\paragraph{}Os códigos desenvolvidos estão em anexo. Todos os algoritmos foram
implementados em C usando-se apenas as bibliotecas padrões da linguagem. 
Cada questão que exigiu um código tem uma arquivo fonte chamado questX.c que pode
ser compilado com "gcc -Wall questX.c -o questX -lm". Existem algumas funções comuns
a todos esses arquivos que estão definidas em bibliotecas, são elas:
\begin{itemize}
  \item matrix-class.h : define a estrutura de matriz e operações
  básicas como criação, deleção, adição e transposição.
  \item usual-math.h : define algumas funções básicas como módulo, máx
  \item LU-fatoration.h: implementa a fatoração LU com permutação. Amplamente
  utilizada para inverter matrizes e resolver sistemas
  \item simply-gaussian-elimination.h : implementa eliminação gaussiana com permutação.
  \item inversa.h : calcula a matriz inversa como sugerido na lista 3.
  \item determinante.h: calcula o determinante como sugerido na lista 3  
  \item Jacobi-method.h: soluciona sistema por método iterativo de Jacobi
  \item Gauss-Seidel-method.h : soluciona sistema por método iterativo de Gauss-Seidel
  \item AlgebricSystem.h: implementa método vetorial de Newton para 2 dimensões
  \item DataFitting.h : funções de interpolação polinomial 
\end{itemize}

\paragraph{}Essas bibliotecas são comuns a todas as questões e foram desenvolvidas
ao longo do curso. Como algumas questões usam ideias parecidas pode haver repetição
de algumas rotinas em alguns .c. Isso ocorre pois cada .c é independente e pode
ser compilado separadamente.
