



\documentclass[journal]{IEEEtran}
\hyphenation{op-tical net-works semi-conduc-tor}
%%%%%%%%%%%%%%%%%%%%%%%%%%%%%%%%%%%%%%%%%%%%%%%%%%%%%%%%%%%%%%%%%%
%%%%%%%%%%%%%%%%%%%%%%%%%%%%%%%%%%%%%%%%%%%%%%%%%%%%%%%%%%%%%%%%%%
%%%%%%%%%%%%%%%%%%%%%%%%% PACKAGES %%%%%%%%%%%%%%%%%%%%%%%%%%%%%%%
%%%%%%%%%%%%%%%%%%%%%%%%%%%%%%%%%%%%%%%%%%%%%%%%%%%%%%%%%%%%%%%%%%
%%%%%%%%%%%%%%%%%%%%%%%%%%%%%%%%%%%%%%%%%%%%%%%%%%%%%%%%%%%%%%%%%%
\usepackage[T1]{fontenc}
\usepackage[utf8]{inputenc}
\usepackage{lmodern}
\usepackage[portuguese]{babel}
\usepackage{graphicx}
\usepackage{fancyhdr}
\usepackage{caption}
\usepackage{subcaption}
\usepackage{placeins}
\pagestyle{fancy}
\usepackage{color}
\usepackage{cancel}
\newcommand{\HRule}{\rule{\linewidth}{0.5mm}}
\usepackage{steinmetz}
\graphicspath{{../images/}}

\begin{document}

\title{
Análise Dinâmica Linear
\HRule
\\
Experimento 6 \\
Modelagem e Estabilização do Pêndulo Invertido
\HRule \\
{\normalsize \today}
}

\author{  \begin{tabular}{llr} \
    & & \\[0.05cm]
    Professor: & Henrique Cezar Ferreira & \\
    Alunos:& & \\
    & Juarez A.S.F                        & 11/0032829\\
    & Luís Henrique Vieira Amaral           & 10/0130488  
      \end{tabular}
      }


\maketitle
%@@@@@@@@@@@@@@@@@@@@@@@@@@@@@@@@@@@@@@@@@@@
%@@@@@@@@@@@@@@@@      OBJETIVOS      @@@@@@@@@@@@@@@@@@
%@@@@@@@@@@@@@@@@@@@@@@@@@@@@@@@@@@@@@@@@@@@
\section{Objetivos}
Estudar a caracterização no espaço de estados de um pêndulo invertido acoplado
à planta linear Quanser, obter uma linearização do modelo e a faixa de operação deste,
posicionar os polos do sistema adequadamente e, finalmente, atuar sobre o sistema de
modo a manter o pêndulo em equilíbrio. 
%@@@@@@@@@@@@@@@@@@@@@@@@@@@@@@@@@@@@@@@@@@@@@
%@@@@@@@@@@@@@@        INTRODUÇÃO       @@@@@@@@@@@@@@@@@@@@ 
%@@@@@@@@@@@@@@@@@@@@@@@@@@@@@@@@@@@@@@@@@@@@@
\section{Introdução Teórica}
Sejam A, B, C e D as matrizes de estado de um sistema linear
qualquer.
Sejam \textbf{x}, \textbf{y} e \textbf{u} as variáveis de estado, a 
saída e a entrada, respectivamente.
As equações de estado são:
\begin{equation}
\left\{
\begin{array}{l}
\dot{\textbf{x}} = A\textbf{x} + B\textbf{u} \\
\textbf{y} = C\textbf{x} + D\textbf{u}
\end{array}
\right.
\end{equation}
Na ausência de entrada, o sistema fica:
\begin{equation}
\left\{
\begin{array}{l}
\dot{\textbf{x}} = A\textbf{x}  \\
\textbf{y} = C\textbf{x} 
\end{array}
\right.
\end{equation}
aplicando a transformada de Laplace e supondo condições iniciais
nulas temos:
\begin{equation}
\left\{
\begin{array}{l}
s\textbf{X(s)} = A\textbf{X(s)}  \\
\textbf{Y(s)} = C\textbf{X(s)} 
\end{array}
\right.
\end{equation}
resolvendo, obtemos:
\begin{equation}
\left\{
\begin{array}{l}
\textbf{X}(s) = (sI - A)^{-1} \\
\textbf{Y(s)} = C(sI - A)^{-1} 
\end{array}
\right.
\end{equation}

Um resultado da álgebra matricial nos diz que a inversa de uma matriz
A genérica é dada por:

\begin{equation}
 A^{-1} = \frac{1}{det|A|} \cdot adjA
\end{equation}
onde adjA é a matriz adjunta de A, dada pela transposta da matriz dos 
cofatores. Aplicando esse resultado no nosso sistema temos:

\begin{equation}
\textbf{Y(s)} = C\frac{adj(sI - A)}{det(sI - A)} 
\end{equation}

mas det(sI - A) nos dá justamente o polinômio característico p(s) de 
A, que pode ser escrito como :
\begin{equation}
p(s) = (s - \lambda _1)(s - \lambda _2) \ldots (s - 
\lambda _n)
\end{equation}

onde $\lambda _n$ é o n-ésimo autovalor da matriz A(os autovalores 
podem se repetir). Dessa forma, no sistema sem entrada os candidatos
a polos do sistema são os autovalores da matriz A.

No controle de sistemas queremos muitas vezes que os polos do sistema
tenham todos parte real negativa. Veja que se isso for verdade, ao 
invertemos a transformada da Laplace, todas as exponenciais terão 
expoentes negativos e a resposta do sistema está limitada a uma
faixa finita. Caso 
contrário teremos expoentes positivos que fazem a solução ir para 
infinito.

Se em uma dada planta tivermos polos com parte real positiva, podemos 
atuar no sistema com um sinal de controle de modo a posicionar os 
polos no semiplano complexo de parte real negativa e estabilizar o 
sistema. Vamos refazer o processo anterior, mas acrescentando uma 
entrada da forma $u = -Kx$. O sistema fica:

\begin{equation}
\left\{
\begin{array}{l}
\dot{\textbf{x}} = (A - BK)\textbf{x}\\
\textbf{y} = (C - DK)\textbf{x} 
\end{array}
\right.
\end{equation}

de modo análogo ao que já fizemos, resolvemos para Y(s) no domínio
de Laplace:

\begin{equation}
 \textbf{Y}(s) = (C - DK)\cdot \frac{adj(sI - (A - BK))}{det(sI - (A 
- BK))} 
\end{equation}

agora os candidatos à polos do sistema serão dados pelos autovalores 
de $(A - BK)$. Podemos ajustar então a matriz K para que os novos
polos estejam todos no semiplano negativo. Ao fazermos isso 
garantimos a estabilidade do sistema.

A análise anterior levou em conta que o sistema físico em estudo 
era linear. Esta não é, infelizmente, a situação da maioria dos 
sistemas físicos reais. Em geral temos:

\begin{equation}
 \left\{
      \begin{array}{l}
       \dot{x} = f(x, u) \\
       y = g(x, u) \\
      \end{array}
 \right.
\end{equation}

A fórmula de Taylor de duas variáveis com condições iniciais nulas 
nos permite escrever:
\begin{equation}
 \left\{
      \begin{array}{l}
       \dot{x} \approx x \frac{\partial f}{\partial x}|_{t = 0} + 
u\frac{\partial f}{\partial u}|_{t = 0}  \\
       y \approx x \frac{\partial g}{\partial x}|_{t = 0} + 
u\frac{\partial g}{\partial u}|_{t = 0}  \\
      \end{array}
 \right.
\end{equation}

fazendo:
\begin{equation}
 \left\{
      \begin{array}{l}
A = \frac{\partial f}{\partial x}|_{t = 0} \\
B = \frac{\partial f}{\partial u}|_{t = 0} \\
C = \frac{\partial g}{\partial x}|_{t = 0} \\
D = \frac{\partial g}{\partial u}|_{t = 0} \\
\end{array}
 \right.
\end{equation}

obtemos uma aproximação linear para nosso sistema em torno da origem e
podemos aplicar a mesma análise feita anteriormente. É importante 
ressaltar que o procedimento se trata de uma aproximação em torno da 
origem e será válido, portanto, para pontos próximos desta. Até quão
longe a aproximação é válida depende do sistema e do grau de precisão
requerido pela aplicação. Essa faixa de operação deve ser determinada 
experimentalmente e aproximações mais complicadas utilizando uma 
ordem maior na série de Taylor podem ser necessárias para melhorar 
a precisão do modelo caso necessário.





%@@@@@@@@@@@@@@@@@@@@@@@@@@@@@@@@@@@@@@@@@@@
%@@@@@@@@@@@@       PROCEDIMENTOS        @@@@@@@@@@@@@@@@@@
%@@@@@@@@@@@@@@@@@@@@@@@@@@@@@@@@@@@@@@@@@@@
\section{Descrição Experimental}
\begin{itemize}
 \item 
  Inicialmente, abriu-se o modelo s\_sip\_freefal\_ss\_vs\_eom.mdl no 
simulink no qual havia dois subsistemas em paralelo que representavam 
as equações lineares no espaço de estados e as equações não lineares 
do pêndulo invertido. 

  \item
  O programa script\_setup\_lab\_ip01\_2\_sip.m foi executado e os 
parâmetros do modelo foram carregados no software Simulink. Em 
seguida definiu-se a variável  IC\_ALPHA0 = deg2rad(0,1). Então, 
configurou-se o tempo de simulação para 3 segundos e o sistema foi 
simulado.


\item
Após a simulação, abriu-se os dois scopes referentes à 
posição linear do carro e posição angular do pendulo e procurou-se 
determinar a região em que a aproximação linear foi adequada ao 
comparar a diferença entre a resposta do modelo linear com do modelo 
não linear.

\item
Determinou-se os polos do sistema em malha ao calcular os
autovalores da matriz A.


\item 
Já na segunda parte do experimento, abriu-se no Simulink o 
modelo  s\_sip\_lqr\_ip02.mdl que possuía o modelo de controle da 
planta para três diferentes entradas.
Assim, executou-se, novamente o script setup\_lab\_ip01\_2\_sip.m 
para carregar os parâmetros e variáveis do sistema.

\item
Definiu-se os polos do sistema para serem:
\begin{displaymath}
\left(
 \begin{array}{l}
-12.7349 + 22.6058i \\
-12.7349 - 22.6058i \\
-1.551 + 1.0488i \\
-1.551 - 1.0488i
 \end{array}
\right)
\end{displaymath}

e com auxílio do comando place(A, B, P), determinou-se a matriz de 
ganho K para posicionar os polos no semiplano negativo. 

\item Determinou-se então os polos do novo sistema ao calcular os
autovalores da matriz A - BK.

\item
Então, posicionou-se o carro no meio do trilho, deixando o 
pêndulo para baixo e o sistema foi inicializado. Rotacionou-se, 
manualmente o pêndulo até que ele atingisse a posição vertical quando 
ocorreu o acionamento, automático, do sistema de controle.

\item
Com o sistema estabilizado, perturbou-se a haste do pêndulo e 
observou-se a resposta.

\item
Em seguida, com o sistema ligado, alterou-se a entrada do 
sistema para um onda quadrada.

\item
Por fim a entrada 3 foi selecionada 
, a bola posicionada no centro do trilho e deixada se movimentar.
\end{itemize}

%@@@@@@@@@@@@@@@@@@@@@@@@@@@@@@@@@@@@@@@@@@@@@@@@
%@@@@@@@@@@@@@@@@@@@       DADOS      @@@@@@@@@@@
%@@@@@@@@@@@@@@@@@@@@@@@@@@@@@@@@@@@@@@@@@@@@@@@@
\section{Resultados}

A matriz K obtida foi:
\begin{equation}
 K = \left[
 \begin{array}{llll}
   -40.8241  &149.4337&  -42.3754  & 13.5353
 \end{array}
 \right]
\end{equation}

Os novos polos obtidos foram:

\begin{equation}
eig(A - B\cdot K) = \left[
 \begin{array}{l}
 -12.7349 +22.6058i \\
 -12.7349 -22.6058i \\
  -1.5551 + 1.0488i \\
  -1.5551 - 1.0488i
 \end{array}
 \right]
\end{equation}

Pôde-se observar ainda que o pêndulo se manteve equilibrado conforme 
esperado:

\begin{itemize}
 \item
para o primeiro sinal de entrada o 
pêndulo manteve-se estável e equilibrado na posição vertical mesmo 
com pequenos empurrões. Nessa primeira etapa notava-se trepidações 
do carrinho para manter o carro em equilíbrio. Quando os empurrões 
foram mais fortes o controle não foi suficiente e o pêndulo caiu.

\item
para o segundo sinal de entrada o carro andou de um lado para o outro
repetidamente, ao mesmo tempo em que mantinha o pêndulo na posição 
vertical.

\item
para o último sinal de entrada o carro parecia seguir a bolinha 
a medida que essa deslizava sobre a superfície. Novamente o movimento
do carrinho era feito mantendo o pêndulo equilibrado e estável.
\end{itemize}
%@@@@@@@@@@@@@@@@@@@@@@@@@@@@@@@@@@@@@@@@@@@
%@@@@@@@@@@@@@@       Análise         @@@@@@@@@@@@@@@@@@@@@@
%@@@@@@@@@@@@@@@@@@@@@@@@@@@@@@@@@@@@@@@@@@@@
\section{Discussão}
A aproximação linear utilizada para o sistema em estudo foi 
satisfatória uma vez que o controle do pêndulo na posição vertical
foi atingido com sucesso. Notamos que ela é válida para as condições 
supostas no modelo, isto é, pequenos ângulos. Quando a 
perturbação é grande e faz o ângulo aumentar demasiadamente, o 
controle se perde e o pêndulo cai. Isso revela que a faixa de operação
de validade do nosso modelo é importante, uma vez que ele se torna 
ineficiente a medida que nos afastamos dela. 

A determinação da faixa foi feita experimentalmente. Simulamos os 
dois modelos, linear e não linear, e procuramos graficamente 
a região em que os dois modelos retornavam o resultados próximos.
Para o sistema em estudo, obteve-se que:

\begin{itemize}
\item até 13º temos um erro de 1º na resposta do modelo
\item até 8º temos um erro de 0.5º na resposta do modelo
\end{itemize}



Como foi visto na introdução, se tivermos acesso às variáveis de
estado da planta, podemos adicionar uma entrada proporcional ao estado
do sistema por uma matriz K para alocarmos adequadamente os polos do 
sistema. Para determinar essa matriz de ganho K, podemos utilizar a 
função place(A, B, P) do matlab. Essa rotina determina justamente o 
ganho K para
que a matriz (A - BK) tenha os polos dados por P.


Ao modificarmos os polos do sistema mudamos a característica da 
resposta temporal. Ao garantirmos que os polos tenham todos partes
reais negativas fazemos com que a resposta temporal apresente 
sempre exponenciais de fatores negativos, o que faz a resposta 
ficar limitada a uma faixa finita de valores. Isto é, ao posicionarmos
os polos no semiplano negativo, impedimos que a resposta do sistema 
vá para infinito.


No sistema em estudo a aproximação linear foi suficiente para o 
controle de pequenos ângulos. Caso fosse necessário o controle para
ângulos maiores, o modelo teria de ser refinado e precisaríamos 
acrescentar termos de ordem mais alta na série de Taylor utilizada.
Um modelo baseado em série de Taylor de segunda ordem teria as 
equações:

\begin{equation}
 \left\{
      \begin{array}{l}
       \dot{x}(x, u) \approx xf_x + uf_u + \frac{1}{2}(x^2 
f_{xx} + 2xuf_{xu} + u^2f_{uu})  \\ \\
       y(x, u) \approx xg_x + ug_u + \frac{1}{2}(x^2 
g_{xx} + 2xug_{xu} + u^2g_{uu}) 
      \end{array}
 \right.
\end{equation}

onde $f_x$ nota a derivada parcial de f com respeito a x, e assim 
por diante.

Durante o experimento teve-se dificuldade inicialmente em fazer 
funcionar o controle. Apesar de todos os procedimentos terem sidos 
seguidos o pêndulo não se equilibrava e o sistema de controle falhava.
Depois de algumas tentativas, notou-se que as engrenagens do carrinho
da planta estava desgastadas, de modo que a roda girava em falso em 
muitas ocasiões. Ao girar em falso, a planta não correspondia ao 
modelo matemático desenvolvido, e portanto o controle utilizado não
era adequado. Mudou-se então para outra planta mais bem conservada e 
e experimento prosseguiu com sucesso. Nota-se aqui que o efeito de 
interferências não previstas no modelo podem inviabilizar a aplicação
deste.

%@@@@@@@@@@@@@@@@@@@@@@@@@@@@@@@@@@@@@@@@@@@@@
%@@@@@@@@@@@@@@       Conclusão         @@@@@@
%@@@@@@@@@@@@@@@@@@@@@@@@@@@@@@@@@@@@@@@@@@@@@

\section{Conclusão}

O experimento permitiu estudar a modelagem no espaço de estados
para o pêndulo invertido acoplado à planta linear Quanser. O sistema
foi linearizado e a região de operação foi determinada 
experimentalmente ao comparar a resposta do modelo linear com a do 
modelo 
não linear. O sinal de controle foi determinado de forma a colocar
os polos do sistema no semiplano negativo e conseguiu-se controlar
o pêndulo de forma a mantê-lo vertical com uma estabilidade 
razoável. Ao final pode-se observar o carrinho seguindo um alvo
de controle ao passo que mantinha o pêndulo equilibrado. O experimento
é um exemplo de como as diversas ferramentas de análise de sistemas
lineares podem ser utilizadas para desenvolver ferramentas de controle
complexo.
%@@@@@@@@@@@@@@@@@@@@@@@@@@@@@@@@@@@@@@@@@@@@@@
%@@@@@@@@@@@@@@       REFERÊNCIAS     @@@@@@@@@@@@@@@@@@@@@@
%@@@@@@@@@@@@@@@@@@@@@@@@@@@@@@@@@@@@@@@@@@@@@@
\begin{thebibliography}{9}    
   \bibitem{ADL-NISE}
      Nise, N.S.
      \emph{Engenharia de Sistemas de Controle}
     5ª ed.
    LTC, 2009.

   \bibitem{ADL-OGATA}
      Ogata, K.
      \emph{Moder Control Engeeniring}
     5ª ed.
    Pearson, 2010.
   
\end{thebibliography}
\end{document}


