\documentclass[journal]{IEEEtran}
\hyphenation{op-tical net-works semi-conduc-tor}

%%%%%%%%%%%%%%%%%%%%%%%%%%%%%%%%%%%%%%%%%%%%%%%%%%%%%%%%%%%%%%%%%%
%%%%%%%%%%%%%%%%%%%%%%%%%%%%%%%%%%%%%%%%%%%%%%%%%%%%%%%%%%%%%%%%%%
%%%%%%%%%%%%%%%%%%%%%%%%% PACKAGES %%%%%%%%%%%%%%%%%%%%%%%%%%%%%%%
%%%%%%%%%%%%%%%%%%%%%%%%%%%%%%%%%%%%%%%%%%%%%%%%%%%%%%%%%%%%%%%%%%
%%%%%%%%%%%%%%%%%%%%%%%%%%%%%%%%%%%%%%%%%%%%%%%%%%%%%%%%%%%%%%%%%%
\usepackage[T1]{fontenc}
\usepackage[utf8]{inputenc}
\usepackage{lmodern}
\usepackage[portuguese]{babel}
\usepackage{graphicx}
\usepackage{fancyhdr}
\usepackage{caption}
\usepackage{subcaption}
\usepackage{placeins}
\pagestyle{fancy}
\usepackage{color}
\newcommand{\HRule}{\rule{\linewidth}{0.5mm}}
\usepackage{listings}


\begin{document}
\title{ 
Detecção de Retas e Círculos com OpenCV}

\author{
		Princípios de Visão Computacional - UnB - 2º/2013 \\
		Professor : Flávio Vidal \\
		Aluno : Juarez Aires Sampaio Filho - 11/0032829 \\
		}


\maketitle
\IEEEpeerreviewmaketitle

\section{Objetivos} 
Desenvolver um código para detectar linhas e círculos em 
vídeos e em imagens.
\section{Introdução}

\IEEEPARstart{E}{m visão computacional} a detecção de formas geométricas
simples é utilizada no processamento de objetos mais complexos.
 A detecção de formas é útil quando conhecemos previamente 
a forma do objeto e em muitas aplicações isso é possível.  A detecção
de prédios e carros, por exemplo, pode ser feita facilmente se 
usarmos o fato de que suas arestas são retas e procurarmos por linhas 
na imagem.

O primeiro passa na detecção de linhas é a detecção de bordas. Isso pode
ser feito usando o algoritmo de \textbf{Canny}. Esse algoritmo usa a variação da intensidade
medida por meio de aproximações para o \textbf{gradiente} da função intensidade para 
determinar pontos de fronteiras entre duas superfícies. Após esse processo a
\textbf{transformada Hough} é aplicada para detectar linhas. Uma versão melhorada dessa
transformada pode também ser aplicada para detectar círculos.

Todas essas funções citadas já estão
implementadas no OpenCV, de modo que aplicações que envolvem detecção
de retas e círculos podem ser rapidamente desenvolvidas. Descrevemos brevemente
as funcionalidade de tais funções;
\begin{itemize}
	\item 
	\textbf{Canny}( src, detected\_edges, lowThreshold, lowThreshold*ratio, kernel\_size ) :

	salva em detected\_edges as quinas detectadas em src. Nessa imagem pontos detectados como quinas estão marcados com branco e o resto é preto. Os outros
	argumentos são parâmetros que determinam a funcionalidade da função. De particular interesse lowThreshold determina a sensibilidade da rotina.Valores baixos
	fazem o método muito sensível à ruídos e valores elevados impossibilitam a detecção. 
	
	\item \textbf{HoughLines}(dst, lines, rho, thetea, threshold, srn, stn) :
	lines é uma coleção de vetores que guarda os coeficientes das retas
	encontradas. Novamente threshold determina a sensibilidade do algoritmo.
	
	\item \textbf{HoughCircles}( src\_gray, circles, CV\_HOUGH\_GRADIENT, dp, min\_dist,param\_1, threshold, min\_r, max\_r ):
	semelhante ao anterior, mas para detecção de círculos.
		
\end{itemize}

\newpage
\section{Materiais}
O código elaborado foi feito em C++ utilizando as bibliotecas:
\begin{itemize}
\item imgproc
\item highgui
\end{itemize}
do \textbf{OpenCV} versão 2.4.6.


\section{Procedimentos}
\subsection{Detectando Formas}
\begin{itemize}
	\item linhas
		\begin{enumerate}
			\item aplicamos o método Canny
			\item aplicamos o método  HoughLines
		\end{enumerate}
	\item círculos
		\begin{enumerate}
			\item convertemos para grayscale
			\item aplicamos um filtro gaussiano para diminuir ruídos
			\item aplicamos o método  HoughCircles
		\end{enumerate}
\end{itemize}

Agora é só desenhar as formas encontradas na figura com 
as funções \textbf{line} e \textbf{circle} do OpenCV. Para
facilitar o processo de calibração dos parâmetros acrescentamos ainda trackbars para podermos variar
os parâmetros em tempo de execução.

O código é feito de forma a aceitar entradas a partir de uma
imagem ou vídeo em disco ou apartir de uma webcam 
instalada via usb.

\subsection{Testando o Código}
As imagens \ref{fig:src-lines-circles} e \ref{fig:src-triangle} são entradas no programa e 
procuram-se os parâmetros que permitem uma boa detecção. O programa também é testado
com um vídeo de um modelo de helicóptero voando sobre ambiente controlado e chão
quadriculado. Um frame do vídeo é mostrado na figura \ref{fig:src-heli}.

As imagens \ref{fig:src-lines-circles} e \ref{fig:src-lines-circles} são impressas
e coladas em superfície rígida. Elas são então mostradas para a webcam e
compara-se o resultado da detecção com o do processo inicial. A comparação 
é feita medindo-se o número de linhas detectadas nos dois casos. O processo
de tomada de medidas no procedimento com a câmera é feito 10 vezes utiliza-se
a média dos valores obtidos para efeito de comparação.
\FloatBarrier
\begin{figure}[!htp]
	\begin{subfigure}[!htp]{0.5\textwidth}
		\centering
		\includegraphics[scale = 0.4]{./../images/linesCircles.jpg}
		\caption{}
		\label{fig:src-lines-circles}
	\end{subfigure}
	\begin{subfigure}[!htp]{0.5\textwidth}
		\centering		
		\includegraphics[scale = 0.4]{./../images/triangle.jpg}
		\caption{}
		\label{fig:src-triangle}
	\end{subfigure}
	\begin{subfigure}[!htp]{0.5\textwidth}
		\centering		
		\includegraphics[scale = 0.4]{./../images/src_heli.jpg}
		\caption{}
		\label{fig:src-heli}
	\end{subfigure}
	\caption{Imagens de teste do algoritmo}
\end{figure}
\FloatBarrier	
%---------------------------------------------------
%-------------RESULTADOS----------------------------
%--------------------------------------------------

\section{Resultados}
Os resultados obtidos são mostrados a seguir nas figuras
\ref{fig:src-lines-circles}, \ref{fig:src-triangle} e \ref{fig:src-heli}. 
As duas primeiras figuras foram obtidas com parâmetro de thereshold mínimo
para o algoritmo linear de 42, enquanto o parâmetro para a terceira foi de 145.
A calibração foi feita manualmente por meio de trackbars até se atingir um resultado razoável.

No teste de comparação obtivemos a seguinte tabela:
\FloatBarrier
\begin{table}[!htp]
	\centering
	\begin{tabular}{|l|r|r|}\hline
	medida 						& baseado em imagem 					& em vídeo \\ \hline
	média de linhas em \ref{fig:src-triangle} & 24 & 62.9 \\ \hline
	média de círculos em \ref{fig:src-triangle} & 0 & 0\\ \hline
	média de linhas em \ref{fig:src-lines-circles} & 35 & 24.4\\ \hline
	média de círculos em \ref{fig:src-lines-circles} & 5 &3.9 \\ \hline	
	\end{tabular}
\end{table}
\FloatBarrier
As figuras em \ref{fig:quest3}  ilustram a detecção com câmera.
\FloatBarrier
\begin{figure}[!htp]
	\begin{subfigure}[!htp]{0.5\textwidth}
		\centering
		\includegraphics[scale = 0.4]{./../images/show_circles.jpg}
		\caption{}
		\label{fig:show-lines-circles}
	\end{subfigure}
	\begin{subfigure}[!htp]{0.5\textwidth}
		\centering		
		\includegraphics[scale = 0.4]{./../images/show_triangle.jpg}
		\caption{}
		\label{fig:show-triangle}
	\end{subfigure}
	\begin{subfigure}[!htp]{0.5\textwidth}
		\centering		
		\includegraphics[scale = 0.4]{./../images/show_heli.jpg}
		\caption{}
		\label{fig:show-heli}
	\end{subfigure}
	\caption{resultados obtidos}
	\label{fig:results}
\end{figure}
\FloatBarrier

\FloatBarrier
\begin{figure}[!htp]
\begin{tabular}{ll}
	\begin{subfigure}[!htp]{0.2\textwidth}
		\centering
		\includegraphics[scale = 0.1]{./../images/quest3T-src.jpg}
		\caption{ imagem original}
		\label{fig:quest3-src-a}
	\end{subfigure}	
&
	\begin{subfigure}[!htp]{0.2\textwidth}
		\centering		
		\includegraphics[scale = 0.1]{./../images/quest3T-show.jpg}
		\caption{detecção}
		\label{fig:quest3-show-a}
	\end{subfigure}
\\
	\begin{subfigure}[!htp]{0.2\textwidth}
		\centering
		\includegraphics[scale = 0.1]{./../images/quest3C-src.jpg}
		\caption{imagem original}
		\label{fig:quest3-src-b}
	\end{subfigure}
&
	\begin{subfigure}[!htp]{0.2\textwidth}
		\centering		
		\includegraphics[scale = 0.1]{./../images/quest3C-show.jpg}
		\caption{detecção}
		\label{fig:quest3-show-b}
	\end{subfigure}

\end{tabular}
	\caption{detectando linhas e círculos com câmera}
	\label{fig:quest3}
\end{figure}
\FloatBarrier

%---------------------------------------------------
%-------------CONCLUSÃO-----------------------------
%--------------------------------------------------
\newpage
\section{Discussão}

Nos resultados da figura \ref{fig:results} vemos a eficiência e as limitações da detecção.
\begin{itemize}


\item Em \ref{fig:show-lines-circles} vemos que:
		\begin{enumerate}
			\item  o círculo no canto esquerdo superior que está cortado
						foi detectado.
			\item o círculo cortado na parte inferior não foi detectado
			\item todos os círculos inteiros foram detectados. 
		\end{enumerate}	
			Temos uma boa garantia portanto que círculos bem definidos serão
		detectados, já quanto a completação de círculos parcialmente ocultos 
		não podemos ter certeza

\item Em \ref{fig:show-triangle} vemos que:
		\begin{enumerate}
			\item linhas compridas e bem definidas foram detectadas
			\item pequenas linhas não foram detectadas
		\end{enumerate}	
				Vemos que o algoritmo trabalha bem com a detecção de linhas bem definidas
			mas que tem problemas com pequenas linhas. Provavelmente isso se deve ao fato 
			de que o algoritmo Canny aplica um filtro para reduzir ruídos na imagem. Dessa forma
			pequenas linhas são identificadas como ruído.

\item Em \ref{fig:show-heli} vemos:
		\begin{enumerate}
			\item novamente temos linhas que não foram detectadas, mas que as todas
			as linhas mais bem definidas foram
		\end{enumerate}	
			Assim como nos anteriores, notamos que o \textbf{algoritmo não é ideal mas que lida
		bem com a maioria dos casos}. Em uma aplicação de localização de helicóptero 
		por visão seria necessário reforçar os traço das linhas no chão e controlar
		a iluminação para melhores
		resultados. Vale ressaltar que poderíamos ter diminuído o threshold para o algoritmo.
		Nesse caso teríamos detectados mais linhas, mas teríamos também maior ruído.
		É necessário um processo de calibração desse parâmetro para cada aplicação.
	
\item 	Notamos uma queda apreciável na qualidade da detecção quando as mesmas 
	imagens são apresentadas a uma câmera e o processamento é feito em cima de um
	vídeo. Podemos citar algumas fontes de erro:
		\begin{enumerate}
			\item\textbf{ a iluminação ambiente} pode afetar as diferentes partes do vídeo diferentemente,
			de forma que as relações de gradiente se alteram. Se o gradiente diminui em um ponto
			pode-se ter que este, que era antes detectado como quina(e portanto tinha chance se ser
			uma reta), passa a ser detectado como ruído e não é processado.
			\item\textbf{ o formato de compactação de vídeo} muda as propriedades da imagem de moda
			que podemos ter o mesmo efeito da iluminação.
			\item se apresentarmos o padrão de forma inclinada, teremos os \textbf{círculos sendo projetados
			como elipses} e não sendo mais detectados.
			\item imperfeições no padrão de cor obtido podem fazer com que \textbf{o que era interpretado
			como uma única reta seja quebrado em várias}. É isso o que acontece em \ref{fig:show-triangle}, onde temos um aumento no número de retas detectadas.

		\end{enumerate}		
\item 		é notável que os parâmetros de threshold para as imagens utilizadas e para o vídeo
			do modelo de helicóptero são muito distintos. Ressalta-se aqui a \textbf{importância do processo
			de calibração} nas aplicações com processamento de imagem. Um mesmo algoritmo pode
			precisar de parâmetros bem distintos dependendo da aplicação que se tem em mente.
	\end{itemize}
	
	Notamos nos resultados características comuns no trabalho com visão computacional:
	os algoritmos são sensíveis às condições de iluminação e os parâmetros são extremamente
	específicos para cada aplicação. Apesar dessas dificuldades presentes nota-se que resultados
	razoáveis podem ser obtidos rapidamente. Apesar de detecção não ser perfeita, o modelo
	de helicóptero do vídeo testado poderia utilizar a detecção de linhas no chão para se posicionar
	em um sistema de coordenadas apropriado. 
	
	\section{Conclusão}
		Foi possível utilizar as bibliotecas do OpenCV para desenvolver rapidamente um código 
	capaz de detectar linhas e círculos em imagens e em vídeos. Os resultados obtidos 
	mostram que os algoritmos utilizados não são perfeitos mas que funcionam adequadamente
	quando os objetos a serem detectados estão bem definidos. Notou-se ainda a importância
	da calibração dos parâmetros dos métodos para cada aplicação e a influência de condições
	internas e externas no processamento de vídeos obtidos por câmeras. O projeto reforça a
	ideia de que algoritmos de visão computacional são dificilmente robustos às condições do 
	meio ambiente e que melhores resultados serão atingidos se for possível controlar
	parâmetros de cena como a iluminação. 

\begin{thebibliography}{1}

\bibitem{livroPVC}
Forsyth, D.A. , \emph{Computer Vision: a Modern Approach}, 1ªed.

\bibitem{docsOpenCV}
 Documentação do OpenCV, \emph{Tutoriais em Processamento de Imagem:Canny Edge Detector, Hough Line Transform e
 Hough Circle Transform}
 Disponível em: http://docs.opencv.org/doc/tutorials/imgproc/table\_of
 
 \_content\_imgproc/table\_of\_content\_imgproc.html
 
 \#table-of-content-imgproc
	 Acesso em 14 de Outubro de 2013.
\end{thebibliography}


\end{document}


